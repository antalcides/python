\documentclass[a4j]{jsarticle}

% Unicode文字を使用できるようにする
\usepackage[T1]{fontenc}
\usepackage[utf8]{inputenc}

% 数式の記述に用いる
\usepackage{amsmath}

% PythonTeXを利用する
\usepackage[]{pythontex}

% Pythonの関数をLaTeXのコマンドで呼び出せるようにする
% 関数は以下で定義
\newcommand{\iseven}[1]{\py{iseven(#1)}}

\title{Python \TeX のテスト}
\author{はむ吉(のんびり)}
\date{\today}

\begin{document}
\maketitle
\section{Hello, Python \TeX !}
\begin{pycode}
#!/usr/bin/env python2
# -*- coding: utf-8 -*-
from __future__ import print_function
import sys
print(r"Hello, Python \TeX !")
print(ur"こんにちは、Python \TeX !" + "\n")
print(u"Pythonのバージョンは{0}です。".format(sys.version))
\end{pycode}

\section{Python側で定義した関数を\LaTeX 側から使う}
\begin{pycode}
def iseven(n):
    if type(n) is not int:
        return u"{0}は整数ではありません。".format(n)
    elif n % 2 == 0:
        return u"{0}は偶数です。".format(n)
    elif n % 2 == 1:
        return u"{0}は奇数です。".format(n)
\end{pycode}
\begin{itemize}
\item \iseven{1024}
\item \iseven{59049}
\item \iseven{8.314}
\end{itemize}

\section{SymPyを利用する}

\begin{sympycode}
x = symbols("x")
func = x**4 + x**3 + x**2 + x + 1

print(r"\begin{align}")
for n in range(1, 5):
    deriv = Derivative(func, x, n)
    print("{0} &= {1}".format(latex(deriv), latex(deriv.doit())))
    if n < 4:
        print(r"\\")
print(r"\end{align}")
\end{sympycode}
\end{document}
