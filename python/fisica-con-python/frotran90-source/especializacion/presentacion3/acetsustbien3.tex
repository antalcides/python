\documentclass[letter]{article}
\usepackage[pdftex]{graphicx}
\usepackage[ansinew]{inputenc}
\usepackage{amsmath}
\usepackage{amsbsy}
\usepackage{amssymb}
\usepackage[spanish]{babel}
\usepackage{float}
\usepackage{fixseminar}
%\usepackage[display]{texpower}
%\usepackage[ams]{pdfslide3}
\usepackage[ams]{pdfslide3}
\newtheorem{theorem}{Teorema}
\newtheorem{lemma}[theorem]{Lema}
\newtheorem{corollary}[theorem]{Corolario}
\newtheorem{definition}[theorem]{Definici\'{o}n}
\newenvironment{proof}[1][Prueba]{\noindent\textbf{#1.} }{\ \rule{0.5em}{0.5em}}
%\newcommand{\@url}{}

\newcommand{\thankyouslide}[1][Thank You!]{\pagestyle{empty}
\mbox{}\vfill{\hfill\LARGE\scshape #1\hfill\mbox{}}\vfill{\hfill
\@bbarraza@uninorte.edu.co\hfill\mbox{}}\vfill}





%\url{http://www.yourwebpage.com/slidepresentation}

\pagestyle{title}

\begin{document}
%\convenio{Convenio}
\orgname{\color{black}Universidad Nacional de Colombia}
%\udea{Universidad del Norte}
\orgurl{\protect\color{black}http://www.unalmed.edu.co}


\title{\color{black}El Principio del M\'{a}ximo en Ecuaciones Diferenciales Parciales El\'{i}pticas y Aplicaciones}
\author{\scalebox{1}[1.3]{\emph{\textbf{\color{black}Autor: BIENVENIDO ~BARRAZA ~MARTINEZ} } }}
\director{\scalebox{1}[1.3]{\emph{\textbf{\color{black}Director:
DR.  VOLKER STALLBOHM }} }}
\address{\color{black}Trabajo presentado como requisito parcial\\para optar al t�tulo de Magister en Matem�ticas\\
%  {\realfootnotesize(para optar al t�tulo de )}\\
                           email: {\tt bbarraza@uninorte.edu.co}}
%\address{Universidad del Norte - Barranquilla, Colombia\\
%  {\realfootnotesize(Departamento de Matem�ticas y F�sica)}
%              email: {\tt bbarraza@uninorte.edu.co}}

\notes{\emph{Esta presentaci�n fue desarrollada usando pdfslide class bajo Mik\TeX{}}.
\newline
{\tiny{Copyright\copyright 2001 Dario Castro Castro. All rights
reserved. \today{}}}}


\overlay{fondo1.png} \maketitle
%%%%%%%%%%%%%%%%%%%%%ACETATO 1%%%%%%%%%%%%%%%%%%%%%%%%

\pagedissolve{Wipe /D 3 /Dm /V /M /O}%%%%%%%%%%%%%%%Efecto de transici�n de p�gina
\overlay{fondo1.png} %%%%%%%%%%%%%%%%%%%%%%%%%%%%%%%%%Fondo
\sffamily %%%%%%%%%%%%%%%%%%%%%%%%%%%%%%%%%%%%%%%%%%Tipo de letra
\Large %%%%%%%%%%%%%%%%%%%%%%%%%%%%%%%%%%%%%%%%%%%%%Tama�o de la letra
\color{black} %%%%%%%%%%%%%%%%%%%%%%%%%%%%%%%%%%%%%%Color de las letras
\headskip=20pt%%%%%%%%%%%%%%%%%%%%%%%%%%%%%%%%%%%%%%Longitud del borde superior al t�tulo
\section{\textcolor[rgb]{0.40,0.20,0.40}{RESUMEN}}
 

En este trabajo se estudian principios del m\'{a}ximo para ecuaciones
diferenciales parciales el\'{\i}pticas de segundo orden y sus aplicaciones.
%\end{document}



%%%%%%%%%%%%%%%%%%%%%ACETATO 2%%%%%%%%%%%%%%%%%%%%%%%%
\headskip=10pt \pagedissolve{Wipe /D 3 /Di /V /M /O}
\overlay{fondo1.png} \large \color{black}
\section{{Principio del  m\'{a}ximo generalizado}}



%\section{Principio del M\'{a}ximo para Soluciones D\'{e}biles}

En lo siguiente $\Omega$ es un dominio acotado en $\mathbb{R}^{n}$ con
$\partial\Omega$ de clase $C^{1}$, y se considera el operador diferencial $L $
en forma de divergencia
\begin{equation}
L=\sum_{i,j=1}^{n}D_{i}\left(  a^{ij}\left(  x\right)  D_{j}+b^{i}\left(
x\right)  \right)  +\sum_{i=1}^{n}c^{i}\left(  x\right)  D_{i}+d\left(
x\right)  ,\tag{1}%
\end{equation}
donde

\begin{description}
\item[a)] $L$ es estrictamente el\'{\i}ptico, es decir, existe $\lambda_{0}>0$
tal que
\begin{equation}
\sum_{i,j=1}^{n}a^{ij}\left(  x\right)  \xi_{i}\xi_{j}\geq\lambda
_{0}\left\vert \xi\right\vert ^{2}\text{ para todo }x\in\Omega\text{ y }\xi
\in\mathbb{R}^{n}.\tag{2}%
\end{equation}


\item[b)] Los coeficientes $a^{ij},b^{i},c^{i}$ y $d$ $\left(
i,j=1,...,n\right)  $ son funciones medibles sobre $\Omega$. Adem\'{a}s son
\textbf{acotadas}, esto es existen constantes positivas $\Lambda$ y $\rho$
tales que para todo $x\in\Omega$ y $\lambda_{0}$ en $\left(  2\right)  $ se
cumple
\begin{equation}
\sum_{i,j=1}^{n}\left\vert a^{ij}\left(  x\right)  \right\vert ^{2}\leq
\Lambda,\text{ \ }\lambda_{0}^{-2}\sum_{i=1}^{n}\left(  \left\vert
b^{i}\left(  x\right)  \right\vert ^{2}+\left\vert c^{i}\left(  x\right)
\right\vert ^{2}\right)  +\lambda^{-1}\left\vert d\left(  x\right)
\right\vert \leq\rho^{2}\tag{3}%
\end{equation}


\item[c)]
\begin{equation}
\int_{\Omega}\left(  dv-\sum_{i=1}^{n}b^{i}D_{i}v\right)  dx\leq0\text{
\ }\forall v\geq0,v\in C_{0}^{1}\left(  \Omega\right)  .\tag{4}%
\end{equation}

\end{description}

\begin{definition}
Si $u\in W^{1,2}(\Omega)$ y $f\in L^{2}\left(  \Omega\right)  $, $u$ es una
soluci\'{o}n d\'{e}bil (subsoluci\'{o}n d\'{e}bil, supersoluci\'{o}n
d\'{e}bil), de
\[
Lu=f\text{ \ \ en }\Omega,
\]
si para cada $v\in C_{0}^{1}\left(  \Omega\right)  $ con $v\geq0$ se cumple
\begin{equation}
\mathcal{L}\left(  u,v\right)  =F\left(  v\right)  \text{ \ }\left(  \leq
F\left(  v\right)  ,\geq F\left(  v\right)  \right)  ,\tag{5}%
\end{equation}
donde
\begin{equation}
\mathcal{L}\left(  u,v\right)  :=\int_{\Omega}\left\{  \sum_{i,j=1}^{n}\left(
a^{ij}D_{j}u+b^{i}u\right)  D_{i}v-\left(  \sum_{i=1}^{n}c^{i}D_{i}%
u+du\right)  v\right\}  dx.\tag{6}%
\end{equation}
$\mathcal{L}$ se denomina la \textbf{forma bilineal asociada} a $L$, y
\begin{equation}
F\left(  v\right)  :=-\int_{\Omega}fvdx\text{ \ }\forall v\in C_{0}^{1}\left(
\Omega\right)  .\tag{7}%
\end{equation}

\end{definition}


%%%%%%%%%%%%%%%%%%%%%ACETATO 34%%%%%%%%%%%%%%%%%%%%%%%%
%\headskip=10pt
\section{}
\overlay{fondo1.png}\large \color{black} \pagedissolve{Wipe /D 1
/Di /H /M /O}


\begin{theorem}
(Principio del m\'{a}ximo generalizado) Sea $L$ el operador dado en $(1),$
cuyos coeficientes satisfacen $\left( 2\right) ,\left( 3\right) $ y $\left(
4\right) $. Si $u\in W^{1,2}\left( \Omega \right) $ es una soluci\'{o}n d%
\'{e}bil de $Lu\geq 0\left( \leq 0\right) $ en $\Omega $, entonces
\begin{equation*}
\sup_{\Omega }u\leq \sup_{\partial \Omega }u^{+}\text{ \ }\left(
\inf_{\Omega }u\geq \inf_{\partial \Omega }u^{+}\right) .
\end{equation*}
\end{theorem}

%\begin{proof}
%Se define $l:=\sup\limits_{\partial \Omega }u^{+}$, y se toma $k\in \mathbb{R%
%}$ tal que $l\leq k<\sup\limits_{\Omega }u$. Entonces $v:=\left( u-k\right)
%^{+}\in W_{0}^{1,2}\left( \Omega \right) $ y $m(\mathbf{supp\ }\nabla v)\neq
%0.$ Por otro lado existe un $C>0,$ independiente de $k,$ tal que $m(\mathbf{%
%supp\ }\nabla v)>C^{-n},$ $n\geq 2,$ esto es una contradicci\'{o}n. En
%consecuencia
%\begin{equation*}
%\sup_{\Omega }u\leq l\text{.}
%\end{equation*}
%\end{proof}

\begin{corollary}
Si $u\in W_{0}^{1,2}\left( \Omega \right) $ y $Lu=0$ en $\Omega $ en el
sentido d\'{e}bil, entonces $u=0$ en $\Omega $ en el sentido d\'{e}bil.
\end{corollary}


%%%%%%%%%%%%%%%%%%%%%ACETATO 5%%%%%%%%%%%%%%%%%%%%%%%%
\headskip=10pt
\section{{Problema de Dirichlet generalizado}}
\pagedissolve{Wipe /D 1 /Di /H /M /O} \overlay{fondo1.png}
\large\color{black}

%\begin{document}
Con el principio del m\'{a}ximo generalizado (colorario) se sigue que
\linebreak existe soluci\'{o}n d\'{e}bil \'{u}nica para el problema de
Direchlet generalizado. Adicionalmente en la prueba se utilizan los siguientes
teoremas: la alternativa Fredholm, Lax-Milgram, y Rellich-Kondrackov el cual
establece que
\[
W_{0}^{1,2}\left(  \Omega\right)  \subset\subset L^{2}\left(  \Omega\right)
\text{ \ si }n>2,
\]
y $\Omega$ es abierto acotado en $\mathbb{R}^{n}$ con $\partial\Omega$ de
clase $C^{1}$.

\begin{definition}
Sean $f\in L^{2}\left(  \Omega\right)  ,$ $g\in W^{1,2}\left(  \Omega\right)
$ y $L$ el operador diferencial definido en $\left(  1\right)  $, $u\in
W^{1,2}\left(  \Omega\right)  $ es una \textbf{soluci\'{o}n d\'{e}bil} de
\[
\left\{
\begin{array}
[c]{c}%
Lu=f\text{ \ en }\Omega,\\
u=g\text{ sobre }\partial\Omega,
\end{array}
\right.
\]
si $u-g\in W_{0}^{1,2}\left(  \Omega\right)  $ y adem\'{a}s
\begin{equation}
\mathcal{L}\left(  u,v\right)  =F\left(  v\right)  =-\int_{\Omega}fvdx\text{,
}\forall v\in C_{0}^{1}\left(  \Omega\right)  ,\tag{8}%
\end{equation}
donde $\mathcal{L}$ viene dado por $\left(  6\right)  .$
\end{definition}

\begin{lemma}
Si $I:W_{0}^{1,2}\left(  \Omega\right)  \longrightarrow\left(  W_{0}%
^{1,2}\left(  \Omega\right)  \right)  ^{\ast}$, $u\longrightarrow Iu$, donde
\[
Iu\left(  v\right)  :=\int_{\Omega}uvdx\text{, }v\in W_{0}^{1,2}\left(
\Omega\right)  .
\]
Entonces $I$ es una inmersi\'{o}n compacta.
\end{lemma}

%%%%%%%%%%%%%%%%%%%%%ACETATO 6%%%%%%%%%%%%%%%%%%%%%%%%
\headskip=10pt
\section{{}}
\pagedissolve{Wipe /D 1 /Di /H /M /O} \overlay{fondo1.png}
\large\color{black}




\begin{theorem}
Sea $L$ el operador diferencial dado en $\left(  1\right)  $ con coeficientes
satisfaciendo $\left(  2\right)  ,\left(  3\right)  $ y $\left(  4\right)  $,
y $g\in W^{1,2}\left(  \Omega\right)  $. Entonces para cualquier $f\in
L^{2}\left(  \Omega\right)  $ el \textbf{problema de Dirichlet generalizado}
\[
\left(  P.D.\right)  \left\{
\begin{array}
[c]{c}%
Lu=f\text{ \ en }\Omega,\\
u=g\text{ sobre }\partial\Omega,
\end{array}
\right.
\]
tiene soluci\'{o}n d\'{e}bil \'{u}nica.
\end{theorem}

%\begin{proof}
%Si $u$ es una soluci\'{o}n d\'{e}bil de $\left(  P.D.\right)  $, entonces
%$w:=u-g\in W_{0}^{1,2}\left(  \Omega\right)  $ y adem\'{a}s para cada $v\in
%W_{0}^{1,2}\left(  \Omega\right)  :$%
%\[
%\mathcal{L}\left(  w,v\right)  =\widetilde{F}\left(  v\right)  ,\text{
%\ }\widetilde{F}\left(  v\right)  \in\left(  W_{0}^{1,2}\left(  \Omega\right)
%\right)  ^{\ast}.
%\]
%
%
%$\mathcal{L}\left(  .,.\right)  $ es una forma bilineal acotada sobre
%$W_{0}^{1,2}\left(  \Omega\right)  \times W_{0}^{1,2}\left(  \Omega\right)  $
%y existe $\sigma_{0}$ tal que
%\[
%\mathcal{L}_{\sigma_{0}}\left(  u,v\right)  :=\mathcal{L}\left(  u,v\right)
%+\sigma_{0}\left(  u,v\right)
%\]
%es coerciva. Luego Lax-Milgram implica que para $F\in\left(  W_{0}%
%^{1,2}\left(  \Omega\right)  \right)  ^{\ast}$ existe un \'{u}nico $u\in
%W_{0}^{1,2}\left(  \Omega\right)  $ tal que
%\[
%\mathcal{L}_{\sigma_{0}}\left(  u,v\right)  =F\left(  v\right)  \text{ \ para
%todo }v\in W_{0}^{1,2}\left(  \Omega\right)  .
%\]
%La aplicaci\'{o}n $L_{\sigma_{0}}:W_{0}^{1,2}\left(  \Omega\right)
%\longrightarrow\left(  W_{0}^{1,2}\left(  \Omega\right)  \right)  ^{\ast}$,
%dada por
%\[
%(L_{\sigma_{0}}u)\left(  v\right)  :=\mathcal{L}_{\sigma_{0}}\left(
%u,v\right)  \text{ \ }\forall v\in W_{0}^{1,2}\left(  \Omega\right)
%\]
%es lineal, acotada uno-uno, sobre y $L_{\sigma_{0}}^{-1}$ tambi\'{e}n es acotada.
%
%As\'{\i} las ecuaciones $Lu=F$ y $u-\sigma_{0}L_{\sigma_{0}}^{-1}%
%Iu=L_{\sigma_{0}}^{-1}F$ son equivalentes. El operador $T:=\sigma_{0}%
%L_{\sigma_{0}}^{-1}I$ es compacto, debido al lema $5$, y por el corolario $3$
%se deduce que $I-T$ es inyectiva. Luego por la alternativa de Fredholm se da
%la conclusi\'{o}n del teorema.
%\end{proof}

%%%%%%%%%%%%%%%%%%%%%ACETATO 7%%%%%%%%%%%%%%%%%%%%%%%%

\headskip=10pt
\section{{Principio del m\'{a}ximo cl\'{a}sico }}
\pagedissolve{Wipe /D 1 /Di /H /M /O} \overlay{fondo1.png}
\large\color{black}





Ahora las funci\'{o}nes $u\in C^{2}\left(  \Omega\right)  \cap C\left(
\overline{\Omega}\right)  $ y el operador $L$ es
\begin{equation}
Lu:=\sum_{i,j=1}^{n}a_{ij}\left(  x\right)  D_{ij}^{2}u+\sum_{i=1}^{n}%
b_{i}\left(  x\right)  D_{i}u+c\left(  x\right)  u,\tag{9}%
\end{equation}
donde los coeficientes $a_{ij},b_{i}$ y $c$ est\`{a}n definidos en
un abierto, no vacio, $\Omega$ en $\mathbb{R}^{n}$, $n\geq2$, y
son acotados. Adem\'{a}s $A=\left[  a_{ij}\left(  x\right) \right]
$ sim\'{e}trica $\forall x\in\Omega$ y $L$ es estrictamente
el\'{\i}ptico, $\left(  2\right)  $.

\begin{theorem}
(Principio del m\'{a}ximo d\'{e}bil para $c\leq0$) si $Lu\geq0$ $\left(
Lu\leq0\right)  $ en $\Omega$ y $c\leq0$, entonces
\[
\max_{\overline{\Omega}}u\leq\max_{\partial\Omega}u^{+}\text{ \ }\left(
\min_{\overline{\Omega}}u\geq\min_{\partial\Omega}u^{-}\right)  .
\]

\end{theorem}


%%%%%%%%%%%%%%%%%%%%%%%%%%%%%%%%%%%%%%%%%%%%%%%%%%%%%%%%%%%%%
\headskip=20pt
\section{{}}
\pagedissolve{Wipe /D 1 /Di /H /M /O} \overlay{fondo1.png}
\large\color{black}




\begin{lemma}
(de Frontera de Hopf) Sup\'{o}ngase que $\Omega$ satisface la condici\'{o}n de
bola interior en $x_{0}\in\partial\Omega$ y $c\equiv0$ en $L$ dado en $\left(
9\right)  $. Si $u\in C^{2}\left(  \Omega\right)  \cap C\left(  \Omega
\cup\left\{  x_{0}\right\}  \right)  $ satisface $Lu\geq0$ en $\Omega$ y
$u\left(  x_{0}\right)  >u\left(  x\right)  $ $\forall x\in\Omega$, entonces
\[
\lim\inf_{t\rightarrow0^{-}}\frac{u\left(  x_{0}+t\nu\right)  -u\left(
x_{0}\right)  }{t}>0\text{,}%
\]
donde $\nu$ es la normal exterior a la bola interior en $x_{0}.$

Si en las hip\'{o}tesis iniciales se cambia $c=0$ en $\Omega$ por $c\leq0$ en
$\Omega$ y $u\left(  x_{0}\right)  \geq0$, se obtiene la misma conclusi\'{o}n.

Si en las hip\'{o}tesis iniciales se cambia $c=0$ en $\Omega$ por $u\left(
x_{0}\right)  =0$ en $\Omega$, se obtiene la misma conclusi\'{o}n
(independiente del signo de $c$).
\end{lemma}


%%%%%%%%%%%%%%%%%%%%%%%%%%%%%%%%%%%%%%%%%%%%%%%%
\headskip=40pt
\section{{}}
\pagedissolve{Wipe /D 1 /Di /H /M /O} \overlay{fondo1.png}
\large\color{black}




\begin{theorem}
$\Omega$ es un dominio, no necesariamente acotado, $u\in
C^{2}\left( \Omega\right)  \cap C\left(  \overline{\Omega}\right)
$, no constante, satisface $Lu\geq0$ $\left(  Lu\leq0\right)  $ en
$\Omega$. Si $c=0$, $u$ no alcanza un m\'{a}ximo (m\'{\i}nimo) en
el interior de $\Omega$. Si $c\leq0$, $u$ no alcanza un m\'{a}ximo
no negativo (m\'{\i}nimo no positivo) en el interior de $\Omega$.
Independiente del signo de $c$, $u$ no puede ser cero en un
m\'{a}ximo (m\'{\i}nimo) interior.
\end{theorem}

%%%%%%%%%%%%%%%%%%%%%%%%%%%%%%%%%%%%%%%%%%%%%%%%%%%%%%%%%%%%%

\headskip=20pt
\section{{Aplicaci\'{o}n a la teor\'{i}a de valores propios}}
\pagedissolve{Wipe /D 1 /Di /H /M /O} \overlay{fondo1.png}
\large\color{black}




%\section{Aplicaci\'{o}n a la teor\'{\i}a de valores propios}

\begin{theorem}
Dado $\Omega$ abierto acotado, $a_{ij}\in C^{\infty}\left(  \overline{\Omega
}\right)  $, $a_{ij}=a_{ji}$, $-L$ estrictamente el\'{\i}ptico, donde
\[
Lu=-\sum_{i,j=1}^{n}\frac{\partial}{\partial x_{i}}\left(  a_{ij}\left(
x\right)  \frac{\partial}{\partial x_{j}}u\right)  ,
\]
entonces

\begin{enumerate}
\item[$a)$] El primer valor propio de $L$, $\lambda_{1},$ es positivo y
adem\'{a}s
\begin{equation}
\lambda_{1}=min\left\{  \mathcal{L}\left(  u,u\right)  :u\in
H_{0}^{1}\left( \Omega\right)  ,\text{ }\left\Vert u\right\Vert
_{L^{2}\left(  \Omega\right)
}=1\right\}  \tag{10}%
\end{equation}


\item[$b)$] El m\'{\i}nimo en $\left(  10\right)  $ es alcanzado por una
funci\'{o}n $w_{1}$ positiva en $\Omega$, que es soluci\'{o}in d\'{e}bil de
\begin{equation}
\left\{
\begin{array}
[c]{c}%
Lw_{1}=\lambda_{1}w_{1}\text{ \ en }\Omega,\\
w_{1}=0\text{ \ sobre }\partial\Omega.
\end{array}
\right.  \tag{11}%
\end{equation}

\end{enumerate}
\end{theorem}

%%%%%%%%%%%%%%%%%%%%%%%%%%%%%%%%%%%%%%%%%%%%%%%%%%%%%%

\headskip=20pt
\section{{Un principio del m\'{a}ximo donde $c$ puede ser positivo}}
\pagedissolve{Wipe /D 1 /Di /H /M /O} \overlay{fondo1.png}
\large\color{black}

Sea $\Omega $ un dominio acotado en $\mathbb{R}^{n}$, $L$ el operador
diferencial definido en $\left( 9\right) $ y $\left( S_{\Omega }\right) $ la
condici\'{o}n:
\begin{equation*}
\left( S_{\Omega }\right) \left\{
\begin{array}{c}
\text{Existen constantes }\lambda _{0}\text{ y }\Lambda _{0}\text{ tales que:%
} \\
\lambda _{0}\in \xi ^{2}\leq \xi ^{T}a\left( x\right) \xi \leq \Lambda
_{0}\xi ^{2}\text{ para todo }\xi \in \mathbb{R}^{n}, \\
\left| b_{i}\left( x\right) \right| \leq \Lambda _{0}\text{ y }-\Lambda
_{0}\leq c\left( x\right) \text{ en }\Omega \text{.}
\end{array}
\right.
\end{equation*}
Adicionalmente se considera la siguiente definici\'{o}n.

\begin{definition}
Usando la notaci\'{o}n $d\left( x\right) =dist\left( x,\partial \Omega
\right) $, $x\in \Omega $, se dice que el conjunto $\Omega $ satisface la
\textbf{condici\'{o}n de bola interior uniforme en el sentido fuerte}, si
para alg\'{u}n $r>0$, y todo $x\in \Omega $ con $d\left( x\right) \leq r$ le
corresponde un punto m\'{a}s cercano $y\in \partial \Omega $, $d\left(
x\right) =\left| x-y\right| $,con la propiedad que $B_{r}\left( z\right)
\subset \Omega $, donde $z=y+\frac{r\left( x-y\right) }{\left| x-y\right| }$.
\end{definition}

%%%%%%%%%%%%%%%%%%%%%%%%%%%%%%%%%%%%%%%%%%%%%%%%%%

\headskip=40pt
\section{}
\pagedissolve{Wipe /D 1 /Di /H /M /O} \overlay{fondo1.png}
\large\color{black}




\begin{theorem}
Sup\'{o}ngase que $\Omega$ satisface la condici\'{o}n de bola interior
uniforme en el sentido fuerte, $L$ dado por $\left(  9\right)  $, $\left(
S_{\Omega}\right)  $ se cumple, $u\in C^{2}\left(  \Omega\right)  \cap
C\left(  \overline{\Omega}\right)  $ satisfaciendo
\[
Lu\leq0\text{ \ en }\Omega\text{ \ y }u\geq0\text{ \ sobre }\partial\Omega
\]
es Lipschitz continua en $\overline{\Omega}$. Si $h\in C^{2}\left(
\Omega\right)  \cap C\left(  \overline{\Omega}\right)  $ satisface
\[
Lh\leq0\text{ \ y \ }h>0\text{ en }\Omega\text{,}%
\]
entonces
\[
u=\beta h\text{ \ en }\Omega\text{ para un }\beta<0\text{ \ \'{o} \ }%
u\equiv0\text{ \ en }\Omega\text{ \ \'{o} }u>0\text{ \ en }\Omega\text{.}%
\]

\end{theorem}

 %%%%%%%%%%%%%%%%%%%%%%%%%%%%%%%%%%%%%%%%%%%%%%

\headskip=20pt
\section{{Extensi\'{o}n del principio del m\'{a}ximo}}
\pagedissolve{Wipe /D 1 /Di /H /M /O} \overlay{fondo1.png}
\large\color{black}





Se considera el operador el\'{\i}ptico sim\'{e}trico
\begin{equation}
Lu=\sum_{i,j=1}^{n}\frac{\partial}{\partial x_{i}}\left(  a_{ij}\left(
x\right)  \frac{\partial}{\partial x_{j}}u\right)  ,\tag{12}%
\end{equation}
donde $a_{ij}\left(  x\right)  \in C^{\infty}\left(  \overline{\Omega}\right)
$ y $\partial\Omega$ es suave.

Del teorema $10$ se tiene que $\lambda_{1}$, valor propio principal de $-L$,
es positivo y existe $w_{1}>0$ tal que
\[
\left\{
\begin{array}
[c]{c}%
Lw_{1}+\lambda_{1}w_{1}=0\text{ \ en }\Omega,\\
w_{1}=0\text{ \ sobre }\partial\Omega.
\end{array}
\right.
\]
Adem\'{a}s se sabe que
\[
\left(  PMF\right)  \left\{
\begin{array}
[c]{c}%
Lu+cu\leq0\text{ \ en }\Omega\text{ y }u\geq0\text{ sobre }\partial\Omega,\\
\text{implica}\\
u\equiv0\text{ \ en }\Omega\text{ \ \'{o} \ }u>0\text{ \ en }\Omega\text{,}%
\end{array}
\right.
\]
es verdadera si $c\leq0$ y falsa si $c\equiv\lambda_{1}.$

\begin{theorem}
Dado $L$ en $\left(  12\right)  $, los coeficientes de $L$, $\Omega$ y $u$
satisfaciendo las mismas hip\'{o}tesis del teorema $12.$ Entonces $\left(
PMF\right)  $ es verdadero si $c(x)\lvertneqq\lambda_{1}$ en $\Omega.$
\end{theorem}

%\begin{proof}
%Es consecuencia del teorema $12$ tomando $h:=w_{1}.$
%\end{proof}


%%%%%%%%%%%%%%%%%%%%%%%%%%%%%%%%%%%%%%%%%%%%%

\headskip=20pt
\section{{Aplicaci\'{o}n: Estimaciones para la soluci\'{o}n de una
ecuaci\'{o}n diferencial}} \pagedissolve{Wipe /D 1 /Di /H /M /O}
\overlay{fondo1.png} \large\color{black}


\begin{theorem}
Sea $\left|  \kappa\right|  \lneqq\lambda_{1}$ ($\lambda_{1}$ valor propio
principal de $-\Delta$), y sup\'{o}ngase que \linebreak $u\in C^{2}\left(
B\right)  \cap C(\overline{B})$ Lipschitz continua en $\overline{B}$
satisface
\[
\left(  P.V.F.\right)  \left\{
\begin{array}
[c]{c}%
\Delta u+\kappa u=C_{0}\text{ \ \ en }B,\\
u=0\text{ \ \ sobre }\partial B,
\end{array}
\right.
\]
donde $B:=B_{R}\left(  0\right)  .$ Si $C_{0}\leq0,$ $-\lambda_{1}\leq
\kappa\leq0$ y existe un $\gamma>0$ tal que $u(x)\geq$ $\gamma>0$ en $B_{\rho
}\left(  0\right)  ,$ para un $0<\rho<R$, entonces existen $M>0$ y $\delta>0$
tales que
\[
\delta\left(  R-\left|  x\right|  \right)  \leq u\left(  x\right)  \leq
M\left(  M-x^{2}\right)  \text{ \ \ en }B\text{.}%
\]
\end{theorem}



%%%%%%%%%%%%%%%%%%%%%%%%%%%%%%%%%%%%%%%%%%%%%%%

\headskip=20pt
\section{{Simetr\'{i}a esf\'{e}rica y monoton\'{i}a radial de soluciones
positivas para la ecuacion de Poisson no-lineal en
$\mathbb{R}^{n}$}} \pagedissolve{Wipe /D 1 /Di /H /M /O}
\overlay{fondo1.png} \large\color{black}





\begin{lemma}
Sean $R>0,$ $u\in C^{2}\left(  \left|  x\right|  >R\right)  \cap C\left(
\left|  x\right|  \geq R\right)  ,$ $u>0$ en $\left|  x\right|  \geq R$,
tendiendo a cero en el infinito y satisfaciendo
\[
Lu\equiv\left(  \Delta+\sum_{j=1}^{n}b_{j}\left(  x\right)  \partial
_{j}+c\left(  x\right)  \right)  u\leq0\text{ \ \ en }\left|  x\right|  >R,
\]
donde $b_{j}=O\left(  \left|  x\right|  ^{1-p}\right)  $, $c\left(  x\right)
=O\left(  \left|  x\right|  ^{-p}\right)  $ en el infinito, $p>2,$ y
adem\'{a}s continuas en $\left|  x\right|  \geq R$. Entonces, existe un
$\mu>0$, tal que
\[
u\left(  x\right)  \geq\frac{\mu}{\left|  x\right|  ^{n-2}}\text{.}%
\]

\end{lemma}

%\begin{proof}
%De la funci\'{o}n comparaci\'{o}n $z=\dfrac{1}{r^{n-2}}+\dfrac{1}{r^{s}},$
%donde $r=\left\vert x\right\vert $ y \linebreak$n-2<s<n-4+p,$ y el principio
%del m\'{a}ximo d\'{e}bil se obtiene la conclusi\'{o}n.
%\end{proof}

%%%%%%%%%%%%%%%%%%%%%%%%%%%%%%%%%%%%%%%%%%%%%
\headskip=20pt
\section{}
\pagedissolve{Wipe /D 1 /Di /H /M /O} \overlay{fondo1.png}
\large\color{black}




\begin{lemma}
Sean $u:\mathbb{R}^{n}\rightarrow\mathbb{R}$ es positiva y $C^{2}$ en
$\mathbb{R}^{n},$ $u=O(\left\vert x\right\vert ^{-m}),$ en el infinito, $m>0,$
$b:\mathbb{R}^{n}\rightarrow\mathbb{R}$ acotado y $g:\mathbb{R}\rightarrow
\mathbb{R}$ es tal que en el intervalo $0\leq u\leq u_{0}=\max u$,
$g=g_{1}+g_{2}$ con $g_{1}\in C^{1}$, $g_{2}$ continua y mon\'{o}tona no
decreciente. $u$ es soluci\'{o}n de
\begin{equation}
\Delta u+b\left(  x\right)  u_{1}+g\left(  u\right)  =0\text{ \ \ en\ }%
\mathbb{R}^{n}\text{.}\tag{13}%
\end{equation}
Sup\'{o}ngase que existe un $\lambda\in\left[  0,\infty\right)  $ tal que

\textbf{1}. $b\left(  x\right)  \geq0$ para todo $x\in\mathbb{R}^{n}$

\textbf{2}. $u_{1}\left(  x\right)  \leq0$ y $u\left(  x\right)  \leq u\left(
x^{\lambda}\right)  $ para todo $x\in\Sigma\left(  \lambda\right)  $,

\textbf{3}. $u\left(  z\right)  \neq u\left(  z^{\lambda}\right)  $ para
alg\'{u}n $z\in\Sigma\left(  \lambda\right)  $.

Entonces
\begin{align*}
u\left(  x\right)    & <u\left(  x^{\lambda}\right)  \text{ \ \ para todo
\ }x\in\Sigma\left(  \lambda\right)  \text{,}\\
u_{1}\left(  x\right)    & <0\text{ \ \ para todo \ }x\in T_{\lambda}\text{.}%
\end{align*}

\end{lemma}

%%%%%%%%%%%%%%%%%%%%%%%%%%%%%%%%%%%%%%%%%%%%%%%%%%%%%%%%%%%%%%%%%%%%%

%\headskip=20pt
%\section{} \pagedissolve{Wipe /D 1 /Di /H /M /O}
%\overlay{fondo1.png} \large\color{black}
%


\begin{proof}
Se define $v:\Sigma^{\prime}\left(  \lambda\right)  \longrightarrow\mathbb{R}$
por $v:=u\left(  x^{\lambda}\right)  $ y se obtiene que
\begin{equation}
\Delta v\left(  x\right)  -b\left(  x^{\lambda}\right)  v_{1}\left(  x\right)
+g\left(  v\left(  x\right)  \right)  =0\text{ \ \ }\forall x\in\Sigma
^{\prime}\left(  \lambda\right)  \text{.}\tag{14}%
\end{equation}
$w:=v-u$ en $\Sigma^{\prime}\left(  \lambda\right)  $ satisface $w\leq0$ y
$w\neq0$. Luego de $\left(  13\right)  $, $\left(  14\right)  $ y el principio
del m\'{a}ximo fuerte se sigue que
\[
u\left(  x\right)  <u\left(  x^{\lambda}\right)  \text{ \ \ para todo \ }%
x\in\Sigma\left(  \lambda\right)  .
\]
Como $w=0$ sobre $T_{\lambda}$, del lema de frontera de Hopf se concluye
\[
u_{1}\left(  x\right)  <0\text{ \ \ para todo \ }x\in T_{\lambda}\text{.}%
\]

\end{proof}


%%%%%%%%%%%%%%%%%%%%%%%%%%%%%%%%%%%%%%%%%%%%%

\headskip=20pt
\section{} \pagedissolve{Wipe /D 1 /Di /H /M /O}
\overlay{fondo1.png} \large\color{black}




\begin{lemma}
Si $f\in C(\mathbb{R}^{n})$ satisface $f\left(  y\right)  =O\left(  \left\vert
y\right\vert ^{-q}\right)  $ cerca al infinito, \linebreak$q>n+1$, $u\in
C^{2}$ y existe un $C>0$ tal que
\[
u\left(  x\right)  =C\int_{\mathbb{R}^{n}}\dfrac{1}{\left\vert x-y\right\vert
^{n-2}}f\left(  y\right)  dy\text{, \ \ }x\in\mathbb{R}^{n}\text{,}%
\]
entonces

\textbf{a)}
\[
\left|  x\right|  ^{n-2}u(x)\longrightarrow C\int f\left(  y\right)  dy\text{,
\ \ cuando }\left|  x\right|  \rightarrow\infty.
\]


\textbf{b)}
\[
\frac{\left|  x\right|  ^{n}}{x_{1}}u_{1}\left(  x\right)  \longrightarrow
-\left(  n-2\right)  C\int f\left(  y\right)  dy,\text{ \ cuando }\left|
x\right|  \rightarrow\infty,\text{\ con }x_{1}\rightarrow\infty\text{.}%
\]


\textbf{c) }Si $\left\{  \lambda^{i}\right\}  \subset\mathbb{R}$ con
$\lambda^{i}\longrightarrow\lambda$ y $\left\{  x^{i}\right\}  $ es una
sucesi\'{o}n en $\mathbb{R}^{n}$ tendiendo al infinito, con $x_{1}^{i}%
<\lambda^{i}$ $\ $para todo $i\in\mathbb{N}$, entonces
\[
\frac{\left\vert x^{i}\right\vert ^{n}}{\lambda^{i}-x_{1}^{i}}\left(  u\left(
x^{i}\right)  -u\left(  x^{i^{\lambda^{i}}}\right)  \right)  \longrightarrow
2\left(  n-2\right)  C\int f\left(  y\right)  \left(  \lambda-y_{1}\right)
dy.
\]

\end{lemma}


%%%%%%%%%%%%%%%%%%%%%%%%%%%%%%%%%%%%%%%%%%%%%

\headskip=20pt
\section{} \pagedissolve{Wipe /D 1 /Di /H /M /O}
\overlay{fondo1.png} \large\color{black}




\begin{theorem}
Sea $u$ soluci\'{o}n positiva y de clase $C^{2}$ de
\begin{equation}
-\Delta u=g\left(  u\right)  \text{ \ \ en }\mathbb{R}^{n}\text{, }%
n\geq3,\tag{15}%
\end{equation}
con $u\left(  x\right)  =O\left(  \left\vert x\right\vert ^{-m}\right)  $ en
el infinito, $m>0$.

Sup\'{o}ngase: \textbf{(i)} En el intervalo $0\leq u\leq u_{0}$ donde
$u_{0}=\max u$, $g=g_{1}+g_{2}$ con $g_{1}\in C^{1}$, $g_{2}$ continua y
mon\'{o}tona no decreciente. \textbf{(ii)} Para alg\'{u}n \linebreak%
$\alpha>\max\left\{  \frac{n+1}{m},\frac{2}{m}+1\right\}  $, $g\left(
u\right)  =O\left(  u^{\alpha}\right)  $ cerca de $u=0$. Entonces $u\left(
x\right)  $ es esf\'{e}ricamente sim\'{e}trica alrededor de alg\'{u}n punto de
$\mathbb{R}^{n}$ y $u_{r}<0$ para $r>0,$ donde $r$ es la coordenada radial
alrededor de ese punto. Adem\'{a}s
\begin{equation}
\lim_{\left\vert x\right\vert \rightarrow\infty}\left\vert x\right\vert
^{n-2}u\left(  x\right)  =k>0\text{.}\tag{16}%
\end{equation}

\end{theorem}


%%%%%%%%%%%%%%%%%%%%%%%%%%%%%%%%%%%%%%%%%%%%%

%\headskip=20pt
%\section{} \pagedissolve{Wipe /D 1 /Di /H /M /O}
%\overlay{fondo1.png} \large\color{black}
%



\begin{proof}
$\left(  15\right)  $ se puede escribir de la forma
\[
\Delta u+c\left(  x\right)  u=0,
\]
donde $p:=m\left(  \alpha-1\right)  >2$, y $c\left(  x\right)  =\frac{g\left(
u\left(  x\right)  \right)  }{u\left(  x\right)  }=O\left(  \left\vert
x\right\vert ^{-p}\right)  .$ Luego de los lemas $15$ y $17a)$ se deduce que
\begin{equation}
\int f\left(  y\right)  dy>0.\tag{17}%
\end{equation}
Con eso y el lema $17b)$ resulta:
\begin{equation}
\text{existe un }\widetilde{\lambda}>0\text{ \ \ tal que \ \ }u_{1}\left(
x\right)  <0\text{ para }x_{1}\geq\widetilde{\lambda}\text{.}\tag{18}%
\end{equation}
Ahora, se hace una traslaci\'{o}n hasta un punto $\mathbf{q=}\left(
q_{1},...,q_{n}\right)  $ y se fijan las coordenadas radiales con respecto a
\'{e}ste punto (nuevo origen) de modo que
\begin{equation}
\int f\left(  y\right)  y_{j}dy=0,\text{ \ \ }j=1,...,n.\tag{19}%
\end{equation}
Luego existe un $\lambda_{0}>0$ tal que para todo $\lambda\geq\lambda_{0},$
\begin{equation}
u\left(  x\right)  >u\left(  x^{\lambda}\right)  \text{ \ \ si }x_{1}%
<\lambda\text{.}\tag{20}%
\end{equation}
De $\left(  18\right)  ,$ $\left(  20\right)  $, el lema $16$ y el lema $17c)$
resulta un intervalo m\'{a}ximal $\left(  \lambda_{1},\infty\right)  ,$
$\lambda_{1}>0,$ donde $\left(  18\right)  $ y $\left(  20\right)  $ valen.
Luego
\[
u\left(  x\right)  \geq u\left(  x^{\lambda_{1}}\right)  \text{ para\ }%
x_{1}<\lambda_{1}\text{ \ \ y \ \ }u_{1}<0\text{ para }x_{1}>\lambda
_{1}\text{.}%
\]
La simetr\'{\i}a de \ $u$ con respecto al hiperplano $x_{1}=0$ se sigue el
lema $16$ y con eso el teorema queda probado.
\end{proof}


%%%%%%%%%%%%%%%%%%%%%%%%%%%%%%%%%%%%%%%%%%%%%



\newpage

\thankyouslide[!`Gracias!]


%%%%%%%%%%%%%%%%%%%%%%%%%%%%%%%%%%%%%%%%%%%%%

%%%%%%%%%%%%%%%%%%%%%%%%%%%%%%%%%%%%%%%%%%%%%%%%%%%%
\end{document}
