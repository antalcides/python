\documentclass{article}%
\usepackage{amsfonts}
\usepackage{graphicx}
\usepackage{amsmath}%
\setcounter{MaxMatrixCols}{30}%
\usepackage{amssymb}
%TCIDATA{OutputFilter=latex2.dll}
%TCIDATA{Version=4.00.0.2312}
%TCIDATA{CSTFile=articulo lateX (resaltado).cst}
%TCIDATA{Created=Friday, October 03, 2003 05:45:44}
%TCIDATA{LastRevised=Friday, October 03, 2003 05:47:58}
%TCIDATA{<META NAME="GraphicsSave" CONTENT="32">}
%TCIDATA{<META NAME="DocumentShell" CONTENT="Estilos en Espannol\Documento en blanco">}
\newtheorem{theorem}{Teorema}
\newtheorem{acknowledgement}[theorem]{Agradecimientos}
\newtheorem{algorithm}[theorem]{Algoritmo}
\newtheorem{axiom}[theorem]{Axioma}
\newtheorem{case}[theorem]{Caso}
\newtheorem{claim}[theorem]{Afirmaci\'on}
\newtheorem{conclusion}[theorem]{Conclusi\'on}
\newtheorem{condition}[theorem]{Condici\'on}
\newtheorem{conjecture}[theorem]{Conjetura}
\newtheorem{corollary}[theorem]{Corolario}
\newtheorem{criterion}[theorem]{Criterio}
\newtheorem{definition}[theorem]{Definci\'on}
\newtheorem{example}[theorem]{Ejemplo}
\newtheorem{exercise}[theorem]{Ejercicio}
\newtheorem{lemma}[theorem]{Lema}
\newtheorem{notation}[theorem]{Notaci\'on}
\newtheorem{problem}[theorem]{Problema}
\newtheorem{proposition}[theorem]{Proposici\'on}
\newtheorem{remark}[theorem]{Nota}
\newtheorem{solution}[theorem]{Soluci\'on}
\newtheorem{summary}[theorem]{Sumario}
\newenvironment{proof}[1][Prueba]{\textbf{#1.} }{\ \rule{0.5em}{0.5em}}
\begin{document}

\title{Art\'{\i}culo LaTeX Estandar}
\author{Margarita Toro\\Universidad Nacional}
\date{La fecha}
\maketitle

\begin{abstract}
Ideas acerca de un art\'{\i}culo en LaTex.

\end{abstract}



\begin{theorem}
(Principio del m\'{a}ximo generalizado) Sea $L$ el operador dado en $(1),$
cuyos coeficientes satisfacen $\left(  2\right)  ,\left(  3\right)  $ y
$\left(  4\right)  $. Si $u\in W^{1,2}\left(  \Omega\right)  $ es una
soluci\'{o}n d\'{e}bil de $Lu\geq0\left(  \leq0\right)  $ en $\Omega$,
entonces
\[
\sup_{\Omega}u\leq\sup_{\partial\Omega}u^{+}\text{ \ }\left(  \inf_{\Omega
}u\geq\inf_{\partial\Omega}u^{+}\right)  .
\]

\end{theorem}

\begin{proof}
Se define $l:=\sup\limits_{\partial\Omega}u^{+}$, y se toma $k\in\mathbb{R}$
tal que $l\leq k<\sup\limits_{\Omega}u$. Entonces $v:=\left(  u-k\right)
^{+}\in W_{0}^{1,2}\left(  \Omega\right)  $ y $m(\mathbf{supp\ }\nabla
v)\neq0.$ Por otro lado existe un $C>0,$ independiente de $k,$ tal que
$m(\mathbf{supp\ }\nabla v)>C^{-n},$ $n\geq2,$ esto es una contradicci\'{o}n.
En consecuencia
\[
\sup_{\Omega}u\leq l\text{.}%
\]

\end{proof}


\end{document}