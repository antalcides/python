



%\section{Aplicaci\'{o}n a la teor\'{\i}a de valores propios}

\begin{theorem}
Dado $\Omega$ abierto acotado, $a_{ij}\in C^{\infty}\left(  \overline{\Omega
}\right)  $, $a_{ij}=a_{ji}$, $-L$ estrictamente el\'{\i}ptico, donde
\[
Lu=-\sum_{i,j=1}^{n}\frac{\partial}{\partial x_{i}}\left(  a_{ij}\left(
x\right)  \frac{\partial}{\partial x_{j}}u\right)  ,
\]
entonces

\begin{enumerate}
\item[$a)$] El primer valor propio de $L$, $\lambda_{1},$ es positivo y
adem\'{a}s
\begin{equation}
\lambda_{1}=min\left\{  \mathcal{L}\left(  u,u\right)  :u\in
H_{0}^{1}\left( \Omega\right)  ,\text{ }\left\Vert u\right\Vert
_{L^{2}\left(  \Omega\right)
}=1\right\}  \tag{10}%
\end{equation}


\item[$b)$] El m\'{\i}nimo en $\left(  10\right)  $ es alcanzado por una
funci\'{o}n $w_{1}$ positiva en $\Omega$, que es soluci\'{o}in d\'{e}bil de
\begin{equation}
\left\{
\begin{array}
[c]{c}%
Lw_{1}=\lambda_{1}w_{1}\text{ \ en }\Omega,\\
w_{1}=0\text{ \ sobre }\partial\Omega.
\end{array}
\right.  \tag{11}%
\end{equation}

\end{enumerate}
\end{theorem}
