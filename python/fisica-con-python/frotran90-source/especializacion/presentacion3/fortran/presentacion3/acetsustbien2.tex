\documentclass[letter]{article}
\usepackage[pdftex]{graphicx}
\usepackage[ansinew]{inputenc}
\usepackage{amsmath}
\usepackage{amsbsy}
\usepackage{amssymb}
\usepackage[spanish]{babel}
\usepackage{float}
\usepackage{fixseminar}
%\usepackage[display]{texpower}
%\usepackage[ams]{pdfslide3}
\usepackage[ams]{pdfslide3}
\newtheorem{theorem}{Teorema}
\newtheorem{lemma}[theorem]{Lema}
\newtheorem{corollary}[theorem]{Corolario}
\newtheorem{definition}[theorem]{Definici\'{o}n}
\newenvironment{proof}[1][Prueba]{\noindent\textbf{#1.} }{\ \rule{0.5em}{0.5em}}
%\newcommand{\@url}{}

\newcommand{\thankyouslide}[1][Thank You!]{\pagestyle{empty}
\mbox{}\vfill{\hfill\LARGE\scshape #1\hfill\mbox{}}\vfill{\hfill
\@bbarraza@uninorte.edu.co\hfill\mbox{}}\vfill}





%\url{http://www.yourwebpage.com/slidepresentation}

\pagestyle{title}

\begin{document}
%\convenio{Convenio}
\orgname{\color{black}Universidad Nacional de Colombia}
%\udea{Universidad del Norte}
\orgurl{\protect\color{black}http://www.unalmed.edu.co}


\title{\color{black}El Principio del M\'{a}ximo en Ecuaciones Diferenciales Parciales El\'{i}ticas y Aplicaciones}
\author{\scalebox{1}[1.3]{\emph{\textbf{\color{black}Autor: BIENVENIDO ~BARRAZA ~MARTINEZ} } }}
\director{\scalebox{1}[1.3]{\emph{\textbf{\color{black}Director:
DR.  VOLKER STALLBOHM }} }}
\address{\color{black}Trabajo presentado como requisito parcial\\para optar al t�tulo de Magister en Matem�ticas\\
%  {\realfootnotesize(para optar al t�tulo de )}\\
                           email: {\tt bbarraza@uninorte.edu.co}}
%\address{Universidad del Norte - Barranquilla, Colombia\\
%  {\realfootnotesize(Departamento de Matem�ticas y F�sica)}
%              email: {\tt bbarraza@uninorte.edu.co}}

\notes{\emph{Esta presentaci�n fue desarrollada usando pdfslide class bajo Mik\TeX{}}.
\newline
{\tiny{Copyright\copyright 2001 Dario Castro Castro. All rights
reserved. \today{}}}}


\overlay{fondo1.png} \maketitle
%%%%%%%%%%%%%%%%%%%%%ACETATO 1%%%%%%%%%%%%%%%%%%%%%%%%

\pagedissolve{Wipe /D 3 /Dm /V /M /O}%%%%%%%%%%%%%%%Efecto de transici�n de p�gina
\overlay{fondo1.png} %%%%%%%%%%%%%%%%%%%%%%%%%%%%%%%%%Fondo
\sffamily %%%%%%%%%%%%%%%%%%%%%%%%%%%%%%%%%%%%%%%%%%Tipo de letra
\Large %%%%%%%%%%%%%%%%%%%%%%%%%%%%%%%%%%%%%%%%%%%%%Tama�o de la letra
\color{black} %%%%%%%%%%%%%%%%%%%%%%%%%%%%%%%%%%%%%%Color de las letras
\headskip=20pt%%%%%%%%%%%%%%%%%%%%%%%%%%%%%%%%%%%%%%Longitud del borde superior al t�tulo
\section{\textcolor[rgb]{0.40,0.20,0.40}{RESUMEN}}
 \input{acetB1.tex}


%%%%%%%%%%%%%%%%%%%%%ACETATO 2%%%%%%%%%%%%%%%%%%%%%%%%
\headskip=10pt \pagedissolve{Wipe /D 3 /Di /V /M /O}
\overlay{fondo1.png} \large \color{black}
\section{{Principio del  m\'{a}ximo generalizado}}



%\section{Principio del M\'{a}ximo para Soluciones D\'{e}biles}

En lo siguiente $\Omega$ es un dominio acotado en $\mathbb{R}^{n}$ con
$\partial\Omega$ de clase $C^{1}$, y se considera el operador diferencial $L $
en forma de divergencia
\begin{equation}
L=\sum_{i,j=1}^{n}D_{i}\left(  a^{ij}\left(  x\right)  D_{j}+b^{i}\left(
x\right)  \right)  +\sum_{i=1}^{n}c^{i}\left(  x\right)  D_{i}+d\left(
x\right)  ,\tag{1}%
\end{equation}
donde

\begin{description}
\item[a)] $L$ es estrictamente el\'{\i}ptico, es decir, existe $\lambda_{0}>0$
tal que
\begin{equation}
\sum_{i,j=1}^{n}a^{ij}\left(  x\right)  \xi_{i}\xi_{j}\geq\lambda
_{0}\left\vert \xi\right\vert ^{2}\text{ para todo }x\in\Omega\text{ y }\xi
\in\mathbb{R}^{n}.\tag{2}%
\end{equation}


\item[b)] Los coeficientes $a^{ij},b^{i},c^{i}$ y $d$ $\left(
i,j=1,...,n\right)  $ son funciones medibles sobre $\Omega$. Adem\'{a}s son
\textbf{acotadas}, esto es existen constantes positivas $\Lambda$ y $\rho$
tales que para todo $x\in\Omega$ y $\lambda_{0}$ en $\left(  2\right)  $ se
cumple
\begin{equation}
\sum_{i,j=1}^{n}\left\vert a^{ij}\left(  x\right)  \right\vert ^{2}\leq
\Lambda,\text{ \ }\lambda_{0}^{-2}\sum_{i=1}^{n}\left(  \left\vert
b^{i}\left(  x\right)  \right\vert ^{2}+\left\vert c^{i}\left(  x\right)
\right\vert ^{2}\right)  +\lambda^{-1}\left\vert d\left(  x\right)
\right\vert \leq\rho^{2}\tag{3}%
\end{equation}


\item[c)]
\begin{equation}
\int_{\Omega}\left(  dv-\sum_{i=1}^{n}b^{i}D_{i}v\right)  dx\leq0\text{
\ }\forall v\geq0,v\in C_{0}^{1}\left(  \Omega\right)  .\tag{4}%
\end{equation}

\end{description}

\begin{definition}
Si $u\in W^{1,2}(\Omega)$ y $f\in L^{2}\left(  \Omega\right)  $, $u$ es una
soluci\'{o}n d\'{e}bil (subsoluci\'{o}n d\'{e}bil, supersoluci\'{o}n
d\'{e}bil), de
\[
Lu=f\text{ \ \ en }\Omega,
\]
si para cada $v\in C_{0}^{1}\left(  \Omega\right)  $ con $v\geq0$ se cumple
\begin{equation}
\mathcal{L}\left(  u,v\right)  =F\left(  v\right)  \text{ \ }\left(  \leq
F\left(  v\right)  ,\geq F\left(  v\right)  \right)  ,\tag{5}%
\end{equation}
donde
\begin{equation}
\mathcal{L}\left(  u,v\right)  :=\int_{\Omega}\left\{  \sum_{i,j=1}^{n}\left(
a^{ij}D_{j}u+b^{i}u\right)  D_{i}v-\left(  \sum_{i=1}^{n}c^{i}D_{i}%
u+du\right)  v\right\}  dx.\tag{6}%
\end{equation}
$\mathcal{L}$ se denomina la \textbf{forma bilineal asociada} a $L$, y
\begin{equation}
F\left(  v\right)  :=-\int_{\Omega}fvdx\text{ \ }\forall v\in C_{0}^{1}\left(
\Omega\right)  .\tag{7}%
\end{equation}

\end{definition}


%%%%%%%%%%%%%%%%%%%%%ACETATO 34%%%%%%%%%%%%%%%%%%%%%%%%
\headskip=10pt
\section{}
\overlay{fondo1.png}\large \color{black} \pagedissolve{Wipe /D 1
/Di /H /M /O}


\begin{theorem}
(Principio del m\'{a}ximo generalizado) Sea $L$ el operador dado en $(1),$
cuyos coeficientes satisfacen $\left( 2\right) ,\left( 3\right) $ y $\left(
4\right) $. Si $u\in W^{1,2}\left( \Omega \right) $ es una soluci\'{o}n d%
\'{e}bil de $Lu\geq 0\left( \leq 0\right) $ en $\Omega $, entonces
\begin{equation*}
\sup_{\Omega }u\leq \sup_{\partial \Omega }u^{+}\text{ \ }\left(
\inf_{\Omega }u\geq \inf_{\partial \Omega }u^{+}\right) .
\end{equation*}
\end{theorem}

%\begin{proof}
%Se define $l:=\sup\limits_{\partial \Omega }u^{+}$, y se toma $k\in \mathbb{R%
%}$ tal que $l\leq k<\sup\limits_{\Omega }u$. Entonces $v:=\left( u-k\right)
%^{+}\in W_{0}^{1,2}\left( \Omega \right) $ y $m(\mathbf{supp\ }\nabla v)\neq
%0.$ Por otro lado existe un $C>0,$ independiente de $k,$ tal que $m(\mathbf{%
%supp\ }\nabla v)>C^{-n},$ $n\geq 2,$ esto es una contradicci\'{o}n. En
%consecuencia
%\begin{equation*}
%\sup_{\Omega }u\leq l\text{.}
%\end{equation*}
%\end{proof}

\begin{corollary}
Si $u\in W_{0}^{1,2}\left( \Omega \right) $ y $Lu=0$ en $\Omega $ en el
sentido d\'{e}bil, entonces $u=0$ en $\Omega $ en el sentido d\'{e}bil.
\end{corollary}


%%%%%%%%%%%%%%%%%%%%%ACETATO 5%%%%%%%%%%%%%%%%%%%%%%%%
\headskip=10pt
\section{{Problema de Dirichlet generalizado}}
\pagedissolve{Wipe /D 1 /Di /H /M /O} \overlay{fondo1.png}
\large\color{black}


%\section{ Problema de Dirichlet generalizado}
Con el principio del m\'{a}ximo generalizado (colorario) se sigue
que existe soluci\'{o}n d\'{e}bil \'{u}nica para el problema de
Direchlet generalizado. Adicionalmente en la prueba se utilizan
los siguientes teoremas: la alternativa Fredholm, Lax-Milgram, y
Rellich-Kondrackov el cual establece que
\[
W_{0}^{1,2}\left(  \Omega\right)  \subset\subset L^{2}\left(  \Omega\right)
\text{ \ si }n>2,
\]
y $\Omega$ es abierto acotado en $\mathbb{R}^{n}$ con $\partial\Omega$ de
clase $C^{1}$.
\begin{definition}
Sean $g$, $f\in L^{2}\left(  \Omega\right)  $ y $L$ el operador diferencial
definido en $\left(  1\right)  $, $u\in W^{1,2}\left(  \Omega\right)  $ es una
\textbf{soluci\'{o}n d\'{e}bil} de
\[
\left\{
\begin{array}
[c]{c}%
Lu=f\text{ \ en }\Omega,\\
u=g\text{ sobre }\partial\Omega,
\end{array}
\right.
\]
si $u-g\in W_{0}^{1,2}\left(  \Omega\right)  $ y adem\'{a}s
\begin{equation}
\mathcal{L}\left(  u,v\right)  =F\left(  v\right)  =-\int_{\Omega}fvdx\text{,
}\forall v\in C_{0}^{1}\left(  \Omega\right)  ,\tag{8}%
\end{equation}
donde $\mathcal{L}$ viene dado por $\left(  6\right)  .$
\end{definition}
\begin{lemma}
Si $I:W_{0}^{1,2}\left(  \Omega\right)  \longrightarrow\left(  W_{0}%
^{1,2}\left(  \Omega\right)  \right)  ^{\ast}$, $u\longrightarrow Iu$, donde
\[
Iu\left(  v\right)  :=\int_{\Omega}uvdx\text{, }v\in W_{0}^{1,2}\left(
\Omega\right)  .
\]
Entonces $I$ es una inmersi\'{o}n compacta.
\end{lemma}

%%%%%%%%%%%%%%%%%%%%%ACETATO 6%%%%%%%%%%%%%%%%%%%%%%%%
\headskip=10pt
\section{{}}
\pagedissolve{Wipe /D 1 /Di /H /M /O} \overlay{fondo1.png}
\large\color{black}




\begin{theorem}
Sea $L$ el operador diferencial dado en $\left(  1\right)  $ con coeficientes
satisfaciendo $\left(  2\right)  ,\left(  3\right)  $ y $\left(  4\right)  $,
y $g\in W^{1,2}\left(  \Omega\right)  $. Entonces para cualquier $f\in
L^{2}\left(  \Omega\right)  $ el \textbf{problema de Dirichlet generalizado}
\[
\left(  P.D.\right)  \left\{
\begin{array}
[c]{c}%
Lu=f\text{ \ en }\Omega,\\
u=g\text{ sobre }\partial\Omega,
\end{array}
\right.
\]
tiene soluci\'{o}n d\'{e}bil \'{u}nica.
\end{theorem}

%\begin{proof}
%Si $u$ es una soluci\'{o}n d\'{e}bil de $\left(  P.D.\right)  $, entonces
%$w:=u-g\in W_{0}^{1,2}\left(  \Omega\right)  $ y adem\'{a}s para cada $v\in
%W_{0}^{1,2}\left(  \Omega\right)  :$%
%\[
%\mathcal{L}\left(  w,v\right)  =\widetilde{F}\left(  v\right)  ,\text{
%\ }\widetilde{F}\left(  v\right)  \in\left(  W_{0}^{1,2}\left(  \Omega\right)
%\right)  ^{\ast}.
%\]
%
%
%$\mathcal{L}\left(  .,.\right)  $ es una forma bilineal acotada sobre
%$W_{0}^{1,2}\left(  \Omega\right)  \times W_{0}^{1,2}\left(  \Omega\right)  $
%y existe $\sigma_{0}$ tal que
%\[
%\mathcal{L}_{\sigma_{0}}\left(  u,v\right)  :=\mathcal{L}\left(  u,v\right)
%+\sigma_{0}\left(  u,v\right)
%\]
%es coerciva. Luego Lax-Milgram implica que para $F\in\left(  W_{0}%
%^{1,2}\left(  \Omega\right)  \right)  ^{\ast}$ existe un \'{u}nico $u\in
%W_{0}^{1,2}\left(  \Omega\right)  $ tal que
%\[
%\mathcal{L}_{\sigma_{0}}\left(  u,v\right)  =F\left(  v\right)  \text{ \ para
%todo }v\in W_{0}^{1,2}\left(  \Omega\right)  .
%\]
%La aplicaci\'{o}n $L_{\sigma_{0}}:W_{0}^{1,2}\left(  \Omega\right)
%\longrightarrow\left(  W_{0}^{1,2}\left(  \Omega\right)  \right)  ^{\ast}$,
%dada por
%\[
%(L_{\sigma_{0}}u)\left(  v\right)  :=\mathcal{L}_{\sigma_{0}}\left(
%u,v\right)  \text{ \ }\forall v\in W_{0}^{1,2}\left(  \Omega\right)
%\]
%es lineal, acotada uno-uno, sobre y $L_{\sigma_{0}}^{-1}$ tambi\'{e}n es acotada.
%
%As\'{\i} las ecuaciones $Lu=F$ y $u-\sigma_{0}L_{\sigma_{0}}^{-1}%
%Iu=L_{\sigma_{0}}^{-1}F$ son equivalentes. El operador $T:=\sigma_{0}%
%L_{\sigma_{0}}^{-1}I$ es compacto, debido al lema $5$, y por el corolario $3$
%se deduce que $I-T$ es inyectiva. Luego por la alternativa de Fredholm se da
%la conclusi\'{o}n del teorema.
%\end{proof}

%%%%%%%%%%%%%%%%%%%%%ACETATO 7%%%%%%%%%%%%%%%%%%%%%%%%

\headskip=10pt
\section{{Principio del m\'{a}ximo cl\'{a}sico }}
\pagedissolve{Wipe /D 1 /Di /H /M /O} \overlay{fondo1.png}
\large\color{black}





Ahora las funci\'{o}nes $u\in C^{2}\left(  \Omega\right)  \cap C\left(
\overline{\Omega}\right)  $ y el operador $L$ es
\begin{equation}
Lu:=\sum_{i,j=1}^{n}a_{ij}\left(  x\right)  D_{ij}^{2}u+\sum_{i=1}^{n}%
b_{i}\left(  x\right)  D_{i}u+c\left(  x\right)  u,\tag{9}%
\end{equation}
donde los coeficientes $a_{ij},b_{i}$ y $c$ est\`{a}n definidos en
un abierto, no vacio, $\Omega$ en $\mathbb{R}^{n}$, $n\geq2$, y
son acotados. Adem\'{a}s $A=\left[  a_{ij}\left(  x\right) \right]
$ sim\'{e}trica $\forall x\in\Omega$ y $L$ es estrictamente
el\'{\i}ptico, $\left(  2\right)  $.

\begin{theorem}
(Principio del m\'{a}ximo d\'{e}bil para $c\leq0$) si $Lu\geq0$ $\left(
Lu\leq0\right)  $ en $\Omega$ y $c\leq0$, entonces
\[
\max_{\overline{\Omega}}u\leq\max_{\partial\Omega}u^{+}\text{ \ }\left(
\min_{\overline{\Omega}}u\geq\min_{\partial\Omega}u^{-}\right)  .
\]

\end{theorem}


%%%%%%%%%%%%%%%%%%%%%%%%%%%%%%%%%%%%%%%%%%%%%%%%%%%%%%%%%%%%%
\headskip=20pt
\section{{}}
\input{acet7.tex}

%%%%%%%%%%%%%%%%%%%%%%%%%%%%%%%%%%%%%%%%%%%%%%%%
\headskip=20pt
\input{acet8.tex}
%%%%%%%%%%%%%%%%%%%%%%%%%%%%%%%%%%%%%%%%%%%%%%%%%%%%%%%%%%%%%

\headskip=20pt
\section{{}}
\input{acet9.tex}
%%%%%%%%%%%%%%%%%%%%%%%%%%%%%%%%%%%%%%%%%%%%%%%%%%%%%%

\headskip=20pt
\section{{Principio del m\'{a}ximo (m\'{i}nimo) fuerte}}
\pagedissolve{Wipe /D 1 /Di /H /M /O} \overlay{fondo1.png}
\large\color{black}
\input{acet10.tex}
%%%%%%%%%%%%%%%%%%%%%%%%%%%%%%%%%%%%%%%%%%%%%%%%%%%

\headskip=40pt
\section{{Aplicaci\'{o}n a la teor\'{i}a de valores propios}}
\pagedissolve{Wipe /D 1 /Di /H /M /O} \overlay{fondo1.png}
\large\color{black}
\input{acet11.tex}
 %%%%%%%%%%%%%%%%%%%%%%%%%%%%%%%%%%%%%%%%%%%%%%

\headskip=20pt
\section{{}}
\pagedissolve{Wipe /D 1 /Di /H /M /O} \overlay{fondo1.png}
\large\color{black}
\input{acet12.tex}

%%%%%%%%%%%%%%%%%%%%%%%%%%%%%%%%%%%%%%%%%%%%%

\headskip=20pt
\section{{Un principio del m\'{a}ximo donde $c$ puede ser positivo}}
\pagedissolve{Wipe /D 1 /Di /H /M /O} \overlay{fondo1.png}
\large\color{black}
\input{acet13.tex}


%%%%%%%%%%%%%%%%%%%%%%%%%%%%%%%%%%%%%%%%%%%%%%%

\headskip=20pt
\section{{}}
\pagedissolve{Wipe /D 1 /Di /H /M /O} \overlay{fondo1.png}
\large\color{black}
\input{acet14.tex}
%%%%%%%%%%%%%%%%%%%%%%%%%%%%%%%%%%%%%%%%%%%%%
\headskip=20pt
\section{{Aplicaci\'{o}n: Extensi\'{o}n del principio del m\'{a}ximo}}
\pagedissolve{Wipe /D 1 /Di /H /M /O} \overlay{fondo1.png}
\large\color{black}
\input{acet15.tex}
%%%%%%%%%%%%%%%%%%%%%%%%%%%%%%%%%%%%%%%%%%%%%%%%%%%%%%%%%%%%%%%%%%%%%

\headskip=20pt
\section{{Aplicaci\'{o}n: Estimaciones para la soluci\'{o}n de una
ecuaci\'{o}n diferencial}} \pagedissolve{Wipe /D 1 /Di /H /M /O}
\overlay{fondo1.png} \large\color{black}
\input{acet16.tex}

%%%%%%%%%%%%%%%%%%%%%%%%%%%%%%%%%%%%%%%%%%%%%

\headskip=20pt
\section{{Simetr\'{i}a esf\'{e}rica y monoton\'{i}a radial de soluciones
positivas para la ecuacion de Poisson no-lineal en
$\mathbb{R}^{n}$}} \pagedissolve{Wipe /D 1 /Di /H /M /O}
\overlay{fondo1.png} \large\color{black}
\input{acet17.tex}

%%%%%%%%%%%%%%%%%%%%%%%%%%%%%%%%%%%%%%%%%%%%%

\headskip=20pt
\section{{}} \pagedissolve{Wipe /D 1 /Di /H /M /O}
\overlay{fondo1.png} \large\color{black}
\input{acet18.tex}

%%%%%%%%%%%%%%%%%%%%%%%%%%%%%%%%%%%%%%%%%%%%%

\headskip=20pt
\section{{}} \pagedissolve{Wipe /D 1 /Di /H /M /O}
\overlay{fondo1.png} \large\color{black}
\input{acet19.tex}

%%%%%%%%%%%%%%%%%%%%%%%%%%%%%%%%%%%%%%%%%%%%%

\headskip=20pt
\section{{}} \pagedissolve{Wipe /D 1 /Di /H /M /O}
\overlay{fondo1.png} \large\color{black}
\input{acet20.tex}

%%%%%%%%%%%%%%%%%%%%%%%%%%%%%%%%%%%%%%%%%%%%%

\headskip=20pt
\section{{}} \pagedissolve{Wipe /D 1 /Di /H /M /O}
\overlay{fondo1.png} \large\color{black}
\input{acet21.tex}

\headskip=20pt
\section{{}} \pagedissolve{Wipe /D 1 /Di /H /M /O}
\overlay{fondo1.png} \large\color{black}
\input{acet22.tex}

\headskip=20pt
\section{{}} \pagedissolve{Wipe /D 1 /Di /H /M /O}
\overlay{fondo1.png} \large\color{black}
\input{acet23.tex}
%%%%%%%%%%%%%%%%%%%%%%%%%%%%%%%%%%%%%%%%%%%

\newpage

\thankyouslide[!`Gracias!]


%%%%%%%%%%%%%%%%%%%%%%%%%%%%%%%%%%%%%%%%%%%%%

%%%%%%%%%%%%%%%%%%%%%%%%%%%%%%%%%%%%%%%%%%%%%%%%%%%%
\end{document}
