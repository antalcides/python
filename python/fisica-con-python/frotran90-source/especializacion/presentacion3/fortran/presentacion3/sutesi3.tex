
\documentclass{article}
\usepackage{amssymb}

%%%%%%%%%%%%%%%%%%%%%%%%%%%%%%%%%%%%%%%%%%%%%%%%%%%%%%%%%%%%%%%%%%%%%%%%%%%%%%%%%%%%%%%%%%%%%%%%%%%%
\usepackage{graphicx}
\usepackage{amsmath}

%TCIDATA{OutputFilter=LATEX.DLL}
%TCIDATA{Created=Wed Oct 01 11:13:00 2003}
%TCIDATA{LastRevised=Fri Oct 03 15:47:42 2003}
%TCIDATA{<META NAME="GraphicsSave" CONTENT="32">}
%TCIDATA{<META NAME="DocumentShell" CONTENT="General\Blank Document">}
%TCIDATA{CSTFile=LaTeX article (bright).cst}

\setcounter{MaxMatrixCols}{10}
\newtheorem{theorem}{Teorema}
\newtheorem{acknowledgement}[theorem]{Acknowledgement}
\newtheorem{algorithm}[theorem]{Algorithm}
\newtheorem{axiom}[theorem]{Axiom}
\newtheorem{case}[theorem]{Case}
\newtheorem{claim}[theorem]{Claim}
\newtheorem{conclusion}[theorem]{Conclusion}
\newtheorem{condition}[theorem]{Condition}
\newtheorem{conjecture}[theorem]{Conjecture}
\newtheorem{corollary}[theorem]{Corolario}
\newtheorem{criterion}[theorem]{Criterion}
\newtheorem{definition}[theorem]{Definici\'{o}n}
\newtheorem{example}[theorem]{Example}
\newtheorem{exercise}[theorem]{Exercise}
\newtheorem{lemma}[theorem]{Lema}
\newtheorem{notation}[theorem]{Notation}
\newtheorem{problem}[theorem]{Problem}
\newtheorem{proposition}[theorem]{Proposition}
\newtheorem{remark}[theorem]{Remark}
\newtheorem{solution}[theorem]{Solution}
\newtheorem{summary}[theorem]{Summary}
\newenvironment{proof}[1][Prueba]{\textbf{#1.} }{\ \rule{0.5em}{0.5em}}
\input{tcilatex}

\begin{document}


RESUMEN: En este trabajo se estudian principios del m\'{a}ximo para
ecuaciones diferenciales parciales el\'{\i}pticas de segundo orden y sus
aplicaciones.

\section{Principio del M\'{a}ximo genearilazo}

En lo siguiente $\Omega $ es un dominio acotado en $\mathbb{R}^{n}$ con $%
\partial \Omega $ de clase $C^{1}$, y se considera el operador diferencial $%
L $ en forma de divergencia 
\begin{equation}
L=\sum_{i,j=1}^{n}D_{i}\left( a^{ij}\left( x\right) D_{j}+b^{i}\left(
x\right) \right) +\sum_{i=1}^{n}c^{i}\left( x\right) D_{i}+d\left( x\right) ,
\tag{1}
\end{equation}
donde

\begin{description}
\item[a)]  $L$ es estrictamente el\'{\i}ptico, es decir, existe $\lambda
_{0}>0$ tal que 
\begin{equation}
\sum_{i,j=1}^{n}a^{ij}\left( x\right) \xi _{i}\xi _{j}\geq \lambda
_{0}\left\vert \xi \right\vert ^{2}\text{ para todo }x\in \Omega \text{ y }%
\xi \in \mathbb{R}^{n}.  \tag{2}
\end{equation}

\item[b)]  Los coeficientes $a^{ij},b^{i},c^{i}$ y $d$ $\left(
i,j=1,...,n\right) $ son funciones medibles sobre $\Omega $. Adem\'{a}s son 
\textbf{acotadas}, esto es existen constantes positivas $\Lambda $ y $\rho $
tales que para todo $x\in \Omega $ y $\lambda _{0}$ en $\left( 2\right) $ se
cumple 
\begin{equation}
\sum_{i,j=1}^{n}\left\vert a^{ij}\left( x\right) \right\vert ^{2}\leq
\Lambda ,\text{ \ }\lambda _{0}^{-2}\sum_{i=1}^{n}\left( \left\vert
b^{i}\left( x\right) \right\vert ^{2}+\left\vert c^{i}\left( x\right)
\right\vert ^{2}\right) +\lambda ^{-1}\left\vert d\left( x\right)
\right\vert \leq \rho ^{2}  \tag{3}
\end{equation}

\item[c)]  
\begin{equation}
\int_{\Omega }\left( dv-\sum_{i=1}^{n}b^{i}D_{i}v\right) dx\leq 0\text{ \ }%
\forall v\geq 0,v\in C_{0}^{1}\left( \Omega \right) .  \tag{4}
\end{equation}
\end{description}

\begin{definition}
Si $u\in W^{1,2}(\Omega )$ y $f\in L^{2}\left( \Omega \right) $, $u$ es una
soluci\'{o}n d\'{e}bil (subsoluci\'{o}n d\'{e}bil, supersoluci\'{o}n
d\'{e}bil), de 
\begin{equation*}
Lu=f\text{ \ \ en }\Omega ,
\end{equation*}
si para cada $v\in C_{0}^{1}\left( \Omega \right) $ con $v\geq 0$ se cumple 
\begin{equation}
\mathcal{L}\left( u,v\right) =F\left( v\right) \text{ \ }\left( \leq F\left(
v\right) ,\geq F\left( v\right) \right) ,  \tag{5}
\end{equation}
donde 
\begin{equation}
\mathcal{L}\left( u,v\right) :=\int_{\Omega }\left\{ \sum_{i,j=1}^{n}\left(
a^{ij}D_{j}u+b^{i}u\right) D_{i}v-\left( \sum_{i=1}^{n}c^{i}D_{i}u+du\right)
v\right\} dx.  \tag{6}
\end{equation}
$\mathcal{L}$ se denomina la \textbf{forma bilineal asociada} a $L$, y 
\begin{equation}
F\left( v\right) :=-\int_{\Omega }fvdx\text{ \ }\forall v\in C_{0}^{1}\left(
\Omega \right) .  \tag{7}
\end{equation}
\end{definition}

\begin{theorem}
(Principio del m\'{a}ximo generalizado) Sea $L$ el operador dado en $(1),$
cuyos coeficientes satisfacen $\left( 2\right) ,\left( 3\right) $ y $\left(
4\right) $. Si $u\in W^{1,2}\left( \Omega \right) $ es una soluci\'{o}n
d\'{e}bil de $Lu\geq 0\left( \leq 0\right) $ en $\Omega $, entonces 
\begin{equation*}
\sup_{\Omega }u\leq \sup_{\partial \Omega }u^{+}\text{ \ }\left(
\inf_{\Omega }u\geq \inf_{\partial \Omega }u^{+}\right) .
\end{equation*}
\end{theorem}

\begin{proof}
Se define $l:=\sup\limits_{\partial \Omega }u^{+}$, y se toma $k\in \mathbb{R%
}$ tal que $l\leq k<\sup\limits_{\Omega }u$. Entonces $v:=\left( u-k\right)
^{+}\in W_{0}^{1,2}\left( \Omega \right) $ y $m(\mathbf{supp\ }\nabla v)\neq
0.$ Por otro lado existe un $C>0,$ independiente de $k,$ tal que $m(\mathbf{%
supp\ }\nabla v)>C^{-n},$ $n\geq 2,$ esto es una contradicci\'{o}n. En
consecuencia 
\begin{equation*}
\sup_{\Omega }u\leq l\text{.}
\end{equation*}
\end{proof}

\begin{corollary}
Si $u\in W_{0}^{1,2}\left( \Omega \right) $ y $Lu=0$ en $\Omega $ en el
sentido d\'{e}bil, entonces $u=0$ en $\Omega $ en el sentido d\'{e}bil.
\end{corollary}

\section{ Problema de Dirichlet generalizado}

Con el principio del m\'{a}ximo generalizado (colorario) se sigue que existe
soluci\'{o}n d\'{e}bil \'{u}nica para el problema de Direchlet 
generalizado. Adicionalmente en la prueba se utilizan los siguientes
teoremas: la alternativa Fredholm, Lax-Milgram, y Rellich-Kondrackov el cual
establece que 
\begin{equation*}
W_{0}^{1,2}\left( \Omega \right) \subset \subset L^{2}\left( \Omega \right) 
\text{ \ si }n>2,
\end{equation*}
y $\Omega $ es abierto acotado en $\mathbb{R}^{n}$ con $\partial \Omega $ de
clase $C^{1}$.

\begin{definition}
Sean $f\in L^{2}\left( \Omega \right) ,$ $g\in W^{1,2}\left( \Omega \right) $
y $L$ el operador diferencial definido en $\left( 1\right) $, $u\in
W^{1,2}\left( \Omega \right) $ es una \textbf{soluci\'{o}n d\'{e}bil} de 
\begin{equation*}
\left\{ 
\begin{array}{c}
Lu=f\text{ \ en }\Omega , \\ 
u=g\text{ sobre }\partial \Omega ,
\end{array}
\right. 
\end{equation*}
si $u-g\in W_{0}^{1,2}\left( \Omega \right) $ y adem\'{a}s 
\begin{equation}
\mathcal{L}\left( u,v\right) =F\left( v\right) =-\int_{\Omega }fvdx\text{, }%
\forall v\in C_{0}^{1}\left( \Omega \right) ,  \tag{8}
\end{equation}
donde $\mathcal{L}$ viene dado por $\left( 6\right) .$
\end{definition}

\begin{lemma}
Si $I:W_{0}^{1,2}\left( \Omega \right) \longrightarrow \left(
W_{0}^{1,2}\left( \Omega \right) \right) ^{\ast }$, $u\longrightarrow Iu$,
donde 
\begin{equation*}
Iu\left( v\right) :=\int_{\Omega }uvdx\text{, }v\in W_{0}^{1,2}\left( \Omega
\right) .
\end{equation*}
Entonces $I$ es una inmersi\'{o}n compacta.
\end{lemma}

\begin{theorem}
Sea $L$ el operador diferencial dado en $\left( 1\right) $ con coeficientes
satisfaciendo $\left( 2\right) ,\left( 3\right) $ y $\left( 4\right) $, y $%
g\in W^{1,2}\left( \Omega \right) $. Entonces para cualquier $f\in
L^{2}\left( \Omega \right) $ el \textbf{problema de Dirichlet generalizado} 
\begin{equation*}
\left( P.D.\right) \left\{ 
\begin{array}{c}
Lu=f\text{ \ en }\Omega , \\ 
u=g\text{ sobre }\partial \Omega ,
\end{array}
\right.
\end{equation*}
tiene soluci\'{o}n d\'{e}bil \'{u}nica.
\end{theorem}

\begin{proof}
Si $u$ es una soluci\'{o}n d\'{e}bil de $\left( P.D.\right) $, entonces $%
w:=u-g\in W_{0}^{1,2}\left( \Omega \right) $ y adem\'{a}s para cada $v\in
W_{0}^{1,2}\left( \Omega \right) :$%
\begin{equation*}
\mathcal{L}\left( w,v\right) =\widetilde{F}\left( v\right) ,\text{ \ }%
\widetilde{F}\left( v\right) \in \left( W_{0}^{1,2}\left( \Omega \right)
\right) ^{\ast }.
\end{equation*}

$\mathcal{L}\left( .,.\right) $ es una forma bilineal acotada sobre $%
W_{0}^{1,2}\left( \Omega \right) \times W_{0}^{1,2}\left( \Omega \right) $ y
existe $\sigma _{0}$ tal que 
\begin{equation*}
\mathcal{L}_{\sigma _{0}}\left( u,v\right) :=\mathcal{L}\left( u,v\right)
+\sigma _{0}\left( u,v\right)
\end{equation*}
es coerciva. Luego Lax-Milgram implica que para $F\in \left(
W_{0}^{1,2}\left( \Omega \right) \right) ^{\ast }$ existe un \'{u}nico $u\in
W_{0}^{1,2}\left( \Omega \right) $ tal que 
\begin{equation*}
\mathcal{L}_{\sigma _{0}}\left( u,v\right) =F\left( v\right) \text{ \ para
todo }v\in W_{0}^{1,2}\left( \Omega \right) .
\end{equation*}
La aplicaci\'{o}n $L_{\sigma _{0}}:W_{0}^{1,2}\left( \Omega \right)
\longrightarrow \left( W_{0}^{1,2}\left( \Omega \right) \right) ^{\ast }$,
dada por 
\begin{equation*}
(L_{\sigma _{0}}u)\left( v\right) :=\mathcal{L}_{\sigma _{0}}\left(
u,v\right) \text{ \ }\forall v\in W_{0}^{1,2}\left( \Omega \right)
\end{equation*}
es lineal, acotada uno-uno, sobre y $L_{\sigma _{0}}^{-1}$ tambi\'{e}n es
acotada.

As\'{\i} las ecuaciones $Lu=F$ y $u-\sigma _{0}L_{\sigma
_{0}}^{-1}Iu=L_{\sigma _{0}}^{-1}F$ son equivalentes. El operador $T:=\sigma
_{0}L_{\sigma _{0}}^{-1}I$ es compacto, debido al lema $5$, y por el
corolario $3$ se deduce que $I-T$ es inyectiva. Luego por la alternativa de
Fredholm se da la conclusi\'{o}n del teorema.
\end{proof}

\section{Principio del m\'{a}ximo cl\'{a}sico}

Ahora las funci\'{o}nes $u\in C^{2}\left( \Omega \right) \cap C\left( 
\overline{\Omega }\right) $ y el operador $L$ es 
\begin{equation}
Lu:=\sum_{i,j=1}^{n}a_{ij}\left( x\right)
D_{ij}^{2}u+\sum_{i=1}^{n}b_{i}\left( x\right) D_{i}u+c\left( x\right) u, 
\tag{9}
\end{equation}
donde los coeficientes $a_{ij},b_{i}$ y $c$ est\`{a}n definidos en un
abierto, no vacio, $\Omega $ en $\mathbb{R}^{n}$, $n\geq 2$, y son acotados.
Adem\'{a}s $A=\left[ a_{ij}\left( x\right) \right] $ simetr\'{\i}a $\forall
x\in \Omega $ y $L$ es estrictamente el\'{\i}ptico, $\left( 2\right) $.

\begin{theorem}
(Principio del m\'{a}ximo d\'{e}bil para $c\leq 0$) si $Lu\geq 0$ $\left(
Lu\leq 0\right) $ en $\Omega $ y $c\leq 0$, entonces 
\begin{equation*}
\max_{\overline{\Omega }}u\leq \max_{\partial \Omega }u^{+}\text{ \ }\left(
\min_{\overline{\Omega }}u\geq \min_{\partial \Omega }u^{-}\right) .
\end{equation*}
\end{theorem}

\begin{lemma}
(de Frontera de Hopf) Sup\'{o}ngase que $\Omega $ satisface la condici\'{o}n
de bola interior en $x_{0}\in \partial \Omega $ y $c\equiv 0$ en $L$ dado en 
$\left( 9\right) $. Si $u\in C^{2}\left( \Omega \right) \cap C\left( \Omega
\cup \left\{ x_{0}\right\} \right) $ satisface $Lu\geq 0$ en $\Omega $ y $%
u\left( x_{0}\right) >u\left( x\right) $ $\forall x\in \Omega $, entonces 
\begin{equation*}
\lim \inf_{t\rightarrow 0^{-}}\frac{u\left( x_{0}+t\nu \right) -u\left(
x_{0}\right) }{t}>0\text{,}
\end{equation*}
donde $\nu $ es la normal exterior a la bola interior en $x_{0}.$

Si en las hip\'{o}tesis iniciales se cambia $c=0$ en $\Omega $ por $c\leq 0$
en $\Omega $ y $u\left( x_{0}\right) \geq 0$, se obtiene la misma
conclusi\'{o}n.

Si en las hip\'{o}tesis iniciales se cambia $c=0$ en $\Omega $ por $u\left(
x_{0}\right) =0$ en $\Omega $, se obtiene la misma conclusi\'{o}n
(independiente del signo de $c$).
\end{lemma}

\begin{theorem}
$\Omega $ es un dominio, no necesariamente acotado, $u\in C^{2}\left( \Omega
\right) \cap C\left( \overline{\Omega }\right) $, no constante, satisface $%
Lu\geq 0$ $\left( Lu\leq 0\right) $ en $\Omega $. Si $c=0$, $u$ no alcanza
un m\'{a}ximo (m\'{\i}nimo) en el interior de $\Omega $. Si $c\leq 0$, $u$
no alcanza un m\'{a}ximo no negativo (m\'{\i}nimo positivo) en el interior
de $\Omega $. Independiente del signo de $c$, $u$ no puede ser cero en un
m\'{a}ximo (m\'{\i}nimo) interior.
\end{theorem}

\section{Aplicaci\'{o}n a la teor\'{\i}a de valores propios}

\begin{theorem}
Dado $\Omega $ abierto acotado, $a_{ij}\in C^{\infty }\left( \overline{%
\Omega }\right) $, $a_{ij}=a_{ji}$, $-L$ estrictamente el\'{\i}ptico, donde 
\begin{equation*}
Lu=-\sum_{i,j=1}^{n}\frac{\partial }{\partial x_{i}}\left( a_{ij}\left(
x\right) \frac{\partial }{\partial x_{j}}u\right) ,
\end{equation*}
entonces

\begin{enumerate}
\item[$a)$]  El primer valor propio de $L$, $\lambda _{1},$ es positivo y
adem\'{a}s 
\begin{equation}
\lambda _{1}=\left\{ \mathcal{L}\left( u,u\right) :u\in H_{0}^{1}\left(
\Omega \right) ,\text{ }\left\Vert u\right\Vert _{L^{2}\left( \Omega \right)
}=1\right\}  \tag{10}
\end{equation}

\item[$b)$]  El m\'{\i}nimo en $\left( 10\right) $ es alcanzado por una
funci\'{o}n $w_{1}$ positiva en $\Omega $, que es soluci\'{o}in d\'{e}bil de 
\begin{equation}
\left\{ 
\begin{array}{c}
Lw_{1}=\lambda _{1}w_{1}\text{ \ en }\Omega , \\ 
w_{1}=0\text{ \ sobre }\partial \Omega .
\end{array}
\right.  \tag{11}
\end{equation}
\end{enumerate}
\end{theorem}

\section{Un principio del m\'{a}ximo donde $c$ puede ser positivo}

Sea $\Omega $ un dominio acotado en $\mathbb{R}^{n}$, $L$ el operador
diferencial definido en $\left( 9\right) $ y $\left( S_{\Omega }\right) $ la
condici\'{o}n: 
\begin{equation*}
\left( S_{\Omega }\right) \left\{ 
\begin{array}{c}
\text{Existen constantes }\lambda _{0}\text{ y }\Lambda _{0}\text{ tales que:%
} \\ 
\lambda _{0}\in \xi ^{2}\leq \xi ^{T}a\left( x\right) \xi \leq \Lambda
_{0}\xi ^{2}\text{ para todo }\xi \in \mathbb{R}^{n}, \\ 
\left\vert b_{i}\left( x\right) \right\vert \leq \Lambda _{0}\text{ y }%
-\Lambda _{0}\leq c\left( x\right) \text{ en }\Omega \text{.}
\end{array}
\right.
\end{equation*}
Adicionalmente se considera la siguiente definici\'{o}n.

\begin{definition}
Usando la notaci\'{o}n $d\left( x\right) =dist\left( x,\partial \Omega
\right) $, $x\in \Omega $, se dice que el conjunto $\Omega $ satisface la 
\textbf{condici\'{o}n de bola interior uniforme en el sentido fuerte}, si
para alg\'{u}n $r>0$, y todo $x\in \Omega $ con $d\left( x\right) \leq r$ le
corresponde un punto m\'{a}s cercano $y\in \partial \Omega $, $d\left(
x\right) =\left\vert x-y\right\vert $,con la propiedad que $B_{r}\left(
z\right) \subset \Omega $, donde $z=y+\frac{r\left( x-y\right) }{\left\vert
x-y\right\vert }$.
\end{definition}

\begin{theorem}
Sup\'{o}ngase que $\Omega $ satisface la condici\'{o}n de bola interior
uniforme en el sentido fuerte, $L$ dado por $\left( 9\right) $, $\left(
S_{\Omega }\right) $ se cumple, $u\in C^{2}\left( \Omega \right) \cap
C\left( \overline{\Omega }\right) $ satisfaciendo 
\begin{equation*}
Lu\leq 0\text{ \ en }\Omega \text{ \ y }u\geq 0\text{ \ sobre }\partial
\Omega
\end{equation*}
es Lipschitz continua en $\overline{\Omega }$. Si $h\in C^{2}\left( \Omega
\right) \cap C\left( \overline{\Omega }\right) $ satisface 
\begin{equation*}
Lh\leq 0\text{ \ y \ }h>0\text{ en }\Omega \text{,}
\end{equation*}
entonces 
\begin{equation*}
u=\beta h\text{ \ en }\Omega \text{ para un }\beta <0\text{ \ \'{o} \ }%
u\equiv 0\text{ \ en }\Omega \text{ \ \'{o} }u>0\text{ \ en }\Omega \text{.}
\end{equation*}
\end{theorem}

\section{Extensi\'{o}n del principio del m\'{a}ximo}

Se considera el operador el\'{\i}ptico sim\'{e}trico 
\begin{equation}
Lu=\sum_{i,j=1}^{n}\frac{\partial }{\partial x_{i}}\left( a_{ij}\left(
x\right) \frac{\partial }{\partial x_{j}}u\right) ,  \tag{12}
\end{equation}
donde $a_{ij}\left( x\right) \in C^{\infty }\left( \overline{\Omega }\right) 
$ y $\partial \Omega $ es suave.

Del teorema $10$ se tiene que $\lambda _{1}$, valor propio principal de $-L$%
, es positivo y existe $w_{1}>0$ tal que 
\begin{equation*}
\left\{ 
\begin{array}{c}
Lw_{1}+\lambda _{1}w_{1}=0\text{ \ en }\Omega , \\ 
w_{1}=0\text{ \ sobre }\partial \Omega .
\end{array}
\right.
\end{equation*}
Adem\'{a}s se sabe que 
\begin{equation*}
\left( PMF\right) \left\{ 
\begin{array}{c}
Lu+cu\leq 0\text{ \ en }\Omega \text{ y }u\geq 0\text{ sobre }\partial
\Omega , \\ 
\text{implica} \\ 
u\equiv 0\text{ \ en }\Omega \text{ \ \'{o} \ }u>0\text{ \ en }\Omega \text{,%
}
\end{array}
\right.
\end{equation*}
es verdadera si $c\leq 0$ y falsa si $c\equiv \lambda _{1}.$

\begin{theorem}
Dado $L$ en $\left( 12\right) $, los coeficientes de $L$, $\Omega $ y $u$
satisfaciendo las mismas hip\'{o}tesis del teorema $12.$ Entonces $\left(
PMF\right) $ es verdadero si $c(x)\lvertneqq \lambda _{1}$ en $\Omega .$
\end{theorem}

\begin{proof}
Es consecuencia del teorema $12$ tomando $h:=w_{1}.$
\end{proof}

\section{Aplicaci\'{o}n: Estimaciones para la soluci\'{o}n de una
ecuaci\'{o}n diferencial}

\begin{theorem}
Sea $\left\vert \kappa \right\vert \lneqq \lambda _{1}$ ($\lambda _{1}$
valor propio principal de $-\Delta $), y sup\'{o}ngase que \linebreak $u\in
C^{2}\left( B\right) \cap C(\overline{B})$ Lipschitz continua en $\overline{B%
}$ satisface 
\begin{equation*}
\left( P.V.F.\right) \left\{ 
\begin{array}{c}
\Delta u+\kappa u=C_{0}\text{ \ \ en }B, \\ 
u=0\text{ \ \ sobre }\partial B,
\end{array}
\right.
\end{equation*}
donde $B:=B_{R}\left( 0\right) $ y $C_{0}\in \mathbb{R}.$

\begin{enumerate}
\item  Si $C_{0}=0,$ entonces el $\left( P.V.F.\right) $ tiene como
\'{u}nica soluci\'{o}n $u\equiv 0$ en $B.$

\item  Si $-\lambda _{1}\leq \kappa \leq 0$, entonces existe una constante
positiva $M$ tal que 
\begin{equation*}
u\left( x\right) \leq M\left( M-x^{2}\right) \text{ \ \ en }B\text{.}
\end{equation*}

\item  Si $C_{0}\leq 0,$ $-\lambda _{1}\leq \kappa \leq 0$ y existe un $%
\gamma >0$ tal que $u(x)\geq $ $\gamma >0$ en $B_{\rho }\left( 0\right) ,$
para un $0<\rho <R$. Entonces existen constantes positivas $M$ y $\delta $
tales que 
\begin{equation*}
\delta \left( R-\left\vert x\right\vert \right) \leq u\left( x\right) \leq
M\left( M-x^{2}\right) \text{ \ \ en }B\text{.}
\end{equation*}
\end{enumerate}
\end{theorem}

\section{Simetr\'{\i}a esf\'{e}rica y monoton\'{\i}a radial de soluciones
positivas para la ecuacion de Poisson no-lineal en $\mathbb{R}^{n}$}

\begin{lemma}
Sean $R>0,$ $u\in C^{2}\left( \left| x\right| >R\right) \cap C\left( \left|
x\right| \geq R\right) ,$ $u>0$ en $\left| x\right| \geq R$, tendiendo a
cero en el infinito y satisfaciendo 
\begin{equation*}
Lu\equiv \left( \Delta +\sum_{j=1}^{n}b_{j}\left( x\right) \partial
_{j}+c\left( x\right) \right) u\leq 0\text{ \ \ en }\left| x\right| >R,
\end{equation*}
donde $b_{j}=O\left( \left| x\right| ^{1-p}\right) $, $c\left( x\right)
=O\left( \left| x\right| ^{-p}\right) $ en el infinito, $p>2,$ y adem\'{a}s
continuas en $\left| x\right| \geq R$. Entonces, existe un $\mu >0$, tal que 
\begin{equation*}
u\left( x\right) \geq \frac{\mu }{\left| x\right| ^{n-2}}\text{.}
\end{equation*}
\end{lemma}

\begin{proof}
De la funci\'{o}n comparaci\'{o}n $z=\dfrac{1}{r^{n-2}}+\dfrac{1}{r^{s}},$
donde $r=\left\vert x\right\vert $ y \linebreak $n-2<s<n-4+p,$ y el
principio del m\'{a}ximo d\'{e}bil se obtiene la conclusi\'{o}n.
\end{proof}

\begin{lemma}
Sean $u:\mathbb{R}^{n}\rightarrow \mathbb{R}$ es positiva y $C^{2}$ en $%
\mathbb{R}^{n},$ $u=O(\left\vert x\right\vert ^{-m}),$ en el infinito, $m>0,$
$b:\mathbb{R}^{n}\rightarrow \mathbb{R}$ acotado y $g:\mathbb{R}\rightarrow 
\mathbb{R}$ es tal que en el intervalo $0\leq u\leq u_{0}=\max u$, $%
g=g_{1}+g_{2}$ con $g_{1}\in C^{1}$, $g_{2}$ continua y mon\'{o}tona no
decreciente. $u$ es soluci\'{o}n de 
\begin{equation}
\Delta u+b\left( x\right) u_{1}+g\left( u\right) =0\text{ \ \ en\ }\mathbb{R}%
^{n}\text{.}  \tag{13}
\end{equation}
Sup\'{o}ngase que existe un $\lambda \in \left[ 0,\infty \right) $ tal que

\textbf{1}. $b\left( x\right) \geq 0$ para todo $x\in \mathbb{R}^{n}$

\textbf{2}. $u_{1}\left( x\right) \leq 0$ y $u\left( x\right) \leq u\left(
x^{\lambda }\right) $ para todo $x\in \Sigma \left( \lambda \right) $,

\textbf{3}. $u\left( z\right) \neq u\left( z^{\lambda }\right) $ para
alg\'{u}n $z\in \Sigma \left( \lambda \right) $.

Entonces 
\begin{eqnarray*}
u\left( x\right) &<&u\left( x^{\lambda }\right) \text{ \ \ para todo \ }x\in
\Sigma \left( \lambda \right) \text{,} \\
u_{1}\left( x\right) &<&0\text{ \ \ para todo \ }x\in T_{\lambda }\text{.}
\end{eqnarray*}
\end{lemma}

\begin{proof}
Se define $v:\Sigma ^{\prime }\left( \lambda \right) \longrightarrow \mathbb{%
R}$ por $v:=u\left( x^{\lambda }\right) $ y se obtiene que 
\begin{equation}
\Delta v\left( x\right) -b\left( x^{\lambda }\right) v_{1}\left( x\right)
+g\left( v\left( x\right) \right) =0\text{ \ \ }\forall x\in \Sigma ^{\prime
}\left( \lambda \right) \text{.}  \tag{14}
\end{equation}
$w:=v-u$ en $\Sigma ^{\prime }\left( \lambda \right) $ satisface $w\leq 0$ y 
$w\neq 0$. Luego de $\left( 13\right) $, $\left( 14\right) $ y el principio
del m\'{a}ximo fuerte se sigue que 
\begin{equation*}
u\left( x\right) <u\left( x^{\lambda }\right) \text{ \ \ para todo \ }x\in
\Sigma \left( \lambda \right) .
\end{equation*}
Como $w=0$ sobre $T_{\lambda }$, del lema de frontera de Hopf se concluye 
\begin{equation*}
u_{1}\left( x\right) <0\text{ \ \ para todo \ }x\in T_{\lambda }\text{.}
\end{equation*}
\end{proof}

\begin{lemma}
Si $f\in C(\mathbb{R}^{n})$ satisface $f\left( y\right) =O\left( \left\vert
y\right\vert ^{-q}\right) $ cerca al infinito, \linebreak $q>n+1$, $u\in
C^{2}$ y existe un $C>0$ tal que 
\begin{equation*}
u\left( x\right) =C\int_{\mathbb{R}^{n}}\dfrac{1}{\left\vert x-y\right\vert
^{n-2}}f\left( y\right) dy\text{, \ \ }x\in \mathbb{R}^{n}\text{,}
\end{equation*}
entonces

\textbf{a)} 
\begin{equation*}
\left| x\right| ^{n-2}u(x)\longrightarrow C\int f\left( y\right) dy\text{, \
\ cuando }\left| x\right| \rightarrow \infty .
\end{equation*}

\textbf{b)} 
\begin{equation*}
\frac{\left| x\right| ^{n}}{x_{1}}u_{1}\left( x\right) \longrightarrow
-\left( n-2\right) C\int f\left( y\right) dy,\text{ \ cuando }\left|
x\right| \rightarrow \infty ,\text{\ con }x_{1}\rightarrow \infty \text{.}
\end{equation*}

\textbf{c) }Si $\left\{ \lambda ^{i}\right\} \subset \mathbb{R}$ con $%
\lambda ^{i}\longrightarrow \lambda $ y $\left\{ x^{i}\right\} $ es una
sucesi\'{o}n en $\mathbb{R}^{n}$ tendiendo al infinito, con $%
x_{1}^{i}<\lambda ^{i}$ $\ $para todo $i\in \mathbb{N}$, entonces 
\begin{equation*}
\frac{\left\vert x^{i}\right\vert ^{n}}{\lambda ^{i}-x_{1}^{i}}\left(
u\left( x^{i}\right) -u\left( x^{i^{\lambda ^{i}}}\right) \right)
\longrightarrow 2\left( n-2\right) C\int f\left( y\right) \left( \lambda
-y_{1}\right) dy.
\end{equation*}
\end{lemma}

\begin{theorem}
Sea $u$ soluci\'{o}n positiva y de clase $C^{2}$ de 
\begin{equation}
-\Delta u=g\left( u\right) \text{ \ \ en }\mathbb{R}^{n}\text{, }n\geq 3, 
\tag{15}
\end{equation}
con $u\left( x\right) =O\left( \left\vert x\right\vert ^{-m}\right) $ en el
infinito, $m>0$.

Sup\'{o}ngase: \textbf{(i)} En el intervalo $0\leq u\leq u_{0}$ donde $%
u_{0}=\max u$, $g=g_{1}+g_{2}$ con $g_{1}\in C^{1}$, $g_{2}$ continua y
mon\'{o}tona no decreciente. \textbf{(ii)} Para alg\'{u}n \linebreak $\alpha
>\max \left\{ \frac{n+1}{m},\frac{2}{m}+1\right\} $, $g\left( u\right)
=O\left( u^{\alpha }\right) $ cerca de $u=0$. Entonces $u\left( x\right) $
es esf\'{e}ricamente sim\'{e}trica alrededor de alg\'{u}n punto de $\mathbb{R%
}^{n}$ y $u_{r}<0$ para $r>0,$ donde $r$ es la coordenada radial alrededor
de ese punto. Adem\'{a}s 
\begin{equation}
\lim_{\left\vert x\right\vert \rightarrow \infty }\left\vert x\right\vert
^{n-2}u\left( x\right) =k>0\text{.}  \tag{16}
\end{equation}
\end{theorem}

\begin{proof}
$\left( 15\right) $ se puede escribir de la forma 
\begin{equation*}
\Delta u+c\left( x\right) u=0,
\end{equation*}
donde $p:=m\left( \alpha -1\right) >2$, y $c\left( x\right) =\frac{g\left(
u\left( x\right) \right) }{u\left( x\right) }=O\left( \left\vert
x\right\vert ^{-p}\right) .$ Luego de los lemas $15$ y $17a)$ se deduce que 
\begin{equation}
\int f\left( y\right) dy>0.  \tag{17}
\end{equation}
Con eso y el lema $17b)$ resulta: 
\begin{equation}
\text{existe un }\widetilde{\lambda }>0\text{ \ \ tal que \ \ }u_{1}\left(
x\right) <0\text{ para }x_{1}\geq \widetilde{\lambda }\text{.}  \tag{18}
\end{equation}
Ahora, se hace una traslaci\'{o}n hasta un punto $\mathbf{q=}\left(
q_{1},...,q_{n}\right) $ y se fijan las coordenadas radiales con respecto a
\'{e}ste punto (nuevo origen) de modo que 
\begin{equation}
\int f\left( y\right) y_{j}dy=0,\text{ \ \ }j=1,...,n.  \tag{19}
\end{equation}
Luego existe un $\lambda _{0}>0$ tal que para todo $\lambda \geq \lambda
_{0},$ 
\begin{equation}
u\left( x\right) >u\left( x^{\lambda }\right) \text{ \ \ si }x_{1}<\lambda 
\text{.}  \tag{20}
\end{equation}
De $\left( 18\right) ,$ $\left( 20\right) $, el lema $16$ y el lema $17c)$
resulta un intervalo m\'{a}ximal $\left( \lambda _{1},\infty \right) ,$ $%
\lambda _{1}>0,$ donde $\left( 18\right) $ y $\left( 20\right) $ valen.
Luego 
\begin{equation*}
u\left( x\right) \geq u\left( x^{\lambda _{1}}\right) \text{ para\ }%
x_{1}<\lambda _{1}\text{ \ \ y \ \ }u_{1}<0\text{ para }x_{1}>\lambda _{1}%
\text{.}
\end{equation*}
La simetr\'{\i}a de \ $u$ con respecto al hiperplano $x_{1}=0$ se sigue el
lema $16$ y con eso el teorema queda probado.
\end{proof}

\end{document}
