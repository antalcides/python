% language support
\documentclass[30pt,landscape]{foils}%
\usepackage[english,spanish]{babel}
\usepackage[latin1]{inputenc}
\usepackage[pdftex]{color}
\usepackage[pdftex]{geometry}
\usepackage{hyperref}
\usepackage{pause}
\usepackage{background}
\usepackage{pp4slide}%
\usepackage{amsmath}%
\setcounter{MaxMatrixCols}{30}%
\usepackage{amsfonts}%
\usepackage{amssymb}%
\usepackage{graphicx}
%\usepackage{m-metapo}
%TCIDATA{OutputFilter=latex2.dll}
%TCIDATA{Version=4.00.0.2312}
%TCIDATA{LastRevised=Friday, August 20, 1999 19:17:48}
%TCIDATA{<META NAME="GraphicsSave" CONTENT="32">}
\geometry{headsep=3ex,hscale=0.9}
\hypersetup{pdftitle={PPower4Example},
pdfsubject={An example to demonstrate PPower4},
pdfauthor={Klaus Guntermann, Systems Programming Group,
Darmstadt University of Technology,
<guntermann@iti.informatik.tu-darmstadt.de>},
pdfkeywords={pdftex, acrobat},
pdfpagemode={FullScreen}
}
\begin{document}
\definecolor{bgred}{rgb}{0.00,0.00,0.53} \hpagecolor{bgred}

\foilhead{Medici�n y C�lculo del error}

\begin{itemize}
\item Itemizations\pause

\begin{itemize}


\item are nested\pause

\item and labelled  \hypertarget{start}{}
%mark any place on the page


\begin{itemize}


\item even nested deeply\pause
\end{itemize}

\item on different levels\pause
\end{itemize}

\item with formulas like $\sum_{i=0}^{\infty}a_{i}\cdot x^{i}$
\end{itemize}

\foilhead{Different page transitions}
\hypersetup{pdfpagetransition=Dissolve}

\begin{itemize}
\item Switch it on for a page\pause

\item But must switch off explicitely
\end{itemize}

\foilhead[2ex]{That's it} \hypersetup{pdfpagetransition=R}

\begin{center}
Just a small example.

You may go \hyperlink{start}{back} to start.
\end{center}
\foilhead[2ex]{That's
it}
 \hypersetup{pdfpagetransition=R}
\pdfximage {pic.pdf}    % read image pic.pdf (from path TEXINPUTS)

% when a pdf image may have more than 1 page, the number of the last image
% opened by \pdfximage is accessible via \pdflastximagepages

The last image has \the\pdflastximagepages\ page(s).

% It also possible to select which page to include. Pages are numbered from
% 1. Out-of-range values are fixed to 1 and bitmapped images have always 1
% page. To include let's say the 2nd page of a pdf image, one can use
% \pdfximage page 2 {file.pdf}

% similarly to \pdfobj, the image created by \pdfximage is held in memory
% and is not written to the pdf output, unless 1) the image is referenced
% by saying \pdfrefximage <object number>; or 2) \pdfximage is preceded by
% \immediate. The object number of the last image created by \pdfximage is
% accessible via \pdflastximage. Note that \immediate\pdfximage does not
% typeset anything; it just ensures that the image is written out to the
% output. To display the image, it is necessary to say
% \pdfrefximage <object number> in the intended place. Another point is
% \pdfrefximage act similarly rather to rules than to boxes concerning
% dimensions and space setting. In order to make images act as boxes
% (concerning dimensions and spacing), it's necessary to wrap them into a
% box.

\pdfrefximage \pdflastximage

% the dimensions of the image can be also controlled via <rule spec>. The
% default values are zero for depth and `running' for height and width. If
% all of them are given, the image will be scaled to fit the specified
% values. If some of them (but not all) are given, the rest will be set to
% a value corresponding to the remaining ones so as to make the image size
% to yield the same proportion of width : (height + depth) as the original
% image size, where depth is treated as zero. If none of them is given then
% the image will take its natural size. An image inserted at its natural
% size often has a resolution of \pdfimageresolution (or 72 if
% \pdfimageresolution is set to zero) given in dots per inch in the output
% file, but some images may contain data specifying the image resolution,
% and in such a case the image will be scaled to the original resolution.
% The dimension of the image can be accessed by enclosing the \pdfrefximage
% command to a box and checking the dimensions of the box.
\foilhead[2ex]{That's it} \hypersetup{pdfpagetransition=Dissolve}
\pdfximage width 6cm {pic.png}  % set the image width and keep the
\pdfrefximage \pdflastximage    % `nature' proportion width : height

\pdfximage height 4cm {pic.tif} % set the image height and keep the
\pdfrefximage \pdflastximage    % `nature' proportion width : height

\pdfximage width 6cm height 4cm % set both image width and height; the
   {pic.jpg}                    % final proportion width : height may be
                                % different from the `nature' one
\foilhead[2ex]{That's it} \hypersetup{pdfpagetransition=Dissolve}
\pdfimageresolution=72          % to open an image at 72 dpi
\pdfximage {pic.pdf}\pause
\setbox0=\hbox{\pdfrefximage\pdflastximage} % get dimensions of the image
                                            % in order to include the image
                                            % at a specific resolution
\dimen0=.06\wd0         % calculate the image width at 1200 dpi (0.06 = 72/1200)
\pdfximage              % include the image at resolution 1200 dpi
    width \dimen0 {pic.pdf}  % by setting image width to the calculated value
\pdfrefximage \pdflastximage

\foilhead[2ex]{That's
it}
 \hypersetup{pdfpagetransition=Dissolve}
 \pdfximage height \vsize {tmp.pdf}      % insert the PDF converted from EPS
\pdfrefximage \pdflastximage
\begin{figure}
\includegraphics[height=5cm,width=5cm,keepaspectratio=true]{C:/texmf/tex/latex/ppower4/pic.jpg}
\caption{ejemplo de .jpg}
\end{figure}
\foilhead[2ex]{That's it}
 \hypersetup{pdfpagetransition=Dissolve}
\begin{figure}
\begin{center}
\includegraphics[height=10cm,width=5cm,keepaspectratio=true,angle=20]{C:/texmf/tex/latex/ppower4/pic.jpg}
\end{center}
\end{figure}




\end{document}
