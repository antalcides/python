



\begin{theorem}
Sea $u$ soluci\'{o}n positiva y de clase $C^{2}$ de
\begin{equation}
-\Delta u=g\left(  u\right)  \text{ \ \ en }\mathbb{R}^{n}\text{, }%
n\geq3,\tag{15}%
\end{equation}
con $u\left(  x\right)  =O\left(  \left\vert x\right\vert ^{-m}\right)  $ en
el infinito, $m>0$.

Sup\'{o}ngase: \textbf{(i)} En el intervalo $0\leq u\leq u_{0}$ donde
$u_{0}=\max u$, $g=g_{1}+g_{2}$ con $g_{1}\in C^{1}$, $g_{2}$ continua y
mon\'{o}tona no decreciente. \textbf{(ii)} Para alg\'{u}n \linebreak%
$\alpha>\max\left\{  \frac{n+1}{m},\frac{2}{m}+1\right\}  $, $g\left(
u\right)  =O\left(  u^{\alpha}\right)  $ cerca de $u=0$. Entonces $u\left(
x\right)  $ es esf\'{e}ricamente sim\'{e}trica alrededor de alg\'{u}n punto de
$\mathbb{R}^{n}$ y $u_{r}<0$ para $r>0,$ donde $r$ es la coordenada radial
alrededor de ese punto. Adem\'{a}s
\begin{equation}
\lim_{\left\vert x\right\vert \rightarrow\infty}\left\vert x\right\vert
^{n-2}u\left(  x\right)  =k>0\text{.}\tag{16}%
\end{equation}

\end{theorem}
