\documentclass[letter]{article}
\usepackage[pdftex]{graphicx}
\usepackage[ansinew]{inputenc}
\usepackage{amsmath}
\usepackage{amsbsy}
\usepackage{amssymb}
\usepackage[spanish]{babel}
\usepackage{float}
\usepackage{fixseminar}
%\usepackage[display]{texpower}
%\usepackage[ams]{pdfslide3}
\usepackage[ams]{pdfslide3}
\newtheorem{theorem}{Teorema}
\newtheorem{lemma}[theorem]{Lema}
\newtheorem{corollary}[theorem]{Corolario}
\newtheorem{definition}[theorem]{Definici\'{o}n}
\newenvironment{proof}[1][Prueba]{\noindent\textbf{#1.} }{\ \rule{0.5em}{0.5em}}
%\newcommand{\@url}{}

\newcommand{\thankyouslide}[1][Thank You!]{\pagestyle{empty}
\mbox{}\vfill{\hfill\LARGE\scshape #1\hfill\mbox{}}\vfill{\hfill \@gjimenez@uninorte.edu.co\hfill\mbox{}}\vfill}




%\url{http://www.yourwebpage.com/slidepresentation}

\pagestyle{title}

\begin{document}
%\convenio{Convenio}
\orgname{\color{black}Universidad Nacional de Colombia}
%\udea{Universidad del Norte}
\orgurl{\protect\color{black}http://www.unalmed.edu.co}


\title{\color{black}El Principio Variacional De Ekeland Y Aplicaciones}
\author{\scalebox{1}[1.3]{\emph{\textbf{\color{black}Autor: GERMAN ~JIMENEZ ~BLANCO} } }}
\director{\scalebox{1}[1.3]{\emph{\textbf{\color{black}Director:
Ph. D.  JORGE COSSIO BETANCUR }} }}
\address{\color{black}Trabajo presentado como requisito parcial\\para optar al t�tulo de Magister en Matem�ticas\\
%  {\realfootnotesize(para optar al t�tulo de )}\\
                           email: {\tt gjimenez@uninorte.edu.co}}
%\address{Universidad del Norte - Barranquilla, Colombia\\
%  {\realfootnotesize(Departamento de Matem�ticas y F�sica)}
%              email: {\tt gjimenez@uninorte.edu.co}}

\notes{\emph{Esta presentaci�n fue desarrollada usando pdfslide class bajo Mik\TeX{}}.
\newline
{\tiny{Copyright\copyright 2001 Dario Castro Castro. All rights
reserved. \today{}}}}


\overlay{fondo1.png} \maketitle
%%%%%%%%%%%%%%%%%%%%%ACETATO 1%%%%%%%%%%%%%%%%%%%%%%%%

\pagedissolve{Wipe /D 3 /Dm /V /M /O}%%%%%%%%%%%%%%%Efecto de transici�n de p�gina
\overlay{fondo1.png} %%%%%%%%%%%%%%%%%%%%%%%%%%%%%%%%%Fondo
\sffamily %%%%%%%%%%%%%%%%%%%%%%%%%%%%%%%%%%%%%%%%%%Tipo de letra
\Large %%%%%%%%%%%%%%%%%%%%%%%%%%%%%%%%%%%%%%%%%%%%%Tama�o de la letra
\color{black} %%%%%%%%%%%%%%%%%%%%%%%%%%%%%%%%%%%%%%Color de las letras
\headskip=20pt%%%%%%%%%%%%%%%%%%%%%%%%%%%%%%%%%%%%%%Longitud del borde superior al t�tulo
\section{\textcolor[rgb]{0.40,0.20,0.40}{RESUMEN}}
 En este trabajo se estudia el Principio Variacional de
Ekeland y algunas aplicaciones a la teor\'{\i}a de puntos
cr\'{\i}ticos y a ecuaciones diferenciales parciales. Se prueba a
partir del Principio Variacional de Ekeland el Teorema del Paso de
la Monta\~{n}a y el Teorema de Punto de Silla; tambi\'{e}n se
demuestra la existencia de soluciones d\'{e}biles para
problemas de Dirichlet semilineales. %

%%%%%%%%%%%%%%%%%%%%%ACETATO 2%%%%%%%%%%%%%%%%%%%%%%%%
\headskip=10pt \pagedissolve{Wipe /D 3 /Di /V /M /O}
\overlay{fondo1.png} \large \color{black}
\section{{Principio Variacional de Ekeland -forma fuerte}}
\begin{theorem}
\label{t1}Sean $\left( X,d\right) $ un espacio m\'{e}trico
completo y \newline$\phi :X\rightarrow \mathbb{R}\cup \left\{
+\infty \right\} $ una funci\'{o}n semicontinua inferiormente, no
id\'{e}nticamente igual\ a$\ +\infty ,\;$ y acotada
inferiormente.\newline
Sean $\varepsilon >0$ y $\overline{u}\in X$ tales que $\phi \left( \overline{%
u}\right) \leq \inf\limits_{X}\phi +\frac{\varepsilon
}{2}$.\newline Entonces para cada $\lambda >0$ \ existe
$u_{\lambda }\in X$ tal que se
cumplen las siguientes condiciones:%
\begin{equation}
\phi (u_{\lambda })\leq \phi (\overline{u})  \label{e1}
\end{equation}%
\begin{equation}
d\left( u_{\lambda },\overline{u}\right) \leq \lambda  \label{e2}
\end{equation}%
\begin{equation}
\phi (u_{\lambda })<\phi (u)+\frac{\varepsilon }{\lambda }d\left(
u_{\lambda },u\right) \quad \forall u\neq u_{\lambda }  \label{e3}
\end{equation}
\end{theorem}

%%%%%%%%%%%%%%%%%%%%%ACETATO 34%%%%%%%%%%%%%%%%%%%%%%%%
\headskip=20pt
\section{{Prueba Forma Fuerte}}
\overlay{fondo1.png}\Large \color{black} \pagedissolve{Wipe /D 1
/Di /H /M /O}
\begin{proof}
Para $\lambda >0,\;$sea $d_{\lambda }(x,y):=\dfrac{d\left(
x,y\right) }{ \lambda }$\newline
Definamos la siguiente relaci\'{o}n en $X\;$%
\[
\ u\leq v\Longleftrightarrow \phi \left( u\right) \leq \phi \left(
v\right) -\varepsilon d_{\lambda }\left( u,v\right) .
\]%
Evidentemente esta relaci\'{o}n es reflexiva antisime\-trica y
transitiva por lo tanto es un orden parcial.\newline Partiendo de
nuestro orden parcial construimos una sucesi\'{o}n $\left(
S_{n}\right) $ de conjuntos tal que

\begin{equation*}
S_{1}\supset S_{2}\supset S_{3}......
\end{equation*}

Demostramos que los $S_{n}$ son cerrados y que
$diamS_{n}\rightarrow 0.$

Por lo tanto existe un \'{u}nico elemento $u_{\lambda }\in X$ tal que%
\begin{equation*}
\bigcap\limits_{n=1}^{\infty }S_{n}=\left\{ u_{\lambda }\right\}
\end{equation*}

Se prueba que este elemento $u_{\lambda }$ satisface las condiciones $(\ref%
{e1})$, $(\ref{e2})$, $(\ref{e3}).$
\end{proof}
%%%%%%%%%%%%%%%%%%%%%ACETATO 5%%%%%%%%%%%%%%%%%%%%%%%%
\headskip=10pt
\section{{Principio Variacional de Ekeland-forma d�bil}}
\pagedissolve{Wipe /D 1 /Di /H /M /O} \overlay{fondo1.png}
\Large\color{black}
\begin{theorem}
 \label{t2} Sean $\left( X,d\right) $ un espacio m\'{e}trico
completo y $\phi :X\rightarrow \mathbb{R}\cup
\left\{ +\infty \right\} $ una funci\'{o}n semicontinua inferiormente, no id%
\'{e}nticamente igual$\ $a$\ +\infty ,$ y acotada inferiormente.
Entonces dado $\varepsilon >0$ existe $u_{\varepsilon }\in X$ tal
que se cumplen las
siguientes condiciones:%
\begin{eqnarray}
\phi \left( u_{\varepsilon }\right)  &\leq &\inf_{X}\phi
+\varepsilon
\label{e5} \\
\phi \left( u_{\varepsilon }\right)  &\leq &\phi \left( u\right)
+\varepsilon d(u,u_{\varepsilon })\;\forall u\neq u_{\varepsilon
}. \label{e6}
\end{eqnarray}
\end{theorem}
%%%%%%%%%%%%%%%%%%%%%ACETATO 6%%%%%%%%%%%%%%%%%%%%%%%%
\headskip=20pt
\section{{Prueba Forma D�bil}}
\pagedissolve{Wipe /D 1 /Di /H /M /O} \overlay{fondo1.png}
\Large\color{black}
\begin{proof}
Sea $\varepsilon >0.$ Usando la definici\'{o}n de \'{\i}nfimo existe $%
\overline{u}\in X\ $ tal que%
\[
\phi (\overline{u})\leq \inf\limits_{X}\phi +\frac{\varepsilon
}{2}.
\]%
Aplicando el teorema anterior con $\lambda =1$ tenemos que existe $%
u_{\varepsilon }$ tal que se cumplen $(\ref{e5})$ y $(\ref{e6})$
\end{proof}
%%%%%%%%%%%%%%%%%%%%%ACETATO 6%%%%%%%%%%%%%%%%%%%%%%%%
\headskip=20pt
\section{{Aplicaci�n a la Teor�a de Puntos Fijos.}}
\pagedissolve{Wipe /D 1 /Di /H /M /O} \overlay{fondo1.png}
\Large\color{black}
\begin{theorem}
 \label{t8}Sean $X$ un espacio m\'{e}trico completo y
$f:X\rightarrow X$ una funci\'{o}n tal que
\begin{equation}
d(u,f\left( u)\right) \leq \phi \left( u\right) -\phi \left(
f\left( u\right) \right) \ \ \forall u\in X,  \label{e10}
\end{equation}%
donde $\phi :X\rightarrow \mathbb{R}$ es una funci\'{o}n
semicontinua inferiormente y acotada inferiormente.\newline
Entonces existe $v\in X$ tal que $f\left( v\right) =v$
\end{theorem}
%%%%%%%%%%%%%%%%%%%%%ACETATO 7%%%%%%%%%%%%%%%%%%%%%%%%
\headskip=10pt
\section{{Puntos Fijos.}}
\pagedissolve{Wipe /D 1 /Di /H /M /O} \overlay{fondo1.png}
\Large\color{black}
\begin{proof}

Aplicando el Teorema $\ref{t2}$ (forma d\'{e}bil) con $\varepsilon
=\frac{1
}{2}$, existe $v\in X$ tal que%
\[
\phi \left( v\right) -\phi \left( w\right) <\frac{1}{2}d\left(
v,w\right) \quad \forall w\in X.
\]%
Tomando $w=f(v)$ en la desigualdad anterior y aplicando $\left( \ref{e10}%
\right) $ tenemos%
\[
d(v,f(v))\leq \phi \left( v\right) -\phi \left( f(v)\right)
<\frac{1}{2} d\left( v,f\left( v\right) \right) .
\]%
Por lo tanto $d\left( v,f\left( v\right) \right) =0.$ Luego$\ \
f\left( v\right) =v$\
\end{proof}

%%%%%%%%%%%%%%%%%%%%%%%%%%%%%%%%%%%%%%%%%%%%%%%%%%%%%%%%%%%%%
\headskip=20pt
\section{{Puntos Fijos .}}
\pagedissolve{Wipe /D 1 /Di /H /M /O} \overlay{fondo1.png}
\Large\color{black}
\begin{corollary}
 (Punto fijo de Banach) Sean $\left( X,d\right) $ un espacio m%
\'{e}trico completo y $T:X\rightarrow X$ una contracci\'{o}n.
Entonces existe $x_{0}$ tal que $T\left( x_{0}\right) =x_{0}$.

\begin{proof}
Se demuestra que
\begin{eqnarray*}
d\left( x,T\left( x\right) \right)  &\leq &\phi \left( x\right)
-\phi \left(
T\left( x\right) \right)  \\
donde\ \ \phi \left( x\right)  &=&\frac{1}{1-k}d\left( x,T\left(
x\right) \right)
\end{eqnarray*}
El resultado se sigue aplicando el Teorema $\ref{t8}$
\end{proof}
\end{corollary}
%%%%%%%%%%%%%%%%%%%%%%%%%%%%%%%%%%%%%%%%%%
\headskip=10pt
\section{{Aplicaci�n a la Optimizaci�n.}}
\pagedissolve{Wipe /D 1 /Di /H /M /O} \overlay{fondo1.png}
\large\color{black}
\begin{theorem}
\label{o1}Sean $X$ un espacio de Banach y $\phi :X\rightarrow \mathbb{R}$ una funci%
\'{o}n semicontinua inferiormente, acotada inferiormente y
Gateaux-diferenciable $\forall x\in X$. Entonces para cada
$\varepsilon \,>0$ y $\overline{u}$ $\in X$ tales que
\begin{equation}
\phi (\overline{u})\leq \inf_{X}\phi +\varepsilon  \label{d4}
\end{equation}%
existe $u_{\varepsilon }\in X$ tal que%
\begin{equation}
\ \quad \phi \left( u_{\varepsilon }\right) \leq \phi
(\overline{u})\quad \label{d1}
\end{equation}%
\begin{equation}
\left\Vert \overline{u}-u_{\varepsilon }\right\Vert \leq
\sqrt{\varepsilon } \label{d2}
\end{equation}%
\begin{equation}
\quad \left\Vert \phi ^{\prime }\left( u_{\varepsilon }\right)
\right\Vert _{X^{\ast }}\leq \sqrt{\varepsilon }  \label{d3}
\end{equation}
\end{theorem}

%%%%%%%%%%%%%%%%%%%%%%%%%%%%%%%%%%%%%%%%%%%%%%%%
\headskip=20pt
\section{{Palais-Smale}}
\pagedissolve{Wipe /D 1 /Di /H /M /O} \overlay{fondo1.png}
\Large\color{black}
\begin{definition} Sean $X$ un espacio de Banach y\newline $\phi :X\rightarrow \mathbb{R}$ una
funci\'{o}n de clase $C^{1}$. Decimos que $\phi $ satisface la
condici\'{o}n de Palais-Smale si toda sucesi\'{o}n $(u_{n})$ en
$X$ que satisface
\begin{eqnarray*}
\left\vert \phi \left( u_{n}\right) \right\vert  &\leq &k\text{, para algun }%
k\in \mathbb{R}\quad  \\
\phi ^{^{\prime }}\left( u_{n}\right)  &\rightarrow &0\;\text{en
}X^{\ast }
\end{eqnarray*}
tiene una subsucesi\'{o}n convergente en
$X$.\newline\newline\newline
\end{definition}
\begin{theorem}
\label{d19}Sean $X$ un espacio de Banach y\newline $\phi
:X\rightarrow \mathbb{R}$ un funcional de clase $C^{1}$ que
satisface la condici\'{o}n de Palais-Smale y tal que $\phi $
est\'{a} acotado inferiormente. Entonces existe $u_{0}\in X$
tal que%
\[
\begin{tabular}{ccc}
$\inf\limits_{X}\phi =\phi \left( u_{0}\right) $ & \ y & $\phi
^{^{\prime
}}\left( u_{0}\right) =0.$%
\end{tabular}%
\]%
Es decir, el \'{\i}nfimo de $\phi $ se asume en $u_{0}\in X\ \ $y
$u_{0}$ es un punto cr\'{\i}tico de $\phi .$
\end{theorem}
%%%%%%%%%%%%%%%%%%%%%%%%%%%%%%%%%%%%%%%%%%%%%%%%%%%%%%%%%%%%%

\headskip=00pt
\section{{Teoremas de Minimax.}}
\pagedissolve{Wipe /D 1 /Di /H /M /O} \overlay{fondo1.png}
\large\color{black}
\begin{theorem}
\label{t4}(Teorema de Minimax) Sean $K$ un espacio m\'{e}trico
compacto, $K_{0}\;\subset K$ un subconjunto cerrado de $K$, $X$ un
espacio de Banach, $\chi \in C\left( K_{0},X\right) $ y $M$
definido por
\begin{eqnarray*}
M &=&\left\{ g\in C\left( K,X\right) \left/ g\left( s\right) =\chi
\left(
s\right) \;\;\forall s\in K_{0}\right. \right\} . \\
Sea\ \ \phi  &\in &C^{1}\left( X,R\right) .\quad
c=\inf\limits_{g\in M}\max\limits_{s\in K}\phi \left( g\left(
s\right) \right) \ \ y\ \ c_{1}=\max\limits_{\chi \left(
K_{0}\right) }\phi .
\end{eqnarray*}%
\newline
Si $c>c_{1}$ entonces para cada $\varepsilon \;>0$ y cada $f\in M$
\ tales que
\[
\max_{s\in K}\phi \left( f\left( s\right) \right) \leq
c+\varepsilon ,
\]%
existe $v\in X$ tal que
\begin{equation}
c-\varepsilon \mathcal{\;}\leq \phi \left( v\right) \leq
\max_{s\in K}\phi \left( f\left( s\right) \right)   \label{d7}
\end{equation}%
\begin{equation}
dist\left( v,f\left( K\right) \right) \leq \sqrt{\varepsilon
}\quad y\quad
\left\Vert \phi ^{^{\prime }}\left( v\right) \right\Vert \leq \sqrt{%
\varepsilon }  \label{d8}
\end{equation}
\end{theorem}

\begin{proof}
Definamos la funci\'{o}n $\Psi :M\rightarrow \mathbb{R}$ de la siguiente manera%
\[
\Psi \left( g\right) =\max_{S\in K}\phi \left( g\left( s\right)
\right)
\]%
Sea $f\in M$ tal que%
\[
\Psi \left( f\right) =\max_{S\in K}\phi \left( f\left( s\right)
\right) \leq c+\varepsilon .
\]%
Aplicando el Principio Variacional de Ekeland-forma fuerte, existe
$h\in M$ tal que
\begin{equation}
\Psi \left( h\right) \leq \Psi \left( f\right) \leq c+\varepsilon
\label{d11}
\end{equation}%
\begin{equation}
d(h,f)\leq \sqrt{\varepsilon }  \label{d12}
\end{equation}%
\begin{equation}
\Psi \left( h\right) <\Psi \left( g\right) +\sqrt{\varepsilon
}d\left( h,g\right) ,\;\;\;\forall g\in M,\;g\neq h.  \label{d13}
\end{equation}%
Posteriormente demostramos que existe $s\in K$ tal que
\begin{equation}
c-\varepsilon \leq \phi \left( h\left( s\right) \right)\quad y\ \
\ \ \ \left\Vert \phi ^{^{\prime }}\left( h\left( s\right) \right)
\right\Vert \leq \sqrt{\varepsilon }. \label{d14}
\end{equation}%
A partir de (\ref{d14}) se sigue la conclusi�n del teorema.
\end{proof}
%%%%%%%%%%%%%%%%%%%%%%%%%%%%%%%%%%%%%%%%%%%%%%%%%%%%%%

\headskip=20pt
\section{{Teoremas de Minimax.}}
\pagedissolve{Wipe /D 1 /Di /H /M /O} \overlay{fondo1.png}
\large\color{black}
\begin{corollary}

 \label{t5}Sean $K$, $K_{0}$, $X$, $\chi $, $M$, $\phi $,
$c$ y $c_{1}$ definidas como en el teorema anterior y supongamos
que existe $S\;\subset \;X $ tal que
\begin{gather*}
g\left( K\right) \cap S\neq \emptyset \;\text{para todo }g\in M \\
\text{y sea }c_{0}=\inf_{S}\phi .
\end{gather*}%
Si $c_{1}<c_{0}$ entonces $c>c_{1}$ y por lo tanto se tienen todas
las conclusiones del teorema anterior.
\end{corollary}
\begin{corollary}

\label{t7}Supongamos las mismas hip\'{o}tesis del Teorema $\ref{t4}$. Si $%
\phi $ satisface la condici\'{o}n de Palais-Smale entonces $c$ es un valor cr%
\'{\i}tico de $\phi .$
\end{corollary}
%%%%%%%%%%%%%%%%%%%%%%%%%%%%%%%%%%%%%%%%%%%%%%%%%%%

\headskip=40pt
\section{{Teoremas de Minimax.}}
\pagedissolve{Wipe /D 1 /Di /H /M /O} \overlay{fondo1.png}
\Large\color{black}
Escogencias adecuadas de $K,K_{0}\ y\ \chi $ en el Teorema $\ref%
{t4}\ $y corolario $\ref{t5}$ nos proporcionan los dos teoremas
siguientes, que son muy importantes en la teor\'{\i}a de puntos
cr\'{\i}ticos.
 %%%%%%%%%%%%%%%%%%%%%%%%%%%%%%%%%%%%%%%%%%%%%%

\headskip=00pt
\section{{Teorema del Paso de Monta�a.}}
\pagedissolve{Wipe /D 1 /Di /H /M /O} \overlay{fondo1.png}
\large\color{black}
\begin{theorem}
\label{pm} Sean $X$ un espacio de Banach y $\phi \in $\
$C^{1}\left( X,\mathbb{R}\right) $.\newline
Supongamos que existen $u_{0}\in X$, $u_{1}\in X$\ y una vecindad acotada $%
\Omega \ $de $u_{0}$ tal que $u_{1}\in \left( X-\overline{\Omega }\right) $ y%
\begin{gather*}
\inf_{\partial \Omega }\phi >\max \left( \phi \left( u_{0}\right)
,\;\phi
\left( u_{1}\right) \right) . \\
Sean\ \ \Gamma =\left\{ g\in C\left( \left[ 0,1\right] ,X\right) \
\left/ \
g\left( 0\right) =u_{0},g\left( 1\right) =u_{1}\right. \right\} \ y\  \\
c=\inf_{g\in \Gamma }\max_{s\in \left[ 0,1\right] }\phi \left(
g\left( s\right) \right) .
\end{gather*}%
Si $\phi $ satisface la condici\'{o}n de Palais-Smale entonces $c$
es un
valor cr\'{\i}tico de $\phi $ y%
\[
c>\max \left( \phi \left( u_{0}\right) ,\phi \left( u_{1}\right)
\right)
\]%
Es decir, existe $u\in X$ tal que $\phi \left( u\right) =c$ y
$\phi ^{^{\prime }}\left( u\right) =0.$
\end{theorem}

%%%%%%%%%%%%%%%%%%%%%%%%%%%%%%%%%%%%%%%%%%%%%

\headskip=20pt
\section{{Teorema del Punto de Silla.}}
\pagedissolve{Wipe /D 1 /Di /H /M /O} \overlay{fondo1.png}
\large\color{black}
\begin{theorem}
\label{pds} Sean $X$ un espacio de Banach y $\phi
:X\rightarrow \mathbb{R}$ $\ $un funcional de clase $C^{1}$ que satisface la condici%
\'{o}n de Palais-Smale. Supongamos que $X=V\oplus W$, donde $V$ y
$W$ son
subespacios cerrados, con dim$V$ $<\infty $. Sean%
\begin{eqnarray*}
S_{R} &=&\left\{ u\in V\,\,\left/ \ \ \left\Vert u\right\Vert
=R\right. \ \ \right\} \text{,}\quad \overline{B_{R}}=\left\{ u\in
V\,\,\left/ \
\left\Vert u\right\Vert \leq R\right. \right\} , \\
M &=&\left\{ g\in C(\overline{B_{R}},X)\quad \left/ \ \ g(s)=s\text{ si }%
s\in S_{R}\right. \right\} \quad y\quad  \\
c &=&\inf_{f\in M}\max_{s\in \overline{B_{R}}}\phi \left( f\left(
s\right) \right)
\end{eqnarray*}%
Si $\inf\limits_{W}\phi >\max\limits_{S_{R}}\phi \ \ $entonces $c$
es un valor cr\'{\i}tico de $\phi .$
\end{theorem}

%%%%%%%%%%%%%%%%%%%%%%%%%%%%%%%%%%%%%%%%%%%%%%%

\headskip=20pt
\section{{Aplicaciones a las Ecuaciones Diferenciales Parciales.}}
\pagedissolve{Wipe /D 1 /Di /H /M /O} \overlay{fondo1.png}
\large\color{black}
\begin{equation}
\left\{
\begin{tabular}{lll}
$-\Delta u$ & $=f(x,u)\;$ & en \ $\ \Omega $ \\
$\;\;\;u$ & $=0\;\;\;\;\;\;$ & en$\text{ \ }\partial \Omega .$%
\end{tabular}%
\right.   \label{a1}
\end{equation}%
 Por una
\textbf{soluci\'{o}n cl\'{a}sica} de (\ref{a1}) entendemos una
funci\'{o}n\newline $u\in C^{2}\left( \Omega \right) \cap C\left( \overline{\Omega }%
\right) $ que satisface (\ref{a1}).\newline Una
\textbf{soluci\'{o}n d\'{e}bil} de (\ref{a1}) es una funci\'{o}n
$u\in H_{0}^{1}(\Omega )$ que satisface
\begin{equation}
\int_{\Omega }\nabla u.\nabla v=\int_{\Omega }f\left( x,u\right)
v\;\;\;\forall v\in H_{0}^{1}.  \label{a3}
\end{equation}%
Sea $\phi :H_{0}^{1}\rightarrow R$ el funcional definido por%
\begin{equation}
\phi \left( u\right) =\frac{1}{2}\int_{\Omega }\left\vert \nabla
u\right\vert ^{2}-\int_{\Omega }F\left( x,u\right) .  \label{aa1}
\end{equation}%
Supongamos que existe una constante $c>0$ tal que%
\begin{equation}
\left\vert f\left( x,s\right) \right\vert \leq c\left\vert
s\right\vert ^{p-1}+b\left( x\right) ,\ b\left( x\right) \in
L^{p^{\prime }},\ 1\leq p\leq \frac{2N}{N-2} \label{ac1}
\end{equation}%
Entonces  $\phi $ es continuamente Fr\'{e}chet
Diferenciable y%
\begin{equation}
\left\langle \phi ^{^{\prime }}\left( u\right) ,v\right\rangle
=\int_{\Omega }\nabla u.\nabla v-\int_{\Omega }f\left( x,u\right)
v\;\;\;\forall v\in H_{0}^{1}.  \label{aa2}
\end{equation}%
De (\ref{a3}) y (\ref{aa2}) se sigue que $u\ $es una soluci\'{o}n
d\'{e}bil
del problema (\ref{a1}) si y s\'{o}lo si $u$ es un punto cr\'{\i}tico de $%
\phi .$
%%%%%%%%%%%%%%%%%%%%%%%%%%%%%%%%%%%%%%%%%%%%%
\headskip=20pt
\section{{Aplicaci�n Utilizando El Principio Variacional de Ekeland}}
\pagedissolve{Wipe /D 1 /Di /H /M /O} \overlay{fondo1.png}
\Large\color{black}
\begin{theorem}
\label{45}Sea $f:\mathbb{R}\rightarrow \mathbb{R}\ \ $una funci\'{o}n continua tal que\newline $%
\lim\limits_{s\rightarrow -\infty }\frac{f(s)}{s}=a\,\,$ y \ $%
\lim\limits_{s\rightarrow \infty }\frac{f(s)}{s}=b, \newline $con
$a$ y $b\in (0,\lambda _{1}).$ Entonces el problema
\begin{equation}
\left\{
\begin{tabular}{rrr}
$\Delta u+f(u)$ & $=0$ & en$\;\ $ $\Omega \ $ \\
$u$ & $=0$ & en$\text{ \ \ }\partial \Omega $%}
\end{tabular}%
\right.   \label{a110}
\end{equation}%
tiene una soluci\'{o}n d\'{e}bil.
\end{theorem}
%%%%%%%%%%%%%%%%%%%%%%%%%%%%%%%%%%%%%%%%%%%%%%%%%%%%%%%%%%%%%%%%%%%%%

\headskip=20pt
\section{{Aplicaci\'{o}n utilizando el Teorema de Punto de
silla.}} \pagedissolve{Wipe /D 1 /Di /H /M /O}
\overlay{fondo1.png} \large\color{black}
\begin{theorem}{(Dolph).}
Supongamos que $h\in C(\overline{\Omega })$ y $f:\mathbb{R}\rightarrow \mathbb{R}$ es una funci%
\'{o}n continua tal que existen constantes $\alpha \ $y $\beta $ $\in R\ \ $%
con\newline$\ \ \lambda _{k}<\alpha ,\beta <\lambda _{k+1}$,
\begin{equation}
\lim_{s\rightarrow -\infty }\frac{f(s)}{s}=\alpha \text{ \ \ y \ \ \ \ }%
\lim_{s\rightarrow \infty }\frac{f(s)}{s}=\beta .  \label{a16}
\end{equation}%
Entonces el problema
\begin{equation}
\left\{
\begin{tabular}{lll}
$-\Delta u$ & $=f(u)+h(x)$ & en$\ \ \ \Omega $ \\
$\ \ \ \ \ u$ & $=0\;\;\;\;\;\;\;\;\;\;$ & en$\text{ \ }\partial \Omega $%
\end{tabular}%
\right.   \label{a17}
\end{equation}%
tiene una soluci\'{o}n d\'{e}bil.
\end{theorem}

%%%%%%%%%%%%%%%%%%%%%%%%%%%%%%%%%%%%%%%%%%%%%

\headskip=00pt
\section{{Aplicaci�n Utilizando El Teorema del Paso de la Monta�a}}
\pagedissolve{Wipe /D 1 /Di /H /M /O} \overlay{fondo1.png}
\large\color{black}
\begin{theorem}{(A. Ambrosetti y P. Rabinowitz).}
 Supongamos que $f$ sa\-tisface las siguientes condiciones
\begin{equation*}
\left\vert f\left( x,s\right) \right\vert \leq c\left\vert
s\right\vert ^{p-1}+b\left( x\right) ,\ donde\ c>0,\; b\left(
x\right) \in L^{p^{\prime }},\ 1\leq p<\frac{2N}{N-2}
\end{equation*}%
y$\ $supongamos que existen%
\begin{equation*}
\theta >2\ y\ s_{0}>0\ \ tales\ que\ \ 0<\theta F(x,s)\leq
sf(x,s)\quad \forall x\in \overline{\Omega }\quad \forall
\left\vert s\right\vert \geq s_{0}
\end{equation*}%
Adicionalmente supongamos que%
\begin{equation}
\lim_{s\rightarrow 0}\frac{f(x,s)}{s}<\lambda _{1}  \label{a20}
\end{equation}%
Entonces el problema
\begin{equation}
\left\{
\begin{tabular}{lll}
$-\Delta u$ & $=f(x,u)$ & en$\ \ \ \Omega $ \\
$\ \ \ \ \ u$ & $=0\;\;\;\;\;\;\;\;\;\;$ & en$\text{ \ }\partial \Omega $%
\end{tabular}%
\right.   \label{a21}
\end{equation}%
tiene una soluci\'{o}n d\'{e}bil no trivial.
\end{theorem}

%%%%%%%%%%%%%%%%%%%%%%%%%%%%%%%%%%%%%%%%%%%

\newpage

\thankyouslide[!`Gracias!]


%%%%%%%%%%%%%%%%%%%%%%%%%%%%%%%%%%%%%%%%%%%%%

%%%%%%%%%%%%%%%%%%%%%%%%%%%%%%%%%%%%%%%%%%%%%%%%%%%%
\end{document}
