



\begin{theorem}
Sea $L$ el operador diferencial dado en $\left(  1\right)  $ con coeficientes
satisfaciendo $\left(  2\right)  ,\left(  3\right)  $ y $\left(  4\right)  $,
y $g\in W^{1,2}\left(  \Omega\right)  $. Entonces para cualquier $f\in
L^{2}\left(  \Omega\right)  $ el \textbf{problema de Dirichlet generalizado}
\[
\left(  P.D.\right)  \left\{
\begin{array}
[c]{c}%
Lu=f\text{ \ en }\Omega,\\
u=g\text{ sobre }\partial\Omega,
\end{array}
\right.
\]
tiene soluci\'{o}n d\'{e}bil \'{u}nica.
\end{theorem}

%\begin{proof}
%Si $u$ es una soluci\'{o}n d\'{e}bil de $\left(  P.D.\right)  $, entonces
%$w:=u-g\in W_{0}^{1,2}\left(  \Omega\right)  $ y adem\'{a}s para cada $v\in
%W_{0}^{1,2}\left(  \Omega\right)  :$%
%\[
%\mathcal{L}\left(  w,v\right)  =\widetilde{F}\left(  v\right)  ,\text{
%\ }\widetilde{F}\left(  v\right)  \in\left(  W_{0}^{1,2}\left(  \Omega\right)
%\right)  ^{\ast}.
%\]
%
%
%$\mathcal{L}\left(  .,.\right)  $ es una forma bilineal acotada sobre
%$W_{0}^{1,2}\left(  \Omega\right)  \times W_{0}^{1,2}\left(  \Omega\right)  $
%y existe $\sigma_{0}$ tal que
%\[
%\mathcal{L}_{\sigma_{0}}\left(  u,v\right)  :=\mathcal{L}\left(  u,v\right)
%+\sigma_{0}\left(  u,v\right)
%\]
%es coerciva. Luego Lax-Milgram implica que para $F\in\left(  W_{0}%
%^{1,2}\left(  \Omega\right)  \right)  ^{\ast}$ existe un \'{u}nico $u\in
%W_{0}^{1,2}\left(  \Omega\right)  $ tal que
%\[
%\mathcal{L}_{\sigma_{0}}\left(  u,v\right)  =F\left(  v\right)  \text{ \ para
%todo }v\in W_{0}^{1,2}\left(  \Omega\right)  .
%\]
%La aplicaci\'{o}n $L_{\sigma_{0}}:W_{0}^{1,2}\left(  \Omega\right)
%\longrightarrow\left(  W_{0}^{1,2}\left(  \Omega\right)  \right)  ^{\ast}$,
%dada por
%\[
%(L_{\sigma_{0}}u)\left(  v\right)  :=\mathcal{L}_{\sigma_{0}}\left(
%u,v\right)  \text{ \ }\forall v\in W_{0}^{1,2}\left(  \Omega\right)
%\]
%es lineal, acotada uno-uno, sobre y $L_{\sigma_{0}}^{-1}$ tambi\'{e}n es acotada.
%
%As\'{\i} las ecuaciones $Lu=F$ y $u-\sigma_{0}L_{\sigma_{0}}^{-1}%
%Iu=L_{\sigma_{0}}^{-1}F$ son equivalentes. El operador $T:=\sigma_{0}%
%L_{\sigma_{0}}^{-1}I$ es compacto, debido al lema $5$, y por el corolario $3$
%se deduce que $I-T$ es inyectiva. Luego por la alternativa de Fredholm se da
%la conclusi\'{o}n del teorema.
%\end{proof}
