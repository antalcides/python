
\documentclass{article}
%%%%%%%%%%%%%%%%%%%%%%%%%%%%%%%%%%%%%%%%%%%%%%%%%%%%%%%%%%%%%%%%%%%%%%%%%%%%%%%%%%%%%%%%%%%%%%%%%%%%%%%%%%%%%%%%%%%%%%%%%%%%
\usepackage{graphicx}
\usepackage{amsmath}

%TCIDATA{OutputFilter=LATEX.DLL}
%TCIDATA{Created=Fri Nov 14 13:09:39 2003}
%TCIDATA{LastRevised=Fri Nov 14 17:46:18 2003}
%TCIDATA{<META NAME="GraphicsSave" CONTENT="32">}
%TCIDATA{<META NAME="DocumentShell" CONTENT="General\Blank Document">}
%TCIDATA{CSTFile=LaTeX article (bright).cst}

\newtheorem{theorem}{Theorem}
\newtheorem{acknowledgement}[theorem]{Acknowledgement}
\newtheorem{algorithm}[theorem]{Algorithm}
\newtheorem{axiom}[theorem]{Axiom}
\newtheorem{case}[theorem]{Case}
\newtheorem{claim}[theorem]{Claim}
\newtheorem{conclusion}[theorem]{Conclusion}
\newtheorem{condition}[theorem]{Condition}
\newtheorem{conjecture}[theorem]{Conjecture}
\newtheorem{corollary}[theorem]{Corollary}
\newtheorem{criterion}[theorem]{Criterion}
\newtheorem{definition}[theorem]{Definition}
\newtheorem{example}[theorem]{Example}
\newtheorem{exercise}[theorem]{Exercise}
\newtheorem{lemma}[theorem]{Lemma}
\newtheorem{notation}[theorem]{Notation}
\newtheorem{problem}[theorem]{Problem}
\newtheorem{proposition}[theorem]{Proposition}
\newtheorem{remark}[theorem]{Remark}
\newtheorem{solution}[theorem]{Solution}
\newtheorem{summary}[theorem]{Summary}
\newenvironment{proof}[1][Proof]{\textbf{#1.} }{\ \rule{0.5em}{0.5em}}
\input{tcilatex}

\begin{document}


\begin{center}
ASIGNACION N$%
%TCIMACRO{\UNICODE[m]{0xb0}}%
%BeginExpansion
{{}^\circ}%
%EndExpansion
4$ DE FISICA MATEMATICA \ Y

COMPUTACIONAL.
\end{center}

\begin{enumerate}
\item  Halle la aproximaci\'{o}n en diferencias hacia delante y hacia
atr\'{a}s de orden $\left( \Delta x\right) $para $\frac{\partial ^{6}f}{%
\left( \partial x\right) ^{6}}$.(sugerencia: utilice las ecuaciones 19,20 y
21 de las notas)\newline
SOLUCION

\item[a.]  Hacia adelante\newline
\begin{eqnarray*}
\frac{\partial ^{6}f}{\left( \partial x\right) ^{6}} &=&\frac{\left( \Delta
^{6}xf_{i}\right) }{\left( \Delta x\right) ^{6}} \\
&=&\frac{\Delta x^{5}\left( \Delta xf_{i}\right) }{\left( \Delta x\right)
^{6}} \\
&=&\frac{\Delta x^{5}\left( f_{i+1}-f_{i}\right) }{\left( \Delta x\right)
^{6}} \\
&=&\frac{\Delta x^{4}\left( \Delta x\left( f_{i+1}-f_{i}\right) \right) }{%
\left( \Delta x\right) ^{6}} \\
&=&\frac{\Delta x^{4}\left( f_{i+2}-2f_{i+1}+f_{i}\right) }{\left( \Delta
x\right) ^{6}} \\
&=&\frac{\Delta x^{3}\left( \Delta x\left( f_{i+2}-2f_{i+1}+f_{i}\right)
\right) }{\left( \Delta x\right) ^{6}} \\
&=&\frac{\Delta x^{3}\left( f_{i+3}-3f_{i+2}+3f_{i+1}-f_{i}\right) }{\left(
\Delta x\right) ^{6}} \\
&=&\frac{\Delta x^{2}\left( \Delta x\left(
f_{i+3}-3f_{i+2}+3f_{i+1}-f_{i}\right) \right) }{\left( \Delta x\right) ^{6}}
\\
&=&\frac{\Delta x^{2}\left( f_{i+4}-4f_{i+3}+6f_{i+2}-4f_{i+1}+f_{i}\right) 
}{\left( \Delta x\right) ^{6}} \\
&=&\frac{\Delta x\left( \Delta x\left(
f_{i+4}-4f_{i+3}+6f_{i+2}-4f_{i+1}+f_{i}\right) \right) }{\left( \Delta
x\right) ^{6}} \\
&=&\frac{\Delta x\left(
f_{i+5}-5f_{i+4}+10f_{i+3}-10f_{i+2}+5f_{i+1}-f_{i}\right) }{\left( \Delta
x\right) ^{6}} \\
\frac{\partial ^{6}f}{\partial x^{6}} &=&\frac{\left(
f_{i+6-}6f_{i+5}+15f_{i+4}-20f_{i+3}+15f_{i+2}-6f_{i+1}+f_{i}\right) }{%
\left( \Delta x\right) ^{6}}
\end{eqnarray*}
\end{enumerate}

\bigskip 

\bigskip 

\bigskip 

\begin{enumerate}
\item[b.]  Hacia atr\'{a}s:\newline
\begin{eqnarray*}
\frac{\partial ^{6}f}{\left( \partial x\right) ^{6}} &=&\frac{\left( \nabla
^{6}xf_{i}\right) }{\left( \Delta x\right) ^{6}} \\
&=&\frac{\nabla ^{5}x\left( \Delta xf_{i}\right) }{\left( \Delta x\right)
^{6}} \\
&=&\frac{\nabla ^{5}x\left( f_{i}-f_{i-1}\right) }{\left( \Delta x\right)
^{6}} \\
&=&\frac{\nabla ^{4}x\left( \nabla x\left( f_{i}-f_{i-1}\right) \right) }{%
\left( \Delta x\right) ^{6}} \\
&=&\frac{\Delta x^{4}\left( f_{i}-2f_{i-1}+f_{i-2}\right) }{\left( \Delta
x\right) ^{6}} \\
&=&\frac{\nabla ^{3}x\left( \nabla x\left( f_{i}-2f_{i-1}+f_{i-2}\right)
\right) }{\left( \Delta x\right) ^{6}} \\
&=&\frac{\nabla ^{3}x\left( f_{i}-3f_{i-1}+3f_{i-2}-f_{i-3}\right) }{\left(
\Delta x\right) ^{6}} \\
&=&\frac{\nabla ^{2}x\left( \nabla x\left(
f_{i}-3f_{i-1}+3f_{i-2}-f_{i-3}\right) \right) }{\left( \Delta x\right) ^{6}}
\\
&=&\frac{\nabla ^{2}x\left( f_{i}-4f_{i-1}+6f_{i-2}-4f_{i-3}+f_{i-4}\right) 
}{\left( \Delta x\right) ^{6}} \\
&=&\frac{\nabla x\left( \nabla x\left(
f_{i}-4f_{i-1}+6f_{i-2}-4f_{i-3}+f_{i-4}\right) \right) }{\left( \Delta
x\right) ^{6}} \\
&=&\frac{\nabla x\left(
f_{i}-5f_{i-1}+10f_{i-2}-10f_{i-3}+5f_{i-4}-f_{i-5}\right) }{\left( \Delta
x\right) ^{6}} \\
\frac{\partial ^{6}f}{\partial x^{6}} &=&\frac{\left(
f_{i-}6f_{i-1}+15f_{i-2}-20f_{i-3}+15f_{i-4}-6f_{i-5}+f_{i-6}\right) }{%
\left( \Delta x\right) ^{6}}
\end{eqnarray*}

\item[2.]  Dada la funci\'{o}n escriba un programa donde calcule las
primeras dos derivadas en x = 1.5 en diferencias centradas de orden para los
siguientes 'step sizes' 0.0005, 0.001, 0.01, 0.1, 0.2. Determine el error num%
\'{e}rico para cada c\'{o}mputo\newline
SOLUCION: \newline
A continuaci\'{o}n \ mostraremos el c\'{o}digo del programa C
Especializacion En Fisica General
\end{enumerate}

\begin{verbatim}
C     Especializacion En Fisica General
C NOMBRE: Derivada
C AUTOR: Antalcides Olivo
C DESCRIPCION: Este programa calcula las derivadas parciales usando
C              diferncias finitas .
C FECHA: 07/11/03 17.50
CCCCCCCCCCCCCCCCCCCCCCCCCCCCCCCCCCCCCCCCCCCCCCCCCCCCCCCCCCCCCCCCCCCCCCCC
C
 
      program derivadas
      integer i,j ,k
 
      real difc(5), dif2(5),dx(5),x,A ,b
      real errorc(5),error2(5)
      Write(*,'(T5,A/)')
     & '***************************************************************'
C     Introducion:
      Write(6,'(T5,A/)')
     & '\UNICODE{0xc5}*******************\UNICODE{0xc5}********************\UNICODE{0xc5}********************\UNICODE{0xc5}'
      write(6,'(T5,A/)')
     & ' calcule la 1\UNICODE{0xa7} y 2\UNICODE{0xa6} derivada de f(x) = sin((pi/2)*x)',
     & 'para los valores de dx 0.0005,0.001,0.01,0.1 y 0.2'
      Write(6,'(T5,A/)')
     & '\UNICODE{0xc5}****************\UNICODE{0xc5}*****************\UNICODE{0xc5}**************************\UNICODE{0xc5}'
      Write(*,'(T5,A/)')
     & '***************************************************************'
      Write(*,*) '                                                     '
      Write(*,*) '                                                     '
 
C**********************************************************************
      !definicion de las variables
      ! dift indica que realizaremos la derivada hacia adelante
      ! dift indica que realizaremos la derivada hacia atras
      ! difc indica que realizaremos la derivada centrada
      ! dx es el paso
      ! errord,errort,errorc representan los errores de las derivadas
      ! definidas anteriormente
      ! X representa el valor numerico de la derivada
      ! A representa el valor analitico de la derivada
C***********************************************************************
 
C     valor a evaluar:
      Write(6,'(T5,A/)') 'Ingrese el valor x a evaluar'
      read (5,*)  x
C     tama\UNICODE{0xa4}o del paso
      dx(1)=0.0005
      dx(2)=0.001
      dx(3)=0.01
      dx(4)=0.1
      dx(5)=0.2
C      ** Valores analiticos de las derivadas
      A=fun1(x)
      b=fun2(x)
      Write(*,*) '                                                     '
      Write(*,*) '                                                     '
      Write(*,*) '                                                     '
 
       WRITE(*,'(3(T5,A/),3(T5,A/T5,(3(A,1PE13.6))/))')
     &' 1\UNICODE{0xa7} Derivada  \UNICODE{0xb3} Valor         \UNICODE{0xb3}  2\UNICODE{0xa6} Derivada  \UNICODE{0xb3}              ',
     &'  diferencia  \UNICODE{0xb3}  del          \UNICODE{0xb3}   diferencia  \UNICODE{0xb3}    N\UNICODE{0xa7} de paso',
     &'    centrada  \UNICODE{0xb3}   paso        \UNICODE{0xb3}    centrada   \UNICODE{0xb3}              ',
     &'              \UNICODE{0xb3}               \UNICODE{0xb3}               \UNICODE{0xb3}              ',
     &'\UNICODE{0xc4}\UNICODE{0xc4}\UNICODE{0xc4}\UNICODE{0xc4}\UNICODE{0xc4}\UNICODE{0xc4}\UNICODE{0xc4}\UNICODE{0xc4}\UNICODE{0xc4}\UNICODE{0xc4}\UNICODE{0xc4}\UNICODE{0xc4}\UNICODE{0xc4}\UNICODE{0xc4}\UNICODE{0xc5}\UNICODE{0xc4}\UNICODE{0xc4}\UNICODE{0xc4}\UNICODE{0xc4}\UNICODE{0xc4}\UNICODE{0xc4}\UNICODE{0xc4}\UNICODE{0xc4}\UNICODE{0xc4}\UNICODE{0xc4}\UNICODE{0xc4}\UNICODE{0xc4}\UNICODE{0xc4}\UNICODE{0xc4}\UNICODE{0xc4}\UNICODE{0xc5}\UNICODE{0xc4}\UNICODE{0xc4}\UNICODE{0xc4}\UNICODE{0xc4}\UNICODE{0xc4}\UNICODE{0xc4}\UNICODE{0xc4}\UNICODE{0xc4}\UNICODE{0xc4}\UNICODE{0xc4}\UNICODE{0xc4}\UNICODE{0xc4}\UNICODE{0xc4}\UNICODE{0xc4}\UNICODE{0xc4}\UNICODE{0xc5}\UNICODE{0xc4}\UNICODE{0xc4}\UNICODE{0xc4}\UNICODE{0xc4}\UNICODE{0xc4}\UNICODE{0xc4}\UNICODE{0xc4}\UNICODE{0xc4}\UNICODE{0xc4}\UNICODE{0xc4}\UNICODE{0xc4}\UNICODE{0xc4}\UNICODE{0xc4}\UNICODE{0xc4}'                                                    '
C     calculamos la derivada:
      do 10 i=1,5
 
      difc(i)=(fun(x+dx(i))-fun(x-dx(i)))/(2*dx(i))
      dif2(i)=(fun(x+dx(i)) -2*fun(x)+fun(x-dx(i)))/(dx(i)**2)
      write(6,100) difc(i),dif2(i),dx(i),i
100   format(t5,f8.5,10x,f8.5,10x,f8.5,10x,f8.5,f8.5,i4)
      open(20,file='derivadatex.dat',status='unknown')
      write(20,105)'La primera derivada es : ', difc(i),i
      write(20,205)'La segunda derivada es :',dif2(i),i
105   format(a25,2x,f8.5,2x,i2)
205   format(a25,2x,f8.5,2x,i2)
      Write(*,*) '                                                     '
      Write(*,*) '                                                     '
10    continue
       WRITE(*,'(3(T5,A/),3(T5,A/T5,(3(A,1PE13.6))/))')
     &'     Error    \UNICODE{0xb3}    Valor  del   \UNICODE{0xb3}  Error        \UNICODE{0xb3}              ',
     &'     En %     \UNICODE{0xb3}     paso        \UNICODE{0xb3}  En %         \UNICODE{0xb3}    N\UNICODE{0xa7} de paso'
 
      do 30 j=1,5
 
      errorc(j)=abs(((A-difc(j))/A)*100)
      error2(j)=abs(((B-dif2(j))/B)*100)
      Write(*,*) '                                                     '
      Write(*,*) '                                                     '
 
        write(6,355) errorc(j),dx(j),error2(j),j
355   format(t5,f8.5,10x,f8.5,10x,f8.5,10x,i4)
 
405   format(a35,2x,f8.5,2x,i2)
30      continue
 
 
 
      close(20)
      do 50 k=1,5
 
      open(40,file='derivadac.dat',status='unknown')
      open(60,file='derivada2.dat',status='unknown')
      write(40,*) dx(k),errorc(k)
      write(60,*) dx(k),error2(k)
50    continue
      close(40)
      close(60)
      end
C***********************************************************************
C           Definimos la funcion a evaluar:
C***********************************************************************
      function fun(X)
      real X ,t
      parameter (pi=3.1416)
      t=pi/2.0
      fun = sin(t*x)
      end
C***********************************************************************
C     Primera derivada
      function fun1(X)
      real X ,t
      parameter (pi=3.1416)
      t=pi/2.0
      fun1 = t*cos(t*x)
      end
C***********************************************************************
C     Segunda derivada
      function fun2(X)
      real X ,t
      parameter (pi=3.1416)
      t=pi/2.0
      fun2 = -(t**2)*sin(t*x)
      end
 
\end{verbatim}

De este c\'{o}digo se obtiene la siguiente salida

\ \ \ \ \ \ \ \ \ \ \ \ \ \ \ \ \ \ \ \ \ \ \ \ \ \ \ \ \ \ \ \ \ \ \ \ \ \
\ \ \ \ \ \ \ \ \ \ \ \ derivada \ \ \ \ \ \ \ paso
\begin{verbatim}
La primera derivada es :   -1.11067   0.00050
 La segunda derivada es :  -1.90735   0.00050
La primera derivada es :   -1.11079   0.00100
 La segunda derivada es :  -1.78814   0.00100
La primera derivada es :   -1.11068   0.01000
 La segunda derivada es :  -1.74463   0.01000
La primera derivada es :   -1.10617   0.10000
 La segunda derivada es :  -1.74114   0.10000
La primera derivada es :   -1.09255   0.20000
 La segunda derivada es :  -1.73041   0.20000
 
                                       Error     Pasos
 El error de la primera derivada es   0.00508   0.00050
 El error de la segunda derivada es   9.32151   0.00050
 El error de la primera derivada es   0.00566   0.00100
 El error de la segunda derivada es   2.48892   0.00100
 El error de la primera derivada es   0.00427   0.01000
 El error de la segunda derivada es   0.00496   0.01000
 El error de la primera derivada es   0.41071   0.10000
 El error de la segunda derivada es   0.20516   0.10000
 El error de la primera derivada es   1.63682   0.20000
 El error de la segunda derivada es   0.81971   0.20000
 
\end{verbatim}

\begin{enumerate}
\item[3.]  Dada la funci\'{o}n escriba un programa para calcular en x = 0.5
y x = 1.5 por diferencias hacia delante y hacia atr\'{a}s de orden y
diferencias centradas de orden . Use 'step sizes' de 0.00001, 0.0001, 0.001,
0.1 y 0.5. Haga una gr\'{a}fica del error para cada una de los c\'{o}mputos.
\end{enumerate}

Soluci\'{o}n:

A continuaci\'{o}n presentmos el c\'{o}digo del programa.
\begin{verbatim}
CCCCCCCCCCCCCCCCCCCCCCCCCCCCCCCCCCCCCCCCCCCCCCCCCCCCCCCCCCCCCCCCCCCCCCCC
C     Curso de Fisica Matematica  Y Computacional
C     Especializacion En Fisica General
C NOMBRE: Derivada
C AUTOR: Antalcides Olivo
C DESCRIPCION: Este programa calcula las derivadas parciales usando
C              diferncias finitas .
C FECHA: 07/11/03 17.50
CCCCCCCCCCCCCCCCCCCCCCCCCCCCCCCCCCCCCCCCCCCCCCCCCCCCCCCCCCCCCCCCCCCCCCCC
C
 
      program derivadas
      integer i,j ,k
 
      real dift(5),difd(5), dif2(5),dx(5),x,A ,b
      real error2(5),errord(5),errort(5)
      Write(*,'(T5,A/)')
     & '***************************************************************'
C     Introducion:
      Write(6,'(T5,A/)')
     & '\UNICODE{0xc5}*******************\UNICODE{0xc5}********************\UNICODE{0xc5}********************\UNICODE{0xc5}'
      write(6,'(T5,A/)')
     & ' calcule la 1\UNICODE{0xa7} y 2\UNICODE{0xa6} derivada de f(x) = (x**3)-5*x',
     & 'para los valores de dx 0.0005,0.001,0.01,0.1 y 0.2'
      Write(6,'(T5,A/)')
     & '\UNICODE{0xc5}****************\UNICODE{0xc5}*****************\UNICODE{0xc5}**************************\UNICODE{0xc5}'
      Write(*,'(T5,A/)')
     & '***************************************************************'
      Write(*,*) '                                                     '
      Write(*,*) '                                                     '
 
C**********************************************************************
      !definicion de las variables
      ! dift indica que realizaremos la derivada hacia adelante
      ! dift indica que realizaremos la derivada hacia atras
      ! difc indica que realizaremos la derivada centrada
      ! dx es el paso
      ! errord,errort,errorc representan los errores de las derivadas
      ! definidas anteriormente
      ! X representa el valor numerico de la derivada
      ! A representa el valor analitico de la derivada
C***********************************************************************
 
C     valor a evaluar:
      Write(6,'(T5,A/)') 'Ingrese el valor x a evaluar'
      read (5,*)  x
C     tama\UNICODE{0xa4}o del paso
      dx(1)=0.00001
      dx(2)=0.0001
      dx(3)=0.001
      dx(4)=0.1
      dx(5)=0.5
C      ** Valores analiticos de las derivadas
      A=fun1(x)
      b=fun2(x)
      open(15,file='derivadatxt.dat',status='unknown')
       Write(15,'(T5,A/)')
     & '\UNICODE{0xc5}*******************\UNICODE{0xc5}********************\UNICODE{0xc5}********************\UNICODE{0xc5}'
      write(15,'(T5,A/)')
     & ' calcule la 1\UNICODE{0xa7} y 2\UNICODE{0xa6} derivada de f(x) = (x**3)-5*x',
     & 'para los valores de dx 0.0005,0.001,0.01,0.1 y 0.2'
      Write(15,'(T5,A/)')
     & '\UNICODE{0xc5}****************\UNICODE{0xc5}*****************\UNICODE{0xc5}**************************\UNICODE{0xc5}'
      Write(15,'(T5,A/)')
     & '***************************************************************'
      Write(15,*) '                                                    '
      Write(15,*) '                                                    '
      Write(15,*) '                                                   '
 
       WRITE(15,'(3(T5,A/),3(T5,A/T5,(3(A,1PE13.6))/))')
     &' 1\UNICODE{0xa7} Derivada  \UNICODE{0xb3} 1\UNICODE{0xa7} Derivada   \UNICODE{0xb3}  2\UNICODE{0xa6} Derivada  \UNICODE{0xb3}              ',
     &'  diferencia  \UNICODE{0xb3}  diferencia   \UNICODE{0xb3}   diferencia  \UNICODE{0xb3}    N\UNICODE{0xa7} de paso',
     &'    hacia     \UNICODE{0xb3}    hacia      \UNICODE{0xb3}    centrada   \UNICODE{0xb3}              ',
     &'  adelante    \UNICODE{0xb3}    atras      \UNICODE{0xb3}               \UNICODE{0xb3}              ',
     &'\UNICODE{0xc4}\UNICODE{0xc4}\UNICODE{0xc4}\UNICODE{0xc4}\UNICODE{0xc4}\UNICODE{0xc4}\UNICODE{0xc4}\UNICODE{0xc4}\UNICODE{0xc4}\UNICODE{0xc4}\UNICODE{0xc4}\UNICODE{0xc4}\UNICODE{0xc4}\UNICODE{0xc4}\UNICODE{0xc5}\UNICODE{0xc4}\UNICODE{0xc4}\UNICODE{0xc4}\UNICODE{0xc4}\UNICODE{0xc4}\UNICODE{0xc4}\UNICODE{0xc4}\UNICODE{0xc4}\UNICODE{0xc4}\UNICODE{0xc4}\UNICODE{0xc4}\UNICODE{0xc4}\UNICODE{0xc4}\UNICODE{0xc4}\UNICODE{0xc4}\UNICODE{0xc5}\UNICODE{0xc4}\UNICODE{0xc4}\UNICODE{0xc4}\UNICODE{0xc4}\UNICODE{0xc4}\UNICODE{0xc4}\UNICODE{0xc4}\UNICODE{0xc4}\UNICODE{0xc4}\UNICODE{0xc4}\UNICODE{0xc4}\UNICODE{0xc4}\UNICODE{0xc4}\UNICODE{0xc4}\UNICODE{0xc4}\UNICODE{0xc5}\UNICODE{0xc4}\UNICODE{0xc4}\UNICODE{0xc4}\UNICODE{0xc4}\UNICODE{0xc4}\UNICODE{0xc4}\UNICODE{0xc4}\UNICODE{0xc4}\UNICODE{0xc4}\UNICODE{0xc4}\UNICODE{0xc4}\UNICODE{0xc4}\UNICODE{0xc4}\UNICODE{0xc4}'
      Write(*,*) '                                                     '
 
       WRITE(*,'(3(T5,A/),3(T5,A/T5,(3(A,1PE13.6))/))')
     &' 1\UNICODE{0xa7} Derivada  \UNICODE{0xb3} 1\UNICODE{0xa7} Derivada   \UNICODE{0xb3}  2\UNICODE{0xa6} Derivada  \UNICODE{0xb3}              ',
     &'  diferencia  \UNICODE{0xb3}  diferencia   \UNICODE{0xb3}   diferencia  \UNICODE{0xb3}    N\UNICODE{0xa7} de paso',
     &'    hacia     \UNICODE{0xb3}    hacia      \UNICODE{0xb3}    centrada   \UNICODE{0xb3}              ',
     &'  adelante    \UNICODE{0xb3}    atras      \UNICODE{0xb3}               \UNICODE{0xb3}              ',
     &'\UNICODE{0xc4}\UNICODE{0xc4}\UNICODE{0xc4}\UNICODE{0xc4}\UNICODE{0xc4}\UNICODE{0xc4}\UNICODE{0xc4}\UNICODE{0xc4}\UNICODE{0xc4}\UNICODE{0xc4}\UNICODE{0xc4}\UNICODE{0xc4}\UNICODE{0xc4}\UNICODE{0xc4}\UNICODE{0xc5}\UNICODE{0xc4}\UNICODE{0xc4}\UNICODE{0xc4}\UNICODE{0xc4}\UNICODE{0xc4}\UNICODE{0xc4}\UNICODE{0xc4}\UNICODE{0xc4}\UNICODE{0xc4}\UNICODE{0xc4}\UNICODE{0xc4}\UNICODE{0xc4}\UNICODE{0xc4}\UNICODE{0xc4}\UNICODE{0xc4}\UNICODE{0xc5}\UNICODE{0xc4}\UNICODE{0xc4}\UNICODE{0xc4}\UNICODE{0xc4}\UNICODE{0xc4}\UNICODE{0xc4}\UNICODE{0xc4}\UNICODE{0xc4}\UNICODE{0xc4}\UNICODE{0xc4}\UNICODE{0xc4}\UNICODE{0xc4}\UNICODE{0xc4}\UNICODE{0xc4}\UNICODE{0xc4}\UNICODE{0xc5}\UNICODE{0xc4}\UNICODE{0xc4}\UNICODE{0xc4}\UNICODE{0xc4}\UNICODE{0xc4}\UNICODE{0xc4}\UNICODE{0xc4}\UNICODE{0xc4}\UNICODE{0xc4}\UNICODE{0xc4}\UNICODE{0xc4}\UNICODE{0xc4}\UNICODE{0xc4}\UNICODE{0xc4}'
C     calculamos la derivada:
      do 10 i=1,5
      difd(i)=(fun(x+dx(i))-fun(x))/dx(i)
      dift(i)=(fun(x)-fun(x-dx(i)))/dx(i)
 
      dif2(i)=(fun(x+dx(i)) -2*fun(x)+fun(x-dx(i)))/(dx(i)**2)
      write(6,100) difd(i), dift(i),dif2(i),dx(i)
 
100   format(t5,f8.5,10x,f8.5,10x,f8.5,10x,f8.5)
      write(15,105) difd(i), dift(i),dif2(i),dx(i)
 
105   format(t5,f8.5,10x,f8.5,10x,f8.5,10x,f8.5)
 
 
10    continue
      write(6,'(T5,A/)')'Resultado analitico'
      write(6,'(T5,A/)')'                                              '
      write(6,200) a, a,b
      write(6,'(T5,A/)')'                                              '
       WRITE(*,'(3(T5,A/),3(T5,A/T5,(3(A,1PE13.6))/))')
     &'     Error    \UNICODE{0xb3}    Error      \UNICODE{0xb3}  Error        \UNICODE{0xb3}              ',
     &'     En %     \UNICODE{0xb3}    En %       \UNICODE{0xb3}  En %         \UNICODE{0xb3}    N\UNICODE{0xa7} de paso'
200   format(t5,f8.5,10x,f8.5,10x,f8.5)
       write(15,'(T5,A/)')'Resultado analitico'
      write(15,'(T5,A/)')'                                             '
      write(15,205) a, a,b
      write(15,'(T5,A/)')'                                             '
       WRITE(15,'(3(T5,A/),3(T5,A/T5,(3(A,1PE13.6))/))')
     &'     Error    \UNICODE{0xb3}    Error      \UNICODE{0xb3}  Error        \UNICODE{0xb3}              ',
     &'     En %     \UNICODE{0xb3}    En %       \UNICODE{0xb3}  En %         \UNICODE{0xb3}    N\UNICODE{0xa7} de paso'
205   format(t5,f8.5,10x,f8.5,10x,f8.5)
 
      do 30 j=1,5
      errord(j)=abs(((A-difd(j))/A)*100)
      errort(j)=abs(((A-dift(j))/A)*100)
 
      error2(j)=abs(((B-dif2(j))/B)*100)
       write(6,300) errord(j),errort(j),error2(j),dx(j)
300   format(t5,f8.5,10x,f8.5,10x,f8.5,10x,f8.5)
      write(15,305) errord(j),errort(j),error2(j),dx(j)
305   format(t5,f8.7,10x,f8.7,10x,f8.7,10x,f8.5)
 
30      continue
 
 
 
      close(15)
      do 50 k=1,5
 
      open(20,file='derivadad2.dat',status='unknown')
      open(40,file='derivada2c2.dat',status='unknown')
      open(60,file='derivadat2.dat',status='unknown')
      write(20,*) dx(k),errord(k)
      write(40,*) dx(k),error2(k)
      write(60,*) dx(k),errort(k)
50    continue
      close(20)
      close(40)
      close(60)
      end
C***********************************************************************
C           Definimos la funcion a evaluar:
C***********************************************************************
      function fun(X)
      real X
      fun = (x**3)-5*x
      end
C***********************************************************************
C     Primera derivada
      function fun1(X)
      real X
      fun1 = 3*(x**2)-5
      end
C***********************************************************************
C     Segunda derivada
      function fun2(X)
      real X
      fun2 = 6*x
      end
\end{verbatim}

\bigskip

El cual tiene la siguiente salida

Para el valor de $x=0.5$
\begin{verbatim}
 \UNICODE{0xc5}*******************\UNICODE{0xc5}********************\UNICODE{0xc5}********************\UNICODE{0xc5}
 
     calcule la 1\UNICODE{0xa7} y 2\UNICODE{0xa6} derivada de f(x) = (x**3)-5*x
 
    para los valores de dx 0.0005,0.001,0.01,0.1 y 0.2
 
    \UNICODE{0xc5}****************\UNICODE{0xc5}*****************\UNICODE{0xc5}**************************\UNICODE{0xc5}
 
    ***************************************************************
 
                                                    
                                                    
                                                   
     1\UNICODE{0xa7} Derivada  \UNICODE{0xb3} 1\UNICODE{0xa7} Derivada   \UNICODE{0xb3}  2\UNICODE{0xa6} Derivada  \UNICODE{0xb3}             
      diferencia  \UNICODE{0xb3}  diferencia   \UNICODE{0xb3}   diferencia  \UNICODE{0xb3}    N\UNICODE{0xa7} de paso
        hacia     \UNICODE{0xb3}    hacia      \UNICODE{0xb3}    centrada   \UNICODE{0xb3}             
      adelante    \UNICODE{0xb3}    atras      \UNICODE{0xb3}               \UNICODE{0xb3}             
    \UNICODE{0xc4}\UNICODE{0xc4}\UNICODE{0xc4}\UNICODE{0xc4}\UNICODE{0xc4}\UNICODE{0xc4}\UNICODE{0xc4}\UNICODE{0xc4}\UNICODE{0xc4}\UNICODE{0xc4}\UNICODE{0xc4}\UNICODE{0xc4}\UNICODE{0xc4}\UNICODE{0xc4}\UNICODE{0xc5}\UNICODE{0xc4}\UNICODE{0xc4}\UNICODE{0xc4}\UNICODE{0xc4}\UNICODE{0xc4}\UNICODE{0xc4}\UNICODE{0xc4}\UNICODE{0xc4}\UNICODE{0xc4}\UNICODE{0xc4}\UNICODE{0xc4}\UNICODE{0xc4}\UNICODE{0xc4}\UNICODE{0xc4}\UNICODE{0xc4}\UNICODE{0xc5}\UNICODE{0xc4}\UNICODE{0xc4}\UNICODE{0xc4}\UNICODE{0xc4}\UNICODE{0xc4}\UNICODE{0xc4}\UNICODE{0xc4}\UNICODE{0xc4}\UNICODE{0xc4}\UNICODE{0xc4}\UNICODE{0xc4}\UNICODE{0xc4}\UNICODE{0xc4}\UNICODE{0xc4}\UNICODE{0xc4}\UNICODE{0xc5}\UNICODE{0xc4}\UNICODE{0xc4}\UNICODE{0xc4}\UNICODE{0xc4}\UNICODE{0xc4}\UNICODE{0xc4}\UNICODE{0xc4}\UNICODE{0xc4}\UNICODE{0xc4}\UNICODE{0xc4}\UNICODE{0xc4}\UNICODE{0xc4}\UNICODE{0xc4}\UNICODE{0xc4}
    -4.24385          -4.26769          ********           0.00001
    -4.25100          -4.24862          ********           0.00010
    -4.24838          -4.25148           3.09944           0.00100
    -4.09000          -4.39000           2.99999           0.10000
    -3.25000          -4.75000           3.00000           0.50000
    Resultado analitico
 
                                                
 
    -4.25000          -4.25000           3.00000
                                                
 
         Error    \UNICODE{0xb3}    Error      \UNICODE{0xb3}  Error        \UNICODE{0xb3}             
         En %     \UNICODE{0xb3}    En %       \UNICODE{0xb3}  En %         \UNICODE{0xb3}    N\UNICODE{0xa7} de paso
 
    .1446892          .4162957          ********           0.00001
    .0236062          .0324922          ********           0.00010
    .0381021          .0348259          ********           0.00100
    ********          ********          .0004927           0.10000
    ********          ********          .0000000           0.50000
\end{verbatim}

Para el valor de $x=1.5$
\begin{verbatim}
  \UNICODE{0xc5}*******************\UNICODE{0xc5}********************\UNICODE{0xc5}********************\UNICODE{0xc5}
 
     calcule la 1\UNICODE{0xa7} y 2\UNICODE{0xa6} derivada de f(x) = (x**3)-5*x
 
    para los valores de dx 0.0005,0.001,0.01,0.1 y 0.2
 
    \UNICODE{0xc5}****************\UNICODE{0xc5}*****************\UNICODE{0xc5}**************************\UNICODE{0xc5}
 
    ***************************************************************
 
                                                    
                                                    
                                                   
     1\UNICODE{0xa7} Derivada  \UNICODE{0xb3} 1\UNICODE{0xa7} Derivada   \UNICODE{0xb3}  2\UNICODE{0xa6} Derivada  \UNICODE{0xb3}             
      diferencia  \UNICODE{0xb3}  diferencia   \UNICODE{0xb3}   diferencia  \UNICODE{0xb3}    N\UNICODE{0xa7} de paso
        hacia     \UNICODE{0xb3}    hacia      \UNICODE{0xb3}    centrada   \UNICODE{0xb3}             
      adelante    \UNICODE{0xb3}    atras      \UNICODE{0xb3}               \UNICODE{0xb3}             
    \UNICODE{0xc4}\UNICODE{0xc4}\UNICODE{0xc4}\UNICODE{0xc4}\UNICODE{0xc4}\UNICODE{0xc4}\UNICODE{0xc4}\UNICODE{0xc4}\UNICODE{0xc4}\UNICODE{0xc4}\UNICODE{0xc4}\UNICODE{0xc4}\UNICODE{0xc4}\UNICODE{0xc4}\UNICODE{0xc5}\UNICODE{0xc4}\UNICODE{0xc4}\UNICODE{0xc4}\UNICODE{0xc4}\UNICODE{0xc4}\UNICODE{0xc4}\UNICODE{0xc4}\UNICODE{0xc4}\UNICODE{0xc4}\UNICODE{0xc4}\UNICODE{0xc4}\UNICODE{0xc4}\UNICODE{0xc4}\UNICODE{0xc4}\UNICODE{0xc4}\UNICODE{0xc5}\UNICODE{0xc4}\UNICODE{0xc4}\UNICODE{0xc4}\UNICODE{0xc4}\UNICODE{0xc4}\UNICODE{0xc4}\UNICODE{0xc4}\UNICODE{0xc4}\UNICODE{0xc4}\UNICODE{0xc4}\UNICODE{0xc4}\UNICODE{0xc4}\UNICODE{0xc4}\UNICODE{0xc4}\UNICODE{0xc4}\UNICODE{0xc5}\UNICODE{0xc4}\UNICODE{0xc4}\UNICODE{0xc4}\UNICODE{0xc4}\UNICODE{0xc4}\UNICODE{0xc4}\UNICODE{0xc4}\UNICODE{0xc4}\UNICODE{0xc4}\UNICODE{0xc4}\UNICODE{0xc4}\UNICODE{0xc4}\UNICODE{0xc4}\UNICODE{0xc4}
     1.76430           1.76430           0.00000           0.00001
     1.74999           1.74999           0.00000           0.00010
     1.75476           1.74570           9.05991           0.00100
     2.21000           1.31000           8.99999           0.10000
     4.25000          -0.25000           9.00000           0.50000
    Resultado analitico
 
                                                
 
     1.75000           1.75000           9.00000
                                                
 
         Error    \UNICODE{0xb3}    Error      \UNICODE{0xb3}  Error        \UNICODE{0xb3}             
         En %     \UNICODE{0xb3}    En %       \UNICODE{0xb3}  En %         \UNICODE{0xb3}    N\UNICODE{0xa7} de paso
 
    .8169991          .8169991          ********           0.00001
    .0004360          .0004360          ********           0.00010
    .2720356          .2456733          .6656117           0.00100
    ********          ********          .0000954           0.10000
    ********          ********          .0000000           0.50000
\end{verbatim}

Las gr\'{a}ficas obtenidas con los datos obtenidos son:

Para  $X=0.5$

Diferencia hacia adelante\FRAME{dhF}{3.8657in}{3.0303in}{0pt}{}{}{%
derivadad.eps}{\special{language "Scientific Word";type
"GRAPHIC";maintain-aspect-ratio TRUE;display "USEDEF";valid_file "F";width
3.8657in;height 3.0303in;depth 0pt;original-width 4.0395in;original-height
3.1609in;cropleft "0";croptop "1";cropright "1";cropbottom "0";filename
'derivadad.EPS';file-properties "XNPEU";}}

Derivada hacia atr\'{a}s\FRAME{fhF}{3.8545in}{3.2154in}{0pt}{}{}{%
derivadat.eps}{\special{language "Scientific Word";type
"GRAPHIC";maintain-aspect-ratio TRUE;display "USEDEF";valid_file "F";width
3.8545in;height 3.2154in;depth 0pt;original-width 4.0274in;original-height
3.3555in;cropleft "0";croptop "1";cropright "1";cropbottom "0";filename
'derivadat.EPS';file-properties "XNPEU";}}

Derivada centrada\FRAME{dtbpF}{3.9064in}{3.282in}{0pt}{}{}{derivada2.eps}{%
\special{language "Scientific Word";type "GRAPHIC";maintain-aspect-ratio
TRUE;display "USEDEF";valid_file "F";width 3.9064in;height 3.282in;depth
0pt;original-width 4.0828in;original-height 3.4264in;cropleft "0";croptop
"1";cropright "1";cropbottom "0";filename 'derivada2.EPS';file-properties
"XNPEU";}}

Para $x=1.5$

\FRAME{dhF}{3.6642in}{3.1479in}{0pt}{}{}{derivadad2.eps}{\special{language
"Scientific Word";type "GRAPHIC";maintain-aspect-ratio TRUE;display
"USEDEF";valid_file "F";width 3.6642in;height 3.1479in;depth
0pt;original-width 4.0395in;original-height 3.4662in;cropleft "0";croptop
"1";cropright "1";cropbottom "0";filename 'derivadad2.EPS';file-properties
"XNPEU";}}

Segunda derivada hacia adelante

\FRAME{dhF}{3.9989in}{3.1903in}{0pt}{}{}{derivada2c2.eps}{\special{language
"Scientific Word";type "GRAPHIC";maintain-aspect-ratio TRUE;display
"USEDEF";valid_file "F";width 3.9989in;height 3.1903in;depth
0pt;original-width 4.1805in;original-height 3.3278in;cropleft "0";croptop
"1";cropright "1";cropbottom "0";filename 'derivada2c2.eps';file-properties
"XNPEU";}}

derivada hacia atr\'{a}s

\FRAME{dhF}{4.0897in}{3.4169in}{0pt}{}{}{derivadat2.eps}{\special{language
"Scientific Word";type "GRAPHIC";maintain-aspect-ratio TRUE;display
"USEDEF";valid_file "F";width 4.0897in;height 3.4169in;depth
0pt;original-width 4.0395in;original-height 3.371in;cropleft "0";croptop
"1";cropright "1";cropbottom "0";filename 'derivadat2.EPS';file-properties
"XNPEU";}}

\FRAME{dhF}{4.2307in}{3.3745in}{0pt}{}{}{derivada2c2.eps}{\special{language
"Scientific Word";type "GRAPHIC";maintain-aspect-ratio TRUE;display
"USEDEF";valid_file "F";width 4.2307in;height 3.3745in;depth
0pt;original-width 4.1805in;original-height 3.3278in;cropleft "0";croptop
"1";cropright "1";cropbottom "0";filename 'derivada2c2.eps';file-properties
"XNPEU";}}

\end{document}
