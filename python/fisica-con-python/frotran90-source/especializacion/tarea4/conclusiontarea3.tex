%% This document created by Scientific Word (R) Version 3.0

\documentclass{article}
\usepackage{graphicx}
\usepackage{amsmath}
\usepackage{amsfonts}
\usepackage{amssymb}
%TCIDATA{OutputFilter=latex2.dll}
%TCIDATA{CSTFile=LaTeX article (bright).cst}
%TCIDATA{Created=Fri Nov 14 16:01:25 2003}
%TCIDATA{LastRevised=Fri Nov 14 16:49:31 2003}
%TCIDATA{<META NAME="GraphicsSave" CONTENT="32">}
%TCIDATA{<META NAME="DocumentShell" CONTENT="General\Blank Document">}
\newtheorem{theorem}{Theorem}
\newtheorem{acknowledgement}[theorem]{Acknowledgement}
\newtheorem{algorithm}[theorem]{Algorithm}
\newtheorem{axiom}[theorem]{Axiom}
\newtheorem{case}[theorem]{Case}
\newtheorem{claim}[theorem]{Claim}
\newtheorem{conclusion}[theorem]{Conclusion}
\newtheorem{condition}[theorem]{Condition}
\newtheorem{conjecture}[theorem]{Conjecture}
\newtheorem{corollary}[theorem]{Corollary}
\newtheorem{criterion}[theorem]{Criterion}
\newtheorem{definition}[theorem]{Definition}
\newtheorem{example}[theorem]{Example}
\newtheorem{exercise}[theorem]{Exercise}
\newtheorem{lemma}[theorem]{Lemma}
\newtheorem{notation}[theorem]{Notation}
\newtheorem{problem}[theorem]{Problem}
\newtheorem{proposition}[theorem]{Proposition}
\newtheorem{remark}[theorem]{Remark}
\newtheorem{solution}[theorem]{Solution}
\newtheorem{summary}[theorem]{Summary}
\newenvironment{proof}[1][Proof]{\textbf{#1.} }{\ \rule{0.5em}{0.5em}}

\begin{document}

\begin{center}
FISICA MATEMATICA Y COMPUTACIONAL\bigskip

ASIGNACION N$%
%TCIMACRO{\UNICODE{0xb0}}%
%BeginExpansion
{{}^\circ}%
%EndExpansion
3$

\newline CONCLUSIONES
\end{center}

\newline Problema 1:

Al elaborar las gr\'{a}ficas de X vs Y para cada $\theta,$ se obtienen
graficas sim\'{e}tricas,  se puede ver que para \'{a}ngulos complementarios,
el alcance es el mismo;el alcance m\'{a}ximo se da para un \'{a}ngulo de 45$%
%TCIMACRO{\UNICODE{0xb0}}%
%BeginExpansion
{{}^\circ}%
%EndExpansion
.$ La altura m\'{a}xima se obtiene para un \'{a}ngulo de 60$%
%TCIMACRO{\UNICODE{0xb0}}%
%BeginExpansion
{{}^\circ}%
%EndExpansion
,$todo lo anterior esta de acuerdo con la teor\'{i}a.

Los errores para los calculos num\'{e}ricos, del alcance, la altura m\'{a}xima
y el tiempo de vuelo para un \'{a}ngulo de 45$%
%TCIMACRO{\UNICODE{0xb0}}%
%BeginExpansion
{{}^\circ}%
%EndExpansion
$ son peque\~{n}os (menores del 5\%), lo cual indica que el m\'{e}todo usado
funciona bien a un \ nivel de confianza menor del 5\%.

Problema 2:

Cuando se tiene en cuenta la fuerza de arrastre o fricci\'{o}n ( drag force),
la gr\'{a}fica deja de ser sim\'{e}trica ,tiene su m\'{a}ximo corrido a la
izquierda, aproximadamente a 2000m, luego se amortigua rapidamente. El
m\'{a}ximo alcance se da para 35$%
%TCIMACRO{\UNICODE{0xb0}}%
%BeginExpansion
{{}^\circ}%
%EndExpansion
$ y el m\'{i}nimo para 60$%
%TCIMACRO{\UNICODE{0xb0}}%
%BeginExpansion
{{}^\circ}%
%EndExpansion
,$ de 35$%
%TCIMACRO{\UNICODE{0xb0}}%
%BeginExpansion
{{}^\circ}%
%EndExpansion
$ en adelante, el alcance empieza a disminuir. La altura m\'{a}xima se da para
un \'{a}ngulode 60$%
%TCIMACRO{\UNICODE{0xb0}}%
%BeginExpansion
{{}^\circ}%
%EndExpansion
,$ que es el del m\'{i}nimo alcance.

\newline Problema 3:

Comparado con el problema 2, el incluir la variaci\'{o}n de la densidad con la
altura no cambia mucho los resultados, las graficas coinciden en una gran
cantidad de puntos.

\newline Problema 4:

Al comparar las gr\'{a}ficas de los tres ejercicios anteriores para $\theta=45%
%TCIMACRO{\UNICODE{0xb0}}%
%BeginExpansion
{{}^\circ}%
%EndExpansion
$ se puede concluir que \ la fuerza de arrastre afecta mucho el comportamiento
del proyectil, su altura m\'{a}xima es casi seis veces menor, y su alcance
disminuye a menos de la mitad. La diferencia \ entre la gr\'{a}fica del
problema 3 y la grafica del problema 2 es muy poca, sin embargo se nota que el
proyectil tiene un mayor alcance cuando no se considera la variaci\'{o}n de la
densidad con la altura, pero tiene una altura m\'{a}xima menor que cuando se considera.

\newline Problema 5:

Al tomar un paso grande (0.04) para el problema del p\'{e}ndulo simple, se
obtiene una grafica que aumenta su amplitud al transcurrir el tiempo, lo que
indica un mal funcionamiento del m\'{e}todo empleado en este caso. al
disminuir el paso \ (0,0005), el resultado se ajusta a la teor\'{i}a . De este
modo se puede concluir que el m\'{e}todo usado mejora \ para pasos
peque\~{n}os. El diagrama de fase esta de acuerdo con la teor\'{i}a, ya que es
una curva cerrada simetrica(una elipse), con un punto de equilibrio estable.

\bigskip
\end{document}