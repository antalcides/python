%% This document created by Scientific Word (R) Version 3.0

\documentclass{article}
\usepackage{graphicx}
\usepackage{amsmath}
\usepackage{amsfonts}
\usepackage{amssymb}
%TCIDATA{OutputFilter=latex2.dll}
%TCIDATA{CSTFile=LaTeX article (bright).cst}
%TCIDATA{Created=Fri Nov 14 16:01:25 2003}
%TCIDATA{LastRevised=Fri Nov 14 17:18:53 2003}
%TCIDATA{<META NAME="GraphicsSave" CONTENT="32">}
%TCIDATA{<META NAME="DocumentShell" CONTENT="General\Blank Document">}
\newtheorem{theorem}{Theorem}
\newtheorem{acknowledgement}[theorem]{Acknowledgement}
\newtheorem{algorithm}[theorem]{Algorithm}
\newtheorem{axiom}[theorem]{Axiom}
\newtheorem{case}[theorem]{Case}
\newtheorem{claim}[theorem]{Claim}
\newtheorem{conclusion}[theorem]{Conclusion}
\newtheorem{condition}[theorem]{Condition}
\newtheorem{conjecture}[theorem]{Conjecture}
\newtheorem{corollary}[theorem]{Corollary}
\newtheorem{criterion}[theorem]{Criterion}
\newtheorem{definition}[theorem]{Definition}
\newtheorem{example}[theorem]{Example}
\newtheorem{exercise}[theorem]{Exercise}
\newtheorem{lemma}[theorem]{Lemma}
\newtheorem{notation}[theorem]{Notation}
\newtheorem{problem}[theorem]{Problem}
\newtheorem{proposition}[theorem]{Proposition}
\newtheorem{remark}[theorem]{Remark}
\newtheorem{solution}[theorem]{Solution}
\newtheorem{summary}[theorem]{Summary}
\newenvironment{proof}[1][Proof]{\textbf{#1.} }{\ \rule{0.5em}{0.5em}}

\begin{document}

\begin{center}
FISICA MATEMATICA Y COMPUTACIONAL\bigskip

ASIGNACION N$%
%TCIMACRO{\UNICODE{0xb0}}%
%BeginExpansion
{{}^\circ}%
%EndExpansion
3$

\newline CONCLUSIONES
\end{center}

\newline Problema 1:

Al elaborar las gr\'{a}ficas de X vs Y para cada $\theta,$ se obtienen
graficas sim\'{e}tricas,  se puede ver que para \'{a}ngulos complementarios,
el alcance es el mismo;el alcance m\'{a}ximo se da para un \'{a}ngulo de 45$%
%TCIMACRO{\UNICODE{0xb0}}%
%BeginExpansion
{{}^\circ}%
%EndExpansion
.$ La altura m\'{a}xima se obtiene para un \'{a}ngulo de 60$%
%TCIMACRO{\UNICODE{0xb0}}%
%BeginExpansion
{{}^\circ}%
%EndExpansion
,$todo lo anterior esta de acuerdo con la teor\'{i}a.

Los errores para los calculos num\'{e}ricos, del alcance, la altura m\'{a}xima
y el tiempo de vuelo para un \'{a}ngulo de 45$%
%TCIMACRO{\UNICODE{0xb0}}%
%BeginExpansion
{{}^\circ}%
%EndExpansion
$ son peque\~{n}os (menores del 5\%), lo cual indica que el m\'{e}todo usado
funciona bien a un \ nivel de confianza menor del 5\%.%

%TCIMACRO{\FRAME{dhF}{4.1286in}{2.9845in}{0pt}{}{}{tirolibre.eps}%
%{\special{ language "Scientific Word";  type "GRAPHIC";
%maintain-aspect-ratio TRUE;  display "USEDEF";  valid_file "F";
%width 4.1286in;  height 2.9845in;  depth 0pt;  original-width 4.5567in;
%original-height 3.2854in;  cropleft "0";  croptop "1";  cropright "1";
%cropbottom "0";  filename 'tirolibre.eps';file-properties "XNPEU";}}}%
%BeginExpansion
\begin{center}
\includegraphics[
height=2.9845in,
width=4.1286in
]%
{tirolibre.eps}%
\end{center}
%EndExpansion

Problema 2:

Cuando se tiene en cuenta la fuerza de arrastre o fricci\'{o}n ( drag force),
la gr\'{a}fica deja de ser sim\'{e}trica ,tiene su m\'{a}ximo corrido a la
izquierda, aproximadamente a 2000m, luego se amortigua rapidamente. El
m\'{a}ximo alcance se da para 35$%
%TCIMACRO{\UNICODE{0xb0}}%
%BeginExpansion
{{}^\circ}%
%EndExpansion
$ y el m\'{i}nimo para 60$%
%TCIMACRO{\UNICODE{0xb0}}%
%BeginExpansion
{{}^\circ}%
%EndExpansion
,$ de 35$%
%TCIMACRO{\UNICODE{0xb0}}%
%BeginExpansion
{{}^\circ}%
%EndExpansion
$ en adelante, el alcance empieza a disminuir. La altura m\'{a}xima se da para
un \'{a}ngulode 60$%
%TCIMACRO{\UNICODE{0xb0}}%
%BeginExpansion
{{}^\circ}%
%EndExpansion
,$ que es el del m\'{i}nimo alcance.%
%TCIMACRO{\FRAME{dhF}{3.7697in}{2.9162in}{0pt}{}{}{tironolibrej.eps}%
%{\special{ language "Scientific Word";  type "GRAPHIC";
%maintain-aspect-ratio TRUE;  display "USEDEF";  valid_file "F";
%width 3.7697in;  height 2.9162in;  depth 0pt;  original-width 4.4036in;
%original-height 3.3987in;  cropleft "0";  croptop "1";  cropright "1";
%cropbottom "0";  filename 'tironolibrej.eps';file-properties "XNPEU";}}}%
%BeginExpansion
\begin{center}
\includegraphics[
height=2.9162in,
width=3.7697in
]%
{tironolibrej.eps}%
\end{center}
%EndExpansion

\newline Problema 3:

Comparado con el problema 2, el incluir la variaci\'{o}n de la densidad con la
altura no cambia mucho los resultados, las graficas coinciden en una gran
cantidad de puntos.%
%TCIMACRO{\FRAME{dhF}{3.5336in}{2.9741in}{0pt}{}{}{tironlda.eps}%
%{\special{ language "Scientific Word";  type "GRAPHIC";
%maintain-aspect-ratio TRUE;  display "USEDEF";  valid_file "F";
%width 3.5336in;  height 2.9741in;  depth 0pt;  original-width 4.1252in;
%original-height 3.4662in;  cropleft "0";  croptop "1";  cropright "1";
%cropbottom "0";  filename 'tironlda.EPS';file-properties "XNPEU";}}}%
%BeginExpansion
\begin{center}
\includegraphics[
natheight=3.466200in,
natwidth=4.125200in,
height=2.9741in,
width=3.5336in
]%
{tironlda.EPS}%
\end{center}
%EndExpansion

\newline Problema 4:

Al comparar las gr\'{a}ficas de los tres ejercicios anteriores para $\theta=45%
%TCIMACRO{\UNICODE{0xb0}}%
%BeginExpansion
{{}^\circ}%
%EndExpansion
$ se puede concluir que \ la fuerza de arrastre afecta mucho el comportamiento
del proyectil, su altura m\'{a}xima es casi seis veces menor, y su alcance
disminuye a menos de la mitad. La diferencia \ entre la gr\'{a}fica del
problema 3 y la grafica del problema 2 es muy poca, sin embargo se nota que el
proyectil tiene un mayor alcance cuando no se considera la variaci\'{o}n de la
densidad con la altura, pero tiene una altura m\'{a}xima menor que cuando se
considera.%
%TCIMACRO{\FRAME{dhF}{3.5682in}{2.9871in}{0pt}{}{}{tironolibre.eps}%
%{\special{ language "Scientific Word";  type "GRAPHIC";
%maintain-aspect-ratio TRUE;  display "USEDEF";  valid_file "F";
%width 3.5682in;  height 2.9871in;  depth 0pt;  original-width 4.1658in;
%original-height 3.4809in;  cropleft "0";  croptop "1";  cropright "1";
%cropbottom "0";  filename 'tironolibre.eps';file-properties "XNPEU";}}}%
%BeginExpansion
\begin{center}
\includegraphics[
height=2.9871in,
width=3.5682in
]%
{tironolibre.eps}%
\end{center}
%EndExpansion%
%TCIMACRO{\FRAME{dtbpF}{3.1592in}{2.3194in}{0pt}{}{}{proyectil45.eps}%
%{\special{ language "Scientific Word";  type "GRAPHIC";
%maintain-aspect-ratio TRUE;  display "USEDEF";  valid_file "F";
%width 3.1592in;  height 2.3194in;  depth 0pt;  original-width 4.529in;
%original-height 3.3131in;  cropleft "0";  croptop "1";  cropright "1";
%cropbottom "0";  filename 'proyectil45.EPS';file-properties "XNPEU";}}}%
%BeginExpansion
\begin{center}
\includegraphics[
natheight=3.313100in,
natwidth=4.529000in,
height=2.3194in,
width=3.1592in
]%
{proyectil45.EPS}%
\end{center}
%EndExpansion

\newline Problema 5:

Al tomar un paso grande (0.04) para el problema del p\'{e}ndulo simple, se
obtiene una grafica que aumenta su amplitud al transcurrir el tiempo, lo que
indica un mal funcionamiento del m\'{e}todo empleado en este caso. al
disminuir el paso \ (0,0005), el resultado se ajusta a la teor\'{i}a . De este
modo se puede concluir que el m\'{e}todo usado mejora \ para pasos
peque\~{n}os. El diagrama de fase esta de acuerdo con la teor\'{i}a, ya que es
una curva cerrada simetrica(una elipse), con un punto de equilibrio estable.%

%TCIMACRO{\FRAME{ftbpF}{2.6481in}{2.3212in}{0pt}{}{}{pendulop.eps}%
%{\special{ language "Scientific Word";  type "GRAPHIC";
%maintain-aspect-ratio TRUE;  display "USEDEF";  valid_file "F";
%width 2.6481in;  height 2.3212in;  depth 0pt;  original-width 3.9297in;
%original-height 3.4385in;  cropleft "0";  croptop "1";  cropright "1";
%cropbottom "0";  filename 'pendulop.eps';file-properties "XNPEU";}}}%
%BeginExpansion
\begin{figure}
[ptb]
\begin{center}
\includegraphics[
height=2.3212in,
width=2.6481in
]%
{pendulop.eps}%
\end{center}
\end{figure}
%EndExpansion%
%TCIMACRO{\FRAME{ftbpF}{2.9205in}{2.3194in}{0pt}{}{}{pendulopn.eps}%
%{\special{ language "Scientific Word";  type "GRAPHIC";
%maintain-aspect-ratio TRUE;  display "USEDEF";  valid_file "F";
%width 2.9205in;  height 2.3194in;  depth 0pt;  original-width 4.3059in;
%original-height 3.4117in;  cropleft "0";  croptop "1";  cropright "1";
%cropbottom "0";  filename 'pendulopn.EPS';file-properties "XNPEU";}}}%
%BeginExpansion
\begin{figure}
[ptbptb]
\begin{center}
\includegraphics[
natheight=3.411700in,
natwidth=4.305900in,
height=2.3194in,
width=2.9205in
]%
{pendulopn.EPS}%
\end{center}
\end{figure}
%EndExpansion%
%TCIMACRO{\FRAME{ftbpF}{2.8686in}{2.3203in}{0pt}{}{}{pendulova.eps}%
%{\special{ language "Scientific Word";  type "GRAPHIC";
%maintain-aspect-ratio TRUE;  display "USEDEF";  valid_file "F";
%width 2.8686in;  height 2.3203in;  depth 0pt;  original-width 3.9851in;
%original-height 3.2154in;  cropleft "0";  croptop "1";  cropright "1";
%cropbottom "0";  filename 'pendulova.eps';file-properties "XNPEU";}}}%
%BeginExpansion
\begin{figure}
[ptbptbptb]
\begin{center}
\includegraphics[
height=2.3203in,
width=2.8686in
]%
{pendulova.eps}%
\end{center}
\end{figure}
%EndExpansion

\bigskip
\end{document}