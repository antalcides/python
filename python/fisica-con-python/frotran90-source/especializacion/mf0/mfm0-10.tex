\chapter[EDO: M{\'e}todos b{\'a}sicos.]{Ecuaciones diferenciales ordinarias: M{\'e}todos b{\'a}sicos.}

\vspace{-1cm}
\hfill {\tiny versi{\'o}n preliminar 2.2-1 julio 2002\footnote{Este cap{\'\i}tulo est{\'a} basado en
    el segundo cap{\'\i}tulo del libro: {\em Numerical Methods for Physics,
      second edition} de Alejandro L. Garcia, editorial {\sc Prentice Hall}.} }

En este cap{\'\i}tulo resolveremos uno de los primeros problemas
considerados por un estudiante de f{\'\i}sica: el vuelo de un proyectil y,
en particular, el de una pelota de {\em baseball}. Sin la resistencia
del aire el problema es f{\'a}cil de resolver. Sin embargo, incluyendo un
arrastre realista, nosotros necesitamos calcular la soluci{\'o}n
num{\'e}ricamente. Para analizar este problema definiremos primero la
diferenciaci{\'o}n num{\'e}rica. De hecho antes de aprender F{\'\i}sica uno aprende
c{\'a}lculo as{\'\i} que no debemos sorprendernos si este es nuestro punto de
partida. En la segunda mitad del cap{\'\i}tulo nos acuparemos de otro viejo
conocido, el p{\'e}ndulo simple, pero sin la aproximaci{\'o}n a {\'a}ngulos
peque{\~n}os. Interesantemente, problemas oscilatorios, tales como el
p{\'e}ndulo, revelan una falla fatal en algunos de los m{\'e}todos num{\'e}ricos
para resolver ecuaciones diferenciales ordinarias.


\section{Movimiento de un proyectil.}
\label{c10-s1}

\subsection{Ecuaciones b{\'a}sicas.}

Consideremos el simple movimiento de un proyectil, digamos un pelota
de {\em baseball}. Para describir el movimiento nosotros debemos
calcular el vector posici{\'o}n $\vec r(t)$ y el vector velocidad
$\vec v(t)$ del proyectil. Las ecuaciones b{\'a}sicas de movimiento son
\begin{equation}
\label{c10-e2.1}
\frac{d\vec v}{dt}= \frac{1}{m} \vec F_a(\vec v\,) - g\hat y\ , \quad
\frac{d\vec r}{dt}= \vec v\ ,
\end{equation}
donde $m$ es la masa del proyectil. La fuerza debido a la resistencia
del aire es $\vec F_a(\vec v\,)$, la aceleraci{\'o}n gravitacional es $g$, e
$\hat y$ es un vector unitario en la direcci{\'o}n $y$. El movimiento es
bidimensional, tal que podemos ignorar la componente $z$ y trabajar en
el plano $xy$.

La resistencia del aire se incrementa con la velocidad del objeto, y
la forma precisa para $\vec F_a(\vec v\,)$ depende del flujo alrededor
del proyectil. Com{\'u}nmente, esta fuerza es aproximada por
\begin{equation}
\label{c10-e2.2}
\vec F_a(\vec v)= -\frac 12 C_d \rho A \modulo{v} \vec v\ ,
\end{equation}
donde $C_d$ es el coeficiente de arrastre, $\rho$ es la densidad del
aire, y $A$ es el {\'a}rea de la secci{\'o}n transversal del proyectil. El
coeficiente de arrastre es un par{\'a}metro adimensional que depende de la
geometr{\'\i}a del proyectil ---Mientras m{\'a}s aerodin{\'a}mico el objeto, el
coeficiente es menor.

Para una esfera suave de radio $R$ movi{\'e}ndose lentamente a trav{\'e}s del
fluido, el coeficiente de arrastre es dado por la Ley de Stokes,
\begin{equation}
\label{c10-e2.3}
C_d=\frac{12\nu}{Rv}=\frac{24}{\text{Re}}\ ,
\end{equation}
donde $\nu$ es la viscosidad del fluido ($\nu \approx 1.5\times10^{-5}$ [m$^2$/s]
para el aire) y $\text{Re}=2Rv/\nu$ es el adimensional {\em n{\'u}mero de
  Reynolds}. Para un objeto del tama{\~n}o de un pelota de baseball
movi{\'e}ndose a trav{\'e}s del aire, la ley de Stokes es v{\'a}lida s{\'o}lo si la
velocidad es menor que 0.2~[mm/s] ($\text{Re}\approx1$).

A velocidades altas (sobre 20~[cm/s], $\text{Re}>10^3$), la estela detr{\'a}s de
la esfera desarrolla v{\'o}rtices y el coeficiente de arrastre es
aproximadamente constante ($C_d\approx0.5$) para un amplio intervalo de
velocidades. Cuando el n{\'u}mero de Reynolds excede un valor cr{\'\i}tico, el
flujo en la estela llega a ser turbulento y el coeficiente de arrastre
cae dram{\'a}ticamente. Esta reducci{\'o}n ocurre porque la turbulencia rompe
la regi{\'o}n de bajas presiones en la estela detr{\'a}s de la
esfera\footnote{D.J. Tritton, {Physical Fluid Dynamics}, 2d ed.
  (Oxford: Clarendon Press, 1988).}. Para una esfera suave este n{\'u}mero
cr{\'\i}tico de Reynolds es aproximadamente $3\times10^5$. Para una pelota de
{\em baseball}, el coeficiente de arrastre es usualmente m{\'a}s peque{\~n}o
que el de una esfera suave, porque las costuras rompen el flujo
laminar precipitando el inicio de la turbulencia. Nosotros podemos
tomar $C_d=0.35$ como un valor promedio para el intervalo de
velocidades t{\'\i}picas de una pelota de {\em baseball}.

Notemos que la fuerza de arrastre, ecuaci{\'o}n (\ref{c10-e2.2}), var{\'\i}a
como el cuadrado de la magnitud de la velocidad ($\vec F_a\propto v^2$) y,
por supuesto, act{\'u}a en la direcci{\'o}n opuesta a la velocidad. La masa y
el di\'ametro de una pelota de {\em baseball} son 0.145~[kg] y 7.4~[cm].
Para una pelota de {\em baseball}, el arrastre y la fuerza
gravitacional son iguales en magnitud cuando $v\approx40$~[m/s].

Nosotros sabemos c{\'o}mo resolver las ecuaciones de movimiento si la
resistencia del aire es despreciable. La trayectoria es
\begin{equation}
\label{c10-e2.4}
\vec{r}(t)=\vec{r}_1+\vec{v}_1t- \frac 12 gt^2 \hat{y}\ ,
\end{equation}
donde $\vec{r}_1\equiv \vec{r}(0)$ y $\vec{v}_1\equiv \vec{v}(0)$ son la
posici{\'o}n y la velocidad inicial. Si el proyectil parte del origen y la
velocidad inicial forma un {\'a}ngulo $\theta$ con la horizontal, entonces
\begin{equation}
\label{c10-e2.5}
x_{\text{m{\'a}x}}=\frac{2v_1^2}{g} \sen \theta\cos\theta \ , \quad
y_{\text{m{\'a}x}} =\frac{v_1^2}{2g} \sen^2 \theta\ ,
\end{equation}
son el alcance horizontal y la altura m{\'a}xima. El tiempo de vuelo es
\begin{equation}
\label{c10-e2.6}
t_{fl}=\frac{2v_1}{g}\sen \theta\ .
\end{equation}

Nuevamente estas expresiones son v{\'a}lidas s{\'o}lo cuando no hay
resistencia con el aire. Es f{\'a}cil demostrar que el m{\'a}ximo alcance
horizontal se obtiene cuando la velocidad forma un {\'a}ngulo de
45$\grados$ con la horizontal. Deseamos mantener esta informaci{\'o}n en
mente cuando construyamos nuestra simulaci{\'o}n. Si se sabe la soluci{\'o}n
exacta para un caso especial, se debe comparar constantemente que el
programa trabaje bien para este caso.

\subsection{Derivada avanzada.}

Para resolver las ecuaciones de movimiento (\ref{c10-e2.1})
necesitamos un m{\'e}todo num{\'e}rico para evaluar la primera derivada. La
definici{\'o}n formal de la derivada es
\begin{equation}
\label{c10-e2.7}
f'(t)\equiv \lim_{\tau\to0}\frac{f(t+\tau)-f(t)}{\tau}\ ,
\end{equation}
donde $\tau$ es el incremento temporal o paso en el tiempo. Como ya vimos
en el cap{\'\i}tulo pasado esta ecuaci{\'o}n debe ser tratada con cuidado. La
figura \ref{c9-f2} ilustra que el uso de un valor extremadamente
peque{\~n}o para $\tau$ causa un gran error en el c{\'a}lculo de
$(f(t+\tau)-f(t))/\tau$. Espec{\'\i}ficamente, los errores de redondeo ocurren en
el c{\'a}lculo de $t+\tau$, en la evaluaci{\'o}n de la funci{\'o}n $f$ y en la
sustracci{\'o}n del numerador. Dado que $\tau$ no puede ser elegido
arbitrariamente peque{\~n}o, nosotros necesitamos estimar la diferencia
entre $f'(t)$ y $(f(t+\tau)-f(t))/\tau$ para un $\tau$ finito.

Para encontrar esta diferencia usaremos una expansi{\'o}n de Taylor. Como
f{\'\i}sicos usualmente vemos las series de Taylor expresadas como
\begin{equation}
\label{c10-e2.8}
f(t+\tau)=f(t)+\tau f'(t)+ \frac{\tau^2}{2}f''(t)+\cdots 
\end{equation}
donde los puntos suspensivos indican t{\'e}rminos de m{\'a}s alto orden que son
usualmente despreciados. Una alternativa, forma equivalente de la
serie de Taylor usada en an{\'a}lisis num{\'e}rico es
\begin{equation}
\label{c10-e2.9}
f(t+\tau)=f(t)+\tau f'(t)+ \frac{\tau^2}{2}f''(\zeta)\ ,
\end{equation}
donde $\zeta$ es un valor entre $t$ y $t+\tau$. No hemos botado ning{\'u}n
t{\'e}rmino, esta expansi{\'o}n tiene un n{\'u}mero finito de t{\'e}rminos. El teorema
de Taylor garantiza que existe {\em alg{\'u}n} valor $\zeta$ para el cual
(\ref{c10-e2.9}) es cierto, pero no sabemos cu{\'a}l valor es \'este.

La ecuaci{\'o}n previa puede ser reescrita
\begin{equation}
\label{c10-e2.10}
f'(t)=\frac{f(t+\tau)-f(t)}{\tau}-\frac 12 \tau f''(\zeta)\ ,
\end{equation} 
donde $t\leq \zeta\leq t +\tau$. Esta ecuaci{\'o}n es conocida como la f{\'o}rmula de la
{\em derivada derecha} o {\em derivada adelantada}. El {\'u}ltimo t{\'e}rmino
de la mano derecha es el {\em error de truncamiento}; este error es
introducido por cortar la serie de Taylor.

En otras palabras, si mantenemos el {\'u}ltimo t{\'e}rmino en
(\ref{c10-e2.10}), nuestra expresi{\'o}n para $f'(t)$ es exacta. Pero no
podemos evaluar este t{\'e}rmino porque no conocemos $\zeta$, todo lo que
conocemos es que $\zeta$ yace en alg{\'u}n lugar entre $t$ y $t+\tau$. As{\'\i}
despreciamos el t{\'e}rmino $f''(\zeta)$ (truncamos) y decimos que el error
que cometemos por despreciar este t{\'e}rmino es el error de truncamiento.
No hay que confundir {\'e}ste con el error de redondeo discutido
anteriormente. El error de redondeo depende del {\em hardware}, el
error de truncamiento depende de la aproximaci{\'o}n usada en el
algoritmo.  Algunas veces veremos la ecuaci{\'o}n (\ref{c10-e2.10})
escrita como
\begin{equation}
\label{c10-e2.11}
f'(t)=\frac{f(t+\tau)-f(t)}{\tau}+\DelOrdenD({\tau}) \ , 
\end{equation} 
donde el error de truncamiento es ahora especificado por su orden en
$\tau$, en este caso el error de truncamiento es lineal en $\tau$. En la
figura \ref{c9-f2} la fuente de error predominante en estimar $f'(x)$
como $[f(x+h)-f(x)]/h$  es el error de redondeo cuando $h<10^{-10}$ y
es el error de truncamiento cuando $h>10^{-10}$.

\subsection{M{\'e}todo de Euler.}

Las ecuaciones de movimiento que nosotros deseamos resolver
num{\'e}ricamente pueden ser escritas como:
\begin{equation}
\label{c10-e2.12}
\frac{d\vec v}{dt}=\vec{a}(\vec r, \vec v\,)\ , \quad \frac{d\vec
  r}{dt}= \vec{v}\ ,
\end{equation}
donde $\vec a$ es la aceleraci{\'o}n. Notemos que \'esta es la forma m{\'a}s
general de las ecuaciones. En el movimiento de proyectiles la
aceleraci{\'o}n es s{\'o}lo funci{\'o}n de $\vec v$ (debido al arrastre), en otros
problemas (e.g., \'orbitas de cometas) la aceleraci{\'o}n depender{\'a} de la
posici{\'o}n.

Usando la derivada adelantada (\ref{c10-e2.11}), nuestras ecuaciones
de movimiento son 
\begin{align}
\label{c10-e2.13}
\frac{\vec v (t+\tau)-\vec v (t)}{\tau} + \DelOrdenD(\tau) &=\vec a(\vec
r(t),\vec v(t)) \ ,\\
\label{c10-e2.14}
\frac{\vec r (t+\tau)-\vec r (t)}{\tau} + \DelOrdenD(\tau) &=\vec v(t) \ , 
\end{align}
o bien
\begin{align}
\label{c10-e2.15}
\vec v (t+\tau) &= \vec v (t) +\tau \vec a(\vec r(t),\vec v(t)) +
\DelOrdenD(\tau^2)\ ,\\
\label{c10-e2.16}
\vec r (t+\tau) &= \vec r (t) +\tau \vec v(t) + \DelOrdenD(\tau^2)\ .
\end{align}
Notemos que $\tau \DelOrdenD(\tau)= \DelOrdenD(\tau^2)$. Este esquema num{\'e}rico
es llamado el {\em m{\'e}todo de Euler}. Antes de discutir los m{\'e}ritos
relativos de este acercamiento, veamos c{\'o}mo ser{\'\i}a usado en la
pr{\'a}ctica.

Primero, introducimos la notaci{\'o}n
\begin{equation}
\label{c10-e2.17}
  f_n=f(t_n)\ , \quad t_n=(n-1)\tau\ , \quad n=1,2,\ldots
\end{equation}
tal que $f_1=f(t=0)$. Nuestras ecuaciones para el m{\'e}todo de Euler
(despreciando el t{\'e}rmino del error) ahora toman la forma
\begin{align}
\label{c10-e2.18}
\vec v_{n+1}  &= \vec v_n + \tau \vec a_n\ ,\\
\label{c10-e2.19}
\vec r_{n+1}  &= \vec r_n + \tau \vec v_n\ ,
\end{align}
donde $\vec a_n= \vec a(\vec r_n,\vec v_n)$. El c{\'a}lculo de la
trayectoria podr{\'\i}a proceder as{\'\i}:
\begin{enumerate}
\item Especifique las condiciones iniciales, $\vec r_1$ y $\vec v_1$.
\item Elija un paso de tiempo $\tau$.
\item Calcule la aceleraci{\'o}n dados los actuales $\vec r$ y $\vec v$.
\item Use las ecuaciones (\ref{c10-e2.18}) y (\ref{c10-e2.19}) para
  calcular los nuevos $\vec r$ y $\vec v$.
\item Vaya al paso 3 hasta que suficientes puntos de trayectoria hayan
  sido calculados.
\end{enumerate}
El m{\'e}todo calcula un conjunto de valores para $\vec r_n$ y $\vec v_n$
que nos da la trayectoria, al menos en un conjunto discreto de
valores. La figura \ref{c10-f1} ilustra el c{\'a}lculo de la trayectoria
para un {\'u}nico paso de tiempo.

\begin{figure}
\begin{center}
\includegraphics[width=7cm]{c10-f1}
\caption{Trayectoria de una part{\'\i}cula despu{\'e}s de un {\'u}nico paso de
  tiempo con el m{\'e}todo de Euler. S{\'o}lo para efectos ilustrativos $\tau$ es
  grande.}\label{c10-f1}
\end{center}
\end{figure}

\subsection{M{\'e}todos de Euler-Cromer y de Punto Medio.}

Una simple (y por ahora injustificada) modificaci{\'o}n del m{\'e}todo de
Euler es usar las siguientes ecuaciones:
\begin{align}
\label{c10-e2.20}
\vec v_{n+1}  &= \vec v_n + \tau \vec a_n\ ,\\
\label{c10-e2.21}
\vec r_{n+1}  &= \vec r_n + \tau \vec v_{n+1}\ .
\end{align}
Notemos el cambio sutil: La velocidad actualizada es usada en la
segunda ecuaci{\'o}n. Esta f{\'o}rmula es llamada {\em m{\'e}todo de
  Euler-Cromer}\footnote{A. Cromer, ``Stable solutions using the
  Euler approximation'', {\em Am. J. Phys.}, {\bf 49} 455-9 (1981).}.
El error de truncamiento es a{\'u}n del orden de $\DelOrdenD(\tau^2)$, no
parece que hemos ganado mucho. Interesantemente, veremos que esta
forma es marcadamente superior al m{\'e}todo de Euler en algunos casos. 

En el {\em m{\'e}todo del punto medio} usamos
\begin{align}
\label{c10-e2.22}
\vec v_{n+1}  &= \vec v_n + \tau \vec a_n\ ,\\
\label{c10-e2.23}
\vec r_{n+1}  &= \vec r_n + \tau \frac{\vec v_{n+1}+\vec v_n}2\ .
\end{align}
Notemos que hemos promediado las dos velocidades. Usando la ecuaci{\'o}n
para la velocidad en la ecuaci{\'o}n de la posici{\'o}n, vemos que
\begin{equation}
\label{c10-e2.24}
\vec r_{n+1} = \vec r_n + \tau \vec v_n + \frac 12  \vec a_n \tau^2\ ,
\end{equation}
lo cual realmente hace esto lucir atractivo. El error de truncamiento
es a{\'u}n del orden de $\tau^2$ en la ecuaci{\'o}n velocidad, pero para la
posici{\'o}n el error de truncamiento es ahora $\tau^3$. Realmente, para el
movimiento de proyectiles este m{\'e}todo trabaja mejor que los otros dos.
Infortunadamente, en otros sistemas f{\'\i}sicos este m{\'e}todo da resultados
pobres.

\subsection{Errores locales, errores globales y elecci{\'o}n del paso de
  tiempo.}

Para juzgar la precisi{\'o}n de estos m{\'e}todos necesitamos distinguir entre
errores de truncamiento locales y globales. Hasta ahora, el error de
truncamiento que hemos discutido ha sido el error local, el error
cometido en un {\'u}nico paso de tiempo. En un problema t{\'\i}pico nosotros
deseamos evaluar la trayectoria desde $t=0$ a $t=T$. El n{\'u}mero de
pasos de tiempo es $N_T=T/\tau$; notemos que si reducimos $\tau$, debemos
tomar m{\'a}s pasos. Si el error local es $\DelOrdenD(\tau^n)$, entonces
estimamos el error global como
\begin{equation}
\label{c10-e2.25}
\begin{split}
\text{error global}&\propto N_T \times (\text{error local})\\
&=N_T\DelOrdenD(\tau^n)=\frac
{T}{\tau}\DelOrdenD(\tau^n)=T\DelOrdenD(\tau^{n-1})\ .
\end{split}
\end{equation}
Por ejemplo, el m{\'e}todo de Euler tiene un error local de truncamiento
de $\DelOrdenD(\tau^2)$, pero un error global de truncamiento de
$\DelOrdenD(\tau)$. Por supuesto, este an{\'a}lisis nos da s{\'o}lo una
estimaci{\'o}n ya que no sabemos si los errores locales se acumular{\'a}n o se
cancelar{\'a}n ({\it i.e.} interferencia constructiva o destructiva). El
verdadero error global para un esquema num{\'e}rico es altamente
dependiente del problema que se est{\'a} estudiando.

Una pregunta que siempre aparece es ?`c{\'o}mo elegir el $\tau$? Tratemos de
responderla. Primero, supongamos que los errores de redondeo son
despreciables tal que s{\'o}lo debemos preocuparnos por los errores de
truncamiento. Desde (\ref{c10-e2.10}) y (\ref{c10-e2.16}), el error
local de truncamiento en el c{\'a}lculo de la posici{\'o}n usando el m{\'e}todo de
Euler es aproximadamente $\tau^2 r''=\tau^2a$. Usando s{\'o}lo estimaciones del
orden de magnitud, tomamos $a\approx10$~[m/s$^2$], el error en un solo paso
en la posici{\'o}n es de $10^{-1}$~[m], cuando $\tau=10^{-1}$~[s]. Si el
tiempo de vuelo $T\approx 10^0$~[s], entonces el error global es del orden
de metros. Si un error de esta magnitud es inaceptable entonces
debemos disminuir el paso en el tiempo. Finalmente usando un paso de
tiempo $10^{-1}$~[s] no introducir{\'\i}amos ning{\'u}n error significativo de
redondeo dada la magnitud de los otros par{\'a}metros del problema.

En el mundo real, a menudo no podemos hacer un an{\'a}lisis tan elegante
por una variedad de razones (ecuaciones complicadas, problemas con el
redondeo, flojera, etc.). Sin embargo, a menudo podemos usar la
intuici{\'o}n f{\'\i}sica. Resp{\'o}ndase usted mismo ``?`en qu{\'e} escala de tiempo
el movimiento es casi lineal?''. Por ejemplo, para la trayectoria
completa de una pelota de {\em baseball}, que es aproximadamente
parab{\'o}lica, el tiempo en el aire son unos pocos segundos, entonces el
movimiento es aproximadamente lineal sobre una escala de tiempo de
unos pocos cent{\'e}simos de segundo. Para revisar nuestra intuici{\'o}n,
nosotros podemos comparar los resultados obtenidos usando
$\tau=10^{-1}$~[s] y $\tau=10^{-2}$~[s] y, si ellos son suficientemente
cercanos, suponemos que todo est{\'a} bien. A veces automatizamos la
prueba de varios valores de $\tau$; el programa es entonces llamado {\em
  adaptativo} (construiremos un programa de este tipo m{\'a}s adelante).
Como con cualquier m{\'e}todo num{\'e}rico, la aplicaci{\'o}n ciega de esta
t{\'e}cnica es poco recomendada, aunque con s{\'o}lo un poco de cuidado {\'e}sta
puede ser usada exitosamente.

\begin{table}
\hrulefill
\begin{itemize}
\item Fijar la posici{\'o}n inicial $\vec r_1$ y la velocidad inicial
  $\vec v_1$ de la pelota de {\em baseball}.
\item Fijar los par{\'a}metros f{\'\i}sicos ($m$, $C_d$, etc.).
\item Iterar hasta que la bola golp\'ee en el piso o el m{\'a}ximo n{\'u}mero de
  pasos sea completado.
  \begin{itemize}
    \item Grabar posici{\'o}n (calculada y te{\'o}rica) para graficar.
    \item Calcular la aceleraci{\'o}n de la pelota de {\em baseball}.
    \item Calcular la nueva posici{\'o}n y velocidad, $\vec r_{n+1}$ y
      $\vec v_{n+1}$, Usando el m{\'e}todo de Euler, (\ref{c10-e2.18}) y
      (\ref{c10-e2.19}). 
    \item Si la pelota alcanza el suelo ($y<0$) para la iteraci{\'o}n.
\end{itemize}
\item Imprimir el alcance m{\'a}ximo y el tiempo de vuelo.
\item Graficar la trayectoria de la pelota de {\em baseball}.
\end{itemize}
\caption{Bosquejo del programa {\tt balle}, el cual calcula la trayectoria
  de una pelota de {\em baseball} usando el m{\'e}todo de Euler.}\label{c10-t1}
\hrulefill
\end{table}

\subsection{Programa de la pelota de {\em baseball}.}

La tabla \ref{c10-t1} bosqueja un simple programa, llamado
\verb|balle|, que usa el m{\'e}todo de Euler para calcular la trayectoria
de una pelota de {\em baseball}. Antes de correr el programa,
establezcamos algunos valores razonables para tomar como entradas. Una
velocidad inicial de $\modulo{\vec v_1}=$15~[m/s] nos da una pelota
que le han pegado d{\'e}bilmente. Partiendo del origen y despreciando la
resistencia del aire, el tiempo de vuelo es de 2.2~[s], y el alcance
horizontal es sobre los 23~[m] cuando el {\'a}ngulo inicial $\theta=45\grados$.
A continuaci{\'o}n, mostramos la salida a pantalla del programa
\verb|balle| en C++ cuando es corrido bajo estas condiciones.

\begin{figure}[!h]
\begin{center}
\includegraphics[angle =-90, width=10cm]{c10-f2}
\caption{Salida del programa {\tt balle} para una altura inicial de
  0~[m], una velocidad inicial de 15~[m/s], y un paso de tiempo
  $\tau=$0.1~[s]. No hay resistencia del aire. La l{\'\i}nea continua es la
  te{\'o}rica y los puntos son los calculados, la diferencia se debe a
  errores de truncamiento.}\label{c10-f2}
\end{center}
\end{figure}
\begin{verbatim}
jrogan@huelen:~/programas$ balle
Ingrese la altura inicial [m] : 0
Ingrese la velocidad inicial [m/s]: 15
Ingrese angulo inicial (grados): 45
Ingrese el paso en el tiempo, tau en [s]: 0.1
Tiempo de vuelo: 2.2
Alcance: 24.3952
\end{verbatim}%$ 
La salida en Octave debiera ser muy similar. 

La trayectoria calculada por el programa es mostrada en la figura
\ref{c10-f2}. Usando un paso de $\tau=0.1$~[s], el error en el alcance
horizontal es sobre un metro, como esper{\'a}bamos del error de
truncamiento. A velocidades bajas los resultados no son muy diferentes
si incluimos la resistencia con el aire, ya que $|\vec F_{a}(\vec
v_1)|/m\approx g/7$.

Ahora tratemos de batear un cuadrangular. Consideremos una velocidad
inicial grande $\modulo{v_1}=50$~[m/s]. Debido a la resistencia,
encontramos que el alcance es reducido a alrededor de 125~[m], menos
de la mitad de su maximo te{\'o}rico. La trayectoria es mostrada figura
\ref{c10-f3}, notemos c\'omo se aparta de la forma parab{\'o}lica.

En nuestras ecuaciones para el vuelo de una pelota de {\em baseball}
no hemos incluido todos los factores en el problema. El coeficiente de
arrastre no es constante sino m{\'a}s bien una complicada funci{\'o}n de la
velocidad. Adem{\'a}s, la rotaci{\'o}n de la pelota ayuda a levantar la pelota
(efecto Magnus).

\begin{figure}[!h]
\begin{center}
\includegraphics[angle =-90, width=10cm]{c10-f3}
\caption{Salida del programa {\tt balle} para una altura inicial de
  1~[m], una velocidad inicial de 50~[m/s], y un paso de tiempo
  $\tau=$0.1~[s]. Con resistencia del aire.}\label{c10-f3}
\end{center}
\end{figure}

\section{P{\'e}ndulo simple.}

\subsection{Ecuaciones b{\'a}sicas.}

El movimiento de los p{\'e}ndulos ha fascinado a f{\'\i}sicos desde que Galileo
fue hipnotizado por la l{\'a}mpara en la Catedral de Pisa. El problema es
tratado en los textos de mec{\'a}nica b{\'a}sica pero antes de apresurarnos a
calcular con el computador, revisemos algunos resultados b{\'a}sicos. Para
un p{\'e}ndulo simple es m{\'a}s conveniente describir la posici{\'o}n en t{\'e}rminos
del desplazamiento angular, $\theta(t)$. La ecuaci{\'o}n de movimiento es
\begin{equation}
\label{c10-e2.26}
\frac{d^2\theta}{dt^2}= - \frac{g}{L}\sen \theta\ ,
\end{equation}
donde $L$ es la longitud del brazo y $g$ es la aceleraci{\'o}n de
gravedad. En la aproximaci{\'o}n para {\'a}ngulo peque{\~n}o, $\sen \theta \approx \theta$, la
ecuaci{\'o}n (\ref{c10-e2.26}) se simplifica a
\begin{equation}
\label{c10-e2.27}
\frac{d^2\theta}{dt^2}= - \frac{g}{L}\theta\ .
\end{equation} 
Esta ecuaci{\'o}n diferencial ordinaria es f{\'a}cilmente resuelta para
obtener
\begin{equation}
\label{c10-e2.28}
\theta(t)=C_1\cos \left ( \frac{2\pi t}{T_s} + C_2 \right )\ ,
\end{equation}
donde las constantes $C_1$ y $C_2$ est{\'a}n determinadas por los valores
iniciales de $\theta$ y $\omega=d\theta/ dt$. El per{\'\i}odo para {\'a}ngulos peque{\~n}os,
$T_s$ es
\begin{equation}
\label{c10-e2.29}
T_s=2\pi\sqrt{\frac Lg}\ .
\end{equation}
Esta aproximaci{\'o}n es razonablemente buena para oscilaciones con
amplitudes menores o iguales a $20\grados$.

Sin la aproximaci{\'o}n para {\'a}ngulos peque{\~n}os, la ecuaci{\'o}n de movimiento
es m{\'a}s dif{\'\i}cil de resolver. Sin embargo, sabemos de la experiencia que
el movimiento es todav{\'\i}a peri{\'o}dico. En efecto, es posible obtener una
expresi{\'o}n para el per{\'\i}odo sin resolver expl{\'\i}citamente $\theta(t)$. La
energ{\'\i}a total es
\begin{equation}
\label{c10-e2.30}
E=\frac 12 m L^2\omega^2- mgL\cos \theta\ ,
\end{equation}
donde $m$ es la masa de la lenteja. La energ{\'\i}a total es conservada e
igual a $E=-mgL\cos \theta_m$, donde $\theta_m$ es el {\'a}ngulo m{\'a}ximo. De lo
anterior, tenemos 
\begin{equation}
\label{c10-e2.31}
\frac 12 m L^2\omega^2 -mgL\cos \theta = -mgL\cos \theta_m\ ,
\end{equation}
o
\begin{equation}
\label{c10-e2.32}
\omega^2 =\frac{2g}L \left ( \cos \theta - \cos \theta_m \right ) \ .
\end{equation}
Ya que $\omega=d\theta/dt$, 
\begin{equation}
\label{c10-e2.33}
dt=\frac{d\theta}{\sqrt{\dfrac {2g}L \left ( \cos \theta - \cos \theta_m \right
    )}}\ .
\end{equation} 
En un per{\'\i}odo el p{\'e}ndulo se balancea de $\theta=\theta_m$ a $\theta=-\theta_m$ y regresa a
$\theta=\theta_m$. As{\'\i}, en medio per{\'\i}odo el p{\'e}ndulo se balancea desde $\theta=\theta_m$ a
$\theta=-\theta_m$. Por {\'u}ltimo, por el mismo argumento, en un cuarto de per{\'\i}odo
el p{\'e}ndulo se balancea desde $\theta=\theta_m$ a $\theta=0$, as{\'\i} integrando ambos
lados de la ecuaci{\'o}n (\ref{c10-e2.33})
\begin{equation}
\label{c10-e2.34}
\frac{T}{4}=\sqrt{\frac L{2g} } \int_0^{\theta_m}\frac{d\theta}{\sqrt{\left ( \cos \theta - \cos \theta_m \right
    )}}\ .
\end{equation} 

Esta integral podr{\'\i}a ser reescrita en t{\'e}rminos de funciones especiales
usando la identidad $\cos 2\theta=1-2\sen^2 \theta$, tal que 
\begin{equation}
\label{c10-e2.35}
T=2\sqrt{\frac Lg } \int_0^{\theta_m}\frac{d\theta}{\sqrt{\left ( \sen^2
      \theta_m/2 -\sen^2 \theta/2\right  )}}\ .
\end{equation} 
Introduciendo $K(x)$, la integral el{\'\i}ptica completa de primera
especie,\footnote{I.S. Gradshteyn and I.M. Ryzhik, {\em Table of
    Integral, Series and Products} (New York: Academic Press, 1965)}
\begin{equation}
\label{c10-e2.36}
K(x)\equiv\int_0^{\pi/2} \frac{dz}{\sqrt{1-x^2\sen^2 z}}\ ,
\end{equation}
podr{\'\i}amos escribir el per{\'\i}odo como 
\begin{equation}
\label{c10-e2.37}
T=4\sqrt{\frac{L}{g}} K(\sen \theta_m/2 )\ ,
\end{equation}
usando el cambio de variable $\sen z= \sen( \theta/2)/ \sen(\theta_m/2)$. Para
valores peque{\~n}os de $\theta_m$, podr{\'\i}amos expandir $K(x)$ para obtener
\begin{equation}
\label{c10-e2.38}
T=2\pi \sqrt{\frac{L}{g} }\left ( 1 + \frac{1}{16}\,  \theta^2_m + \ldots \right
)\ .
\end{equation}
Note que el primer t{\'e}rmino es la aproximaci{\'o}n para {\'a}ngulo peque{\~n}o
(\ref{c10-e2.29}). 

\subsection{F{\'o}rmulas para la derivada centrada.}

Antes de programar el problema del p{\'e}ndulo miremos un par de otros
esquemas para calcular el movimiento de un objeto. El m{\'e}todo de Euler
est{\'a} basado en la formulaci{\'o}n de la derivada derecha para $df/dt$ dado
por (\ref{c10-e2.7}). Una definici{\'o}n equivalente para la derivada es 
\begin{equation}
\label{c10-e2.39}
f'(t)=\lim_{\tau\to0} \frac{f(t+\tau)-f(t-\tau)}{2\tau}\ .
\end{equation}
Esta f{\'o}rmula se dice centrada en $t$. Mientras esta f{\'o}rmula parece muy
similar a la ecuaci{\'o}n (\ref{c10-e2.7}), hay una gran diferencia cuando
$\tau$ es finito. Nuevamente, usando la expansi{\'o}n de Taylor,
\begin{align}
\label{c10-e2.40}
f(t+\tau)&=f(t)+\tau f'(t)+\frac 12 \tau^2 f''(t)+\frac 16 \tau^3 f^{(3)}(\zeta_+)\ ,\\
\label{c10-e2.41}
f(t-\tau)&=f(t)-\tau f'(t)+\frac 12 \tau^2 f''(t)-\frac 16 \tau^3 f^{(3)}(\zeta_-)\ ,
\end{align}
donde $f^{(3)}$ es la tercera derivada de $f(t)$ y $\zeta_{\pm}$ es un
valor ente $t$ y $t \pm\tau$. Restando las dos ecuaciones anteriores y
reordenando tenemos,
\begin{equation}
\label{c10-e2.42}
f'(t)=\frac{f(t+\tau)-f(t-\tau)}{2\tau}-\frac 16 \tau^2 f^{(3)}(\zeta)\ ,
\end{equation}
donde $t-\tau\leq \zeta\leq t+\tau$. Esta es la {\em aproximaci{\'o}n en la primera
  derivada centrada}. El punto clave es que el error de truncamiento
es ahora cuadr{\'a}tico en $\tau$, lo cual es un gran progreso sobre la
aproximaci{\'o}n de las derivadas adelantadas que tiene un error de
truncamiento $\DelOrdenD(\tau)$.

Usando las expansiones de Taylor para $f(t+\tau)$ y $f(t-\tau)$ podemos
construir una f{\'o}rmula centrada para la segunda derivada. La que tiene
la forma
\begin{equation}
\label{c10-e2.43}
f''(t)=\frac{f(t+\tau)+f(t-\tau)-2f(t)}{\tau^2}- \frac 1{12}\tau^2 f^{(4)}(\zeta)\ ,
\end{equation}
donde $t-\tau\leq\zeta\leq t+\tau$. De nuevo, el error de truncamiento es cuadr{\'a}tico
en $\tau$. La mejor manera de entender esta f{\'o}rmula es pensar que la
segunda derivada est{\'a} compuesta de una derivada derecha y de una
derivada izquierda, cada una con incrementos de $\tau/2$.

Usted podr{\'\i}a pensar que el pr{\'o}ximo paso ser{\'\i}a preparar f{\'o}rmulas m{\'a}s
complicadas que tengan errores de truncamiento a{\'u}n m{\'a}s peque{\~n}os,
quiz{\'a}s usando ambas $f(t\pm\tau)$ y $f(t\pm 2\tau)$. Aunque tales f{\'o}rmulas
existen y son ocasionalmente usadas, las ecuaciones (\ref{c10-e2.10}),
(\ref{c10-e2.42}) y (\ref{c10-e2.43}) sirven como el ``caballo de
trabajo'' para calcular las derivadas primera y segunda.

\subsection{M{\'e}todos del ``salto de la rana'' y de Verlet.}

Para el p{\'e}ndulo, las posiciones y velocidades generalizadas son $\theta$ y
$\omega$, pero para mantener la misma notaci{\'o}n anterior trabajaremos con $\vec
r$ y $\vec v$. Comenzaremos de las ecuaciones de movimiento escritas
como 
\begin{align}
\label{c10-e2.44}
\frac{d\vec v}{dt} &= \vec a(\vec r(t))\ ,\\
\label{c10-e2.45}
\frac{d\vec r}{dt} &= \vec v(t)\ .
\end{align}
Note que expl{\'\i}citamente escribimos la aceleraci{\'o}n dependiente
solamente de la posici{\'o}n. Discretizando la derivada temporal usando la
aproximaci{\'o}n de derivada centrada da,
\begin{equation}
\label{c10-e2.46}
\frac {\vec v(t+\tau)- \vec v(t-\tau)}{2\tau}+\DelOrdenD(\tau^2)=\vec a(\vec
r(t))\ ,
\end{equation}
para la ecuaci{\'o}n de la velocidad. Note que aunque los valores de
velocidad son evaluados en $t+\tau$ y $t-\tau$, la aceleraci{\'o}n es evaluada
en el tiempo $t$.

Por razones que pronto ser{\'a}n claras, la discretizaci{\'o}n de la ecuaci{\'o}n
de posici{\'o}n estar{\'a} centrada entre $t+2\tau$ y $t$, 
\begin{equation}
\label{c10-e2.47}
\frac{\vec r(t+2\tau)- \vec r(t)}{2\tau}+\DelOrdenD(\tau^2)=\vec v(t+\tau)\ .
\end{equation}
De nuevo usamos la notaci{\'o}n $f_n\equiv f(t=(n-1)\tau)$, en la cual la
ecuaci{\'o}n (\ref{c10-e2.47}) y (\ref{c10-e2.46}) son escritas como, 
\begin{align}
\label{c10-e2.48}
\frac {\vec v_{n+1}- \vec v_{n-1}}{2\tau}+\DelOrdenD(\tau^2)&=\vec a(\vec r_n)\ ,\\
\label{c10-e2.49}
\frac{\vec r_{n+2}- \vec r_n}{2\tau}+\DelOrdenD(\tau^2)&=\vec v_{n+1}\ .
\end{align}
Reordenando los t{\'e}rminos para obtener los valores futuros a la
izquierda, 
\begin{align}
\label{c10-e2.50}
\vec v_{n+1}&= \vec v_{n-1}+2\tau\vec a(\vec r_n)+ \DelOrdenD(\tau^3)\ ,\\
\label{c10-e2.51}
\vec r_{n+2} &=\vec r_n + 2\tau\vec v_{n+1}+\DelOrdenD(\tau^3)\ ,
\end{align}
el cual es el {\em m{\'e}todo del ``salto de la rana'' (leap frog)}.
Naturalmente, cuando el m{\'e}todo es usado en un programa, el t{\'e}rmino
$\DelOrdenD(\tau^3)$ no va y por lo tanto constituye el error de
truncamiento para el m{\'e}todo.

El nombre ``salto de la rana'' es usado ya que la soluci{\'o}n avanza en
pasos de $2\tau$, con la posici{\'o}n evaluada en valores impares ($\vec
r_1$, $\vec r_3$, $\vec r_5$, \ldots), mientras que la velocidad est{\'a}
calculada en los valores pares ($\vec v_2$, $\vec v_4$, $\vec v_6$,
\ldots). Este entrelazamiento es necesario ya que la aceleraci{\'o}n, la cual
es una funci{\'o}n de la posici{\'o}n, necesita ser evaluada en a tiempo, esto
es centrada entre la nueva velocidad y la antigua. Algunas veces el
esquema del ``salto de la rana'' es formulado como
\begin{align}
\label{c10-e2.52}
\vec v_{n+1/2}&= \vec v_{n-1/2}+\tau\vec a(\vec r_n)\ ,\\
\label{c10-e2.53}
\vec r_{n+1} &=\vec r_n + \tau\vec v_{n+1/2}\ ,
\end{align}
con $\vec v_{n\pm1/2}\equiv \vec v(t=(n-1\pm1/2)\tau)$. En esta forma, el
esquema es funcionalmente equivalente al m{\'e}todo de Euler-Cromer. 

Para el {\'u}ltimo esquema num{\'e}rico de este cap{\'\i}tulo tomaremos una
aproximaci{\'o}n diferente y empezaremos con,
\begin{align}
\label{c10-e2.54}
\frac{d\vec r}{dt}&=\vec v(t)\ ,\\
\label{c10-e2.55}
\frac{d^2\vec r}{dt^2}&=\vec a(\vec r)\ .
\end{align}
Usando las f{\'o}rmulas diferenciales centradas para la primera y segunda
derivada, tenemos 
\begin{align}
\label{c10-e2.56}
\frac{\vec r_{n+1}-\vec r_{n-1}}{2\tau}+ \DelOrdenD(\tau^2)&=\vec v_n\ ,\\
\label{c10-e2.57}
\frac{\vec r_{n+1}+\vec r_{n-1}-2\vec r_n}{\tau^2}+ \DelOrdenD(\tau^2)&=\vec a_n\ ,
\end{align}
donde $\vec a_n\equiv \vec a (\vec r_n)$. Reordenando t{\'e}rminos, 
\begin{align}
\label{c10-e2.58}
\vec v_n &= \frac{\vec r_{n+1}-\vec r_{n-1}}{2\tau}+ \DelOrdenD(\tau^2)\ ,\\[2mm]
\label{c10-e2.59}
\vec r_{n+1}&=2\vec r_{n}-\vec r_{n-1}+ \tau^2\vec a_n +\DelOrdenD(\tau^4)\ .
\end{align}

Estas ecuaciones, conocidas como el {\em m{\'e}todo de
  Verlet}\footnote{L.Verlet, ``Computer experiments on classical fluid
  I. Thermodynamical properties of Lennard-Jones molecules'', {\em
    Phys. Rev.} {\bf 159}, 98-103 (1967).}, podr{\'\i}an parecer extra{\~n}as a
primera vista, pero ellas son f{\'a}ciles de usar. Suponga que conocemos
$\vec r_0$ y $\vec r_1$; usando la ecuaci{\'o}n (\ref{c10-e2.59}),
obtenemos $\vec r_2$. Conociendo $\vec r_1$ y $\vec r_2$ podr{\'\i}amos
ahora calcular $\vec r_3$, luego usando la ecuaci{\'o}n (\ref{c10-e2.58})
obtenemos $\vec v_2$, y as{\'\i} sucesivamente.

Los m{\'e}todos del ``salto de la rana'' y de Verlet tienen la desventaja que
no son ``autoiniciados''. Usualmente tenemos las condiciones iniciales
$\vec r_1=\vec r(t=0)$ y $\vec v_1=\vec v(t=0)$, pero no $\vec v_0=
\vec v(t=-\tau)$ [necesitado por el ``salto de la rana'' en la ecuaci{\'o}n
(\ref{c10-e2.50})] o $\vec r_0=\vec r(t=-\tau)$ [necesitado por Verlet
en la ecuaci{\'o}n (\ref{c10-e2.59})]. Este es el precio que hay que pagar
para los esquemas centrados en el tiempo.

Para lograr que estos m{\'e}todos partan, tenemos una variedad de
opciones. El m{\'e}todo de Euler-Cromer, usando la ecuaci{\'o}n
(\ref{c10-e2.53}), toma $\vec v_{1/2}=\vec v_1$, lo cual es simple
pero no muy precisa. Una alternativa es usar otro esquema para lograr
que las cosas partan, por ejemplo, en el ``salto de la rana'' uno
podr{\'\i}a tomar un paso tipo Euler para atr{\'a}s, $\vec v_0=\vec v_1-\tau\vec
a_1$. Algunas precauciones deber{\'\i}an ser tomadas en este primer paso
para preservar la precisi{\'o}n del m{\'e}todo; usando
\begin{equation}
\label{c10-e2.60}
\vec r_0=\vec r_1-\tau \vec v_1 + \frac{\tau^2}{2} \, \vec a(\vec r_1)\ ,
\end{equation}
es una buena manera de comenzar el m{\'e}todo de Verlet.

Adem{\'a}s de su simplicidad, el m{\'e}todo del ``salto de la rana'' a menudo
tiene propiedades favorables ({\it e.g.} conservaci{\'o}n de la energ{\'\i}a)
cuando resuelve ciertos problemas. El m{\'e}todo de Verlet tiene muchas
ventajas. Primero, la ecuaci{\'o}n de posici{\'o}n tiene un error de
truncamiento menor que otros m{\'e}todos. Segundo, si la fuerza es
solamente una funci{\'o}n de la posici{\'o}n y si nos preocuparnos s{\'o}lo de la
trayectoria de la part{\'\i}cula y no de su velocidad (como en muchos
problemas de mec{\'a}nica celeste), podemos saltarnos completamente el
c{\'a}lculo de velocidad. El m{\'e}todo es popular para el c{\'a}lculo de las
trayectorias en sistemas con muchas part{\'\i}culas, por ejemplo, el
estudio de fluidos a nivel microsc{\'o}pico.

\begin{table}
\hrulefill
\begin{itemize}
\item Seleccionar el m{\'e}todo a usar: Euler o Verlet.
\item Fijar la posici{\'o}n inicial $\theta_1$ y la velocidad $\omega_1=0$ del p{\'e}ndulo.
\item Fijar los par{\'a}metros f{\'\i}sicos y otras variables.
\item Tomar un paso para atr{\'a}s para partir Verlet; ver ecuaci{\'o}n (\ref{c10-e2.60}).
\item Iterar sobre el n{\'u}mero deseado de pasos con el paso de tiempo y
  m{\'e}todo num{\'e}rico dado.
  \begin{itemize}
  \item Grabar {\'a}ngulo y tiempo para graficar.
  \item Calcular la nueva posici{\'o}n y velocidad usando el m{\'e}todo de
    Euler o de Verlet.
  \item Comprobar si el p{\'e}ndulo a pasado a trav{\'e}s de $\theta=0$; Si es
    as{\'\i} usar el tiempo transcurrido para estimar el per{\'\i}odo.
\end{itemize}
\item Estima el per{\'\i}odo de oscilaci{\'o}n, incluyendo barra de error.
\item Graficar las oscilaciones como $\theta$ versus $t$.
\end{itemize}
\caption{Bosquejo del programa {\tt pendulo}, el cual calcula el tiempo
  de evoluci{\'o}n de un p{\'e}ndulo simple usando el m{\'e}todo de Euler o Verlet.}\label{c10-t2}
\hrulefill
\end{table}

\subsection{Programa de p{\'e}ndulo simple.}

Las ecuaciones de movimiento para un p{\'e}ndulo simple son
\begin{equation}
\label{c10-e2.61}
\frac{d\omega}{dt}=\alpha(\theta)\, \quad \frac{d\theta}{dt}=\omega\ ,
\end{equation}
donde la aceleraci{\'o}n angular $\alpha(\theta)=-g \sen\theta /L$. El m{\'e}todo de Euler
para resolver estas ecuaciones diferenciales ordinarias es iterar las
ecuaciones: 
\begin{align}
\label{c10-e2.62}
\theta_{n+1}=\theta_n+\tau\omega_n\ ,\\
\label{c10-e2.63}
\omega_{n+1}=\omega_n+\tau\alpha_n\ .
\end{align}
Si estamos interesados solamente en el {\'a}ngulo y no la velocidad, el
m{\'e}todo de Verlet s{\'o}lo usa la ecuaci{\'o}n
\begin{equation}
\label{c10-e2.64}
\theta_{n+1}=2\theta_n-\theta_{n-1}+\tau^2\alpha_n\ .
\end{equation}

En vez de usar las unidades SI, usaremos las unidades adimensionales
naturales del problema. Hay solamente dos par{\'a}metros en el problema,
$g$ y $L$ y ellos siempre aparecen en la raz{\'o}n $g/L$. Fijando esta
raz{\'o}n a la unidad, el per{\'\i}odo para peque{\~n}as amplitudes $T_s=2\pi$. En
otras palabras, necesitamos s{\'o}lamente una unidad en el problema: una
escala de tiempo. Ajustamos nuestra unidad de tiempo tal que el
per{\'\i}odo de peque{\~n}as amplitudes sea $2\pi$.

La tabla \ref{c10-t2} presenta un bosquejo del programa \verb|pendulo|,
el cual calcula el movimiento de un p{\'e}ndulo simple usando o el m{\'e}todo
de Euler o el de Verlet. El programa estima el per{\'\i}odo por registrar
cuando el {\'a}ngulo cambia de signo; esto es verificar si $\theta_n$ y
$\theta_{n+1}$ tienen signos opuestos probando si $\theta_n*\theta_{n+1}<0$. Cada
cambio de signo da una estimaci{\'o}n para el per{\'\i}odo,
$\tilde{T}_k=2\tau(n_{k+1}-n_k)$, donde $n_k$ es el paso de tiempo en el
cual el $k$-{\'e}simo cambio de signo ocurre. El per{\'\i}odo estimado de cada
inversi{\'o}n de signo es registrado, y el valor medio calculado como
\begin{equation}
\label{c10-e2.65}
\left \langle \tilde{T}\right \rangle =\frac 1M\sum_{k=1}^M \tilde{T}_k\ ,
\end{equation}
donde $M$ es el n{\'u}mero de veces que $\tilde{T}$ es evaluado. La barra
de error para esta medici{\'o}n del per{\'\i}odo es estimada como $\sigma=s/M$,
donde
\begin{equation}
\label{c10-e2.66}
s=\sqrt{\frac{1}{M-1}\sum_{k=1}^M 
\left (\tilde{T}_k-\left \langle \tilde{T}\right \rangle\right )^2}\ ,
\end{equation}
es la desviaci{\'o}n estandar de la muestra $\tilde{T}$. Note que cuando
el n{\'u}mero de medidas se incrementa, la desviaci{\'o}n estandar de la
muestra tiende a una constante, mientras que la barra de error
estimado decrese.

Para comprobar el programa \verb|pendulo|, primero tratamos con {\'a}ngulos
iniciales peque{\~n}os, $\theta_m$, ya que conocemos el per{\'\i}odo $T\approx 2\pi$.
Tomando $\tau=0.1$ tendremos sobre 60 puntos por oscilaci{\'o}n; tomando 300
pasos deber{\'\i}amos tener como cinco oscilaciones. Para $\theta_m=10\grados$,
El m{\'e}todo de Euler calcula un per{\'\i}odo estimado de $\langle \tilde{T}\rangle=
6.375\pm0.025$ sobre un 1.5\% mayor que el esperado $T=2\pi(1.002)$ dado
por la ecuaci{\'o}n (\ref{c10-e2.38}). Nuestro error estimado para el
per{\'\i}odo es entorno a $\pm\tau$ en cada medida. Cinco oscilaciones son 9
medidas de $\tilde T$ , as{\'\i} que nuestro error estimado para el per{\'\i}odo
deber{\'\i}a ser $(\tau/2)/ \sqrt{9}\approx0.02$. Notemos que la estimaci{\'o}n est{\'a} en
buen acuerdo con los resultados obtenidos usando la desviaci{\'o}n
estandar.  Hasta aqu{\'\i} todo parece razonable.

\begin{figure}[!h]
\begin{center}
\includegraphics[angle =-90, width=10cm]{c10-f4}
\caption{Salida del programa {\tt pendulo} usando el m{\'e}todo de
  Euler. El {\'a}ngulo inicial es $\theta_m=10\grados$, el paso en el tiempo es
  $\tau=0.1$, y 300 iteraciones fueron calculadas.}\label{c10-f4}
\end{center}
\end{figure}

Infortunadamente si miramos el grafico \ref{c10-f3} nos muestra los
problemas del m{\'e}todo de Euler. La amplitud de oscilaci{\'o}n crece con el
tiempo. Ya que la energ{\'\i}a es proporcional al {\'a}ngulo m{\'a}ximo, esto
significa que la energ{\'\i}a total se incrementa en el tiempo. El error
global de truncamiento en el m{\'e}todo de Euler se acumula en este caso.
Para pasos de tiempos peque{\~n}os $\tau=0.05$ e incrementos en el n{\'u}mero de
pasos (600) podemos mejorar los resultados, ver figura \ref{c10-f4},
pero no eliminamos el error. El m{\'e}todo del punto medio tiene la misma
inestabilidad num{\'e}rica.

\begin{figure}[!h]
\begin{center}
\includegraphics[angle =-90, width=10cm]{c10-f5}
\caption{Salida del programa {\tt pendulo} usando el m{\'e}todo de
  Euler. El {\'a}ngulo inicial es $\theta_m=10\grados$, el paso en el tiempo es
  $\tau=0.05$ y 600 iteraciones fueron calculadas.}\label{c10-f5}
\end{center}
\end{figure}

\begin{figure}[!h]
\begin{center}
\includegraphics[angle =-90, width=10cm]{c10-f6}
\caption{Salida del programa {\tt pendulo} usando el m{\'e}todo de
  Verlet. El {\'a}ngulo inicial es $\theta_m=10\grados$, el paso en el tiempo es
  $\tau=0.1$ y 300 iteraciones fueron calculadas.}\label{c10-f6}
\end{center}
\end{figure}

Usando el m{\'e}todo de Verlet con $\theta_m=10\grados$, el paso en el tiempo
$\tau=0.1$ y 300 iteraciones obtenemos los resultados graficados en
\ref{c10-f5}. Estos resultados son mucho mejores; la amplitud de
oscilaci{\'o}n se mantiene cerca de los $10\grados$ y $\langle \tilde
T\rangle=6.275\pm0.037$. Afortunadamente el m{\'e}todo de Verlet, el del ``salto
de rana'' y el de Euler-Cromer no sufren de la inestabilidad num{\'e}rica
encontrada al usar el m{\'e}todo de Euler.

\begin{figure}[!h]
\begin{center}
\includegraphics[angle =-90, width=10cm]{c10-f7}
\caption{Salida del programa {\tt pendulo} usando el m{\'e}todo de
  Verlet. El {\'a}ngulo inicial es $\theta_m=170\grados$, el paso en el tiempo es
  $\tau=0.1$ y 300 iteraciones fueron calculadas.}\label{c10-f7}
\end{center}
\end{figure}

Para $\theta_m=90\grados$, la primera correcci{\'o}n para la aproximaci{\'o}n de
{\'a}ngulo peque{\~n}o, ecuaci{\'o}n (\ref{c10-e2.38}), da $T=7.252$. Usando el
m{\'e}todo de Verlet, el programa da un per{\'\i}odo estimado de
$\langle\tilde{T}\rangle=7.414\pm 0.014$, lo cual indica que (\ref{c10-e2.38}) es
una buena aproximaci{\'o}n (alrededor de un 2\% de error), a{\'u}n para {\'a}ngulos
grandes. Para el {\'a}ngulo muy grande de $\theta_m=170\grados$, vemos la
trayectoria en la figura \ref{c10-f6}. Notemos como la curva tiende a
aplanarse en los puntos de retorno. En este caso el per{\'\i}odo estimado
es $\langle\tilde{T}\rangle=15.3333+/-0.0667$, mientr{\'a}s que (\ref{c10-e2.38}) da
$T=9.740$, indicando que esta aproximaci{\'o}n para (\ref{c10-e2.37}) deja
de ser v{\'a}lida para este {\'a}ngulo tan grande.  


\section{Listado de los  programas.}

\subsection{\tt balle.cc}

\begin{verbatim}
#include "NumMeth.h"

main()
{
  const double Cd=0.35;
  const double rho=1.293;               // [kg/m^3]
  const double radio=0.037;                     // [m]
  double A= M_PI*radio*radio ;  
  double m=0.145;                               // [kg] 
  double g=9.8;                 // [m/s^2]
  double a = -Cd*rho*A/(2.0e0*m) ;

  double v0, theta0, tau;
  ofstream salida ("salida.txt") ;
  ofstream salidaT ("salidaT.txt") ;

  double x0, y0; 
  x0=0.0e0 ;
  cout << "Ingrese la altura inicial [m] : ";
  cin >> y0;
  cout << "Ingrese la velocidad inicial [m/s]: ";
  cin >> v0;
  cout <<"Ingrese angulo inicial (grados): ";
  cin >> theta0;

  int flagRA = 2 ;
  while (flagRA!=0 && flagRA !=1) {
    cout <<"Con resistencia del aire, Si= 1, No= 0: ";
    cin >> flagRA;
  }
  cout <<"Ingrese el paso en el tiempo, tau en [s]: "; 
  cin >> tau ;
  double vxn=v0*cos(M_PI*theta0/180.0) ;
  double vyn=v0*sin(M_PI*theta0/180.0) ;
  double xn=x0 ;
  double yn=y0 ;
  double tiempo = -tau;
  while( yn >= y0) {
    tiempo +=tau ;
    salidaT << x0+v0*cos(M_PI*theta0/180.0) *tiempo <<" " ;
    salidaT << y0+v0*sin(M_PI*theta0/180.0) *tiempo -g*tiempo*tiempo/2.0e0<< endl;
    salida << xn << " " << yn << endl;
    if(flagRA==0) a=0.0e0 ; 
    double v=sqrt(vxn*vxn+vyn*vyn) ;
    double axn= a*v*vxn ;
    double ayn= a*v*vyn -g ;
    double xnp1 = xn + tau*vxn ;
    double ynp1 = yn + tau*vyn ;
    double vxnp1 = vxn + tau*axn;
    double vynp1 = vyn + tau*ayn;
    vxn=vxnp1;
    vyn=vynp1;
    xn=xnp1 ;
    yn=ynp1 ;
  }
  cout << "Tiempo de vuelo: " << tiempo<< endl;
  cout << "Alcance: " << xn<<endl;
  salida.close();
}
\end{verbatim}

\subsection{\tt pendulo.cc}

\begin{verbatim}
#include "NumMeth.h"

main()
{
  int respuesta=2 ;
  while(respuesta != 0 && respuesta !=1 ) {
    cout << "Eliga el metodo: Euler=0 y Verlet=1: " ;
    cin >> respuesta ;
  }
  double theta1 ;
  double omega1 = 0.0e0;
  cout << "Ingrese el angulo inicial (grados): ";
  cin >> theta1 ;
  theta1*=M_PI/180.0e0 ;
  double tau ;
  cout << "Ingrese el paso de tiempo: ";
  cin >> tau ;
  int pasos ;
  cout << "Ingrese el numero de pasos: ";
  cin >> pasos ;
  
  double * periodo = new double[pasos] ;

  ofstream salida ("salidaPendulo.txt");

  double theta0= theta1-tau*omega1-tau*tau*sin(theta1) ;
  
  double thetaNm1=theta0 ;
  double thetaN=theta1 ;
  double omegaN=omega1;

  double thetaNp1, omegaNp1 ;

  int nK=1;
  int M=0 ;

  for(int i=1; i< pasos; i++) {
    double alphaN=-sin(thetaN);
    if(respuesta==0) {          // Euler
      thetaNp1=thetaN+tau*omegaN ;
      omegaNp1=omegaN+tau*alphaN ;
    } else {
      thetaNp1=2.0e0*thetaN-thetaNm1+tau*tau*alphaN ;
    }
    salida << (i-1)*tau<<" " <<thetaNp1*180/M_PI<< endl ;
    if (thetaNp1*thetaN<0) {
      if(M==0) {
        periodo[M++]=0.0e0;
        nK=i ;
      } else {
        periodo[M++] = 2.0e0*tau*double(i-nK);
        nK=i ;
      }
    }

    thetaNm1=thetaN ;
    thetaN=thetaNp1 ;
    omegaN=omegaNp1 ;
  }
  double Tprom=0.0e0;
  for (int i=1; i < M; i++) Tprom+=periodo[i] ;
  Tprom/=double(M-1) ;
  double ssr=0.0 ;
  for (int i=1; i < M; i++) ssr+=(periodo[i]-Tprom)* (periodo[i]-Tprom);
  ssr/=double(M-2);
  double sigma =sqrt(ssr/double(M-1)) ;
  cout <<" Periodo = "<< Tprom << "+/-"<< sigma << endl ; 
  salida.close() ;
  delete [] periodo;
}
\end{verbatim}

% Local Variables: 
% TeX-master: "mfm"
% End: 
