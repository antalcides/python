\documentclass[12pt,keys]{mfm2}      
\pagestyle{empty}        
\addtolength{\textwidth}{-5cm}
\addtolength{\oddsidemargin}{2.5cm}
\begin{document}

\noindent
{\bf Caso 1:} \verb+\documentclass[12pt]{mfm2}+
\vspace{.5cm}

La opci\'on \verb+keys+ resulta muy \'util cuando tengo objetos numerados
autom\'aticamente, como una ecuaci\'on:
\begin{equation}
  \vec F = m \vec a \ . 
\end{equation}
y luego quiero referirme a ella: Ec.\  \eqref{newton}.

\vspace{2cm}
\addtocounter{equation}{-1}

\noindent
{\bf Caso 2:} \verb+\documentclass[keys,12pt]{mfm2}+
\vspace{.5cm}

La opci\'on \verb+keys+ resulta muy \'util cuando tengo objetos numerados
autom\'aticamente, como una ecuaci\'on:
\begin{equation}
  \label{newton}
  \vec F = m \vec a \ . 
\end{equation}
y luego quiero referirme a ella: Ec.\  \eqref{newton}.

\end{document}


