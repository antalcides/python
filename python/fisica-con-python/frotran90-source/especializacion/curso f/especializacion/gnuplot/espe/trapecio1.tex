\documentclass{article}%
\usepackage{amsmath}%
\setcounter{MaxMatrixCols}{30}%
\usepackage{amsfonts}%
\usepackage{amssymb}%
\usepackage{graphicx}
%TCIDATA{OutputFilter=latex2.dll}
%TCIDATA{Version=4.00.0.2312}
%TCIDATA{CSTFile=40 LaTeX article.cst}
%TCIDATA{Created=Friday, October 24, 2003 07:58:30}
%TCIDATA{LastRevised=Friday, October 24, 2003 10:02:02}
%TCIDATA{<META NAME="GraphicsSave" CONTENT="32">}
%TCIDATA{<META NAME="DocumentShell" CONTENT="Standard LaTeX\Blank - Standard LaTeX Article">}
\newtheorem{theorem}{Theorem}
\newtheorem{acknowledgement}[theorem]{Acknowledgement}
\newtheorem{algorithm}[theorem]{Algorithm}
\newtheorem{axiom}[theorem]{Axiom}
\newtheorem{case}[theorem]{Case}
\newtheorem{claim}[theorem]{Claim}
\newtheorem{conclusion}[theorem]{Conclusion}
\newtheorem{condition}[theorem]{Condition}
\newtheorem{conjecture}[theorem]{Conjecture}
\newtheorem{corollary}[theorem]{Corollary}
\newtheorem{criterion}[theorem]{Criterion}
\newtheorem{definition}[theorem]{Definition}
\newtheorem{example}[theorem]{Example}
\newtheorem{exercise}[theorem]{Exercise}
\newtheorem{lemma}[theorem]{Lemma}
\newtheorem{notation}[theorem]{Notation}
\newtheorem{problem}[theorem]{Problem}
\newtheorem{proposition}[theorem]{Proposition}
\newtheorem{remark}[theorem]{Remark}
\newtheorem{solution}[theorem]{Solution}
\newtheorem{summary}[theorem]{Summary}
\newenvironment{proof}[1][Proof]{\noindent\textbf{#1.} }{\ \rule{0.5em}{0.5em}}
\begin{document}
\section{METODOS DE INTEGRACI\'{O}N NUMERICA}

\subsection{LA REGLA DEL TRAPECIO}

El m\'{e}todo de Euler aproxima la derivada en $\left[  x_{n},x_{n+1}\right]
$ por una constante, concretamente por su valor en $x_{n}$.
\textquestiondown Por qu\'{e} privilegiar al punto $x_{n}?$
\textquestiondown No ser\'{a} mejor tomar , por ejemplo, como aproximaci\'{o}n
constante de la derivada el promedio de sus valores en los extremos del
intervalo? En ese caso
\[
y(x_{n+1})=y\left(  x_{n}\right)  +h\frac{f\left(  x_{n},y\left(
x_{n}\right)  \right)  f\left(  x_{n+1},y\left(  x_{n+1}\right)  \right)  }%
{2}+R_{n}%
\]
donde $R_{n}$ es el error de truncaci\'{o}n, por lo que se aproxima de la forma.%

\[
y_{n+1}\approx y_{n}+h\frac{f\left(  x_{n},y\left(  x_{n}\right)  \right)
f\left(  x_{n+1},y\left(  x_{n+1}\right)  \right)  }{2}%
\]


Esta aproximaci\'{o}n conduce a la llamada regla del trapecio

la cual \ es convergente de grado dos, es decir%
\[
\left\Vert e_{n}\right\Vert \leq e^{\left(  b-a\right)  L/\left(  1-\frac
{hL}{2}\right)  }\left\Vert e_{0}\right\Vert +\frac{Ch^{2}}{L}\left(
e^{\left(  b-a\right)  L/\left(  1-\frac{hL}{2}\right)  }-1\right)  ,\text{ si
}0\leq n\leq N
\]%
\[
\text{donde }e_{n}=y\left(  x_{n}\right)  -y_{n},\ \frac{hL}{2}<1,\ C=\frac
{5}{12}\max_{x\in\left[  a,b\right]  }\left\Vert y^{\prime\prime\prime}\left(
x\right)  \right\Vert \
\]


a pesar de que la regla del trapecio tiene un orden de convergencia mayor  que
el m\'{e}todo de Euler. eso no significa que sea un m\'{e}todo  mejor , pues
hay otra diferencia importante entre ambos m\'{e}todos. el m\'{e}todo de Euler
es \textit{expl\'{\i}cito,} el valor de $y_{n+1}$ viene dado
expl\'{\i}citamente en t\'{e}rmino del valor anterior $y_{n}$ y se puede
calcular f\'{a}cilmente mediante la evaluaci\'{o}n de $f$ y una pocas
operaciones aritm\'{e}ticas. por el contrario, la regla del trapecio es un
m\'{e}todo \textit{impl\'{\i}cito}, para calcular $y_{n+1}$ hay que resolver
un sistema de ecuaciones no lineales, lo que en general es computacionalmente
costoso. de lo que se deduce, el orden no lo es todo. Aunque en un m\'{e}todo
de orden alto en principio hay que dar menos pasos, estos pueden ser muy
costosos, por lo que puede ser mejor un m\'{e}todo de menos orden en el que a
pesar de dar m\'{a}s pasos  cada uno de ellos lleve menos trabajo computacional.

\subsection{Aplicaciones}

Para aplicar este m\'{e}todo resolveremos dos problemas de f\'{\i}sica
m\'{e}canica anal\'{\i}ticamente y usando Fortran lo resolveremos
computacionalmente y luego calculamos el error.

\begin{enumerate}
\item un objeto que pesa $48\ lb$ se suelta desde el reposo en la parte
superior de un plano inclinado met\'{a}lico que tiene una inclinaci\'{o}n de
$30%
%TCIMACRO{\U{ba}}%
%BeginExpansion
{{}^o}%
%EndExpansion
$ respecto a la horizontal. la resistencia del aire es
num�ricamente igual a un medio de la velocidad (en pies por
segundo), el coeficiente de rozamiento es $\frac{1}{4}$

\begin{enumerate}
\item \textquestiondown Cu\'{a}l es la velocidad del objeto dos segundos
despu\'{e}s de haberse soltado?

\item Si el plano mide mide $24~ft$ de longitud, \textquestiondown C\'{u}al es
la velocidad del cuerpo en el momento que llega al punto inferior?%
%TCIMACRO{\FRAME{dhF}{2.7691in}{1.8273in}{0pt}{}{}{fortran.jpg}%
%{\special{ language "Scientific Word";  type "GRAPHIC";
%maintain-aspect-ratio TRUE;  display "USEDEF";  valid_file "F";
%width 2.7691in;  height 1.8273in;  depth 0pt;  original-width 2.7268in;
%original-height 1.7902in;  cropleft "0";  croptop "1";  cropright "1";
%cropbottom "0";  filename 'fortran.jpg';file-properties "XNPEU";}}}%
%BeginExpansion
\begin{center}
\includegraphics[
height=1.8273in,
width=2.7691in
]%
{fortran.jpg}%
\end{center}
%EndExpansion

\end{enumerate}
\end{enumerate}

\begin{solution}
La l\'{\i}nea de movimiento es a lo largo del plano, si escogemos el origen
del sistema de referencia en la parte superior y el sentido positivo de las
$x$ hacia abajo del plano, entonces las%
%TCIMACRO{\FRAME{dhF}{2.2053in}{1.6725in}{0pt}{}{}{plano.jpg}%
%{\special{ language "Scientific Word";  type "GRAPHIC";
%maintain-aspect-ratio TRUE;  display "USEDEF";  valid_file "F";
%width 2.2053in;  height 1.6725in;  depth 0pt;  original-width 2.8098in;
%original-height 2.1231in;  cropleft "0";  croptop "1";  cropright "1";
%cropbottom "0";  filename 'plano.jpg';file-properties "XNPEU";}}}%
%BeginExpansion
\begin{center}
\includegraphics[
height=1.6725in,
width=2.2053in
]%
{plano.jpg}%
\end{center}
%EndExpansion
fuerzas que act\'{u}na sobre el objeto A son:\newline Su peso de 48 lb que
act\'{u}a verticalmente hacia abajo.\newline La fuerza normal, N que ejerce el
plano sobre el objeto la cual act\'{u}a en la direcci\'{o}n positiva y
perpendicular al plano.\newline La resitencia del aire $f_{a}$,que tiene un
valor num\'{e}rico igual a $\frac{v}{2}$, puesto que $v>0$ esta fuerza tiene
una direcci\'{o}n negativa en $x.$\newline La fuerza de rozamiento, que tiene
un valor $\mu N$ y tiene una direcci\'{o}n negativa en $x.$\newline de la
gr\'{a}fica 2 y aplicando la segunda ley de Newton obtenemos
\begin{align*}
\text{ en }x  & :\ \ ma\ =Wsen30%
%TCIMACRO{\U{ba}}%
%BeginExpansion
{{}^o}%
%EndExpansion
-f_{s}-f_{r}\\
f_{r}  & =\mu N\\
\text{en }y  & :0=N-W\cos30%
%TCIMACRO{\U{ba}}%
%BeginExpansion
{{}^o}%
%EndExpansion
\end{align*}
reemplazando los valores y resolviendo el sistema de las tres ecuaciones
anteriores se obtiene:%
\[
\frac{3}{2}\frac{dv}{dt}=24-6\sqrt{3}-\frac{1}{2}v
\]
al separar variables se obtiene%
\[
\frac{dv}{48-12\sqrt{3}-v}=\frac{dt}{3}%
\]
como cuando $t=0,\ v=0$ se tiene%
\begin{align}
\int_{0}^{v_{f}}\frac{dv}{48-12\sqrt{3}-v}  & =\frac{dv}{48-12\sqrt{3}-v}%
=\int_{0}^{2}\frac{dt}{3}\label{ec1}\\
v  & =\left(  48-12\sqrt{3}\right)  \left(  1-e^{-\frac{t}{3}}\right)
\nonumber
\end{align}
de modo que al resolver la integral nos da
\[
v\left(  2\right)  =10.2\left(  ft/s\right)
\]
Para la segunda parte se integra \ref{ec1} y como $x\left(  0\right)  =0$%
\[
x=\left(  48-12\sqrt{3}\right)  \left(  t+3e^{-\frac{t}{3}}-3\right)
\]
despejando $t$ y reemplazando en \ref{ec1} se obtiene
\[
v=12.3~ft/s
\]

\end{solution}}}







\end{document}
