%&LaTeX
\documentclass{article}
                                                                                                    
                                                                                                    
                                                                                                    
                                                                                                    
                                                                                                    
                                                                                                    
                                                                                                    
                                                                                                    
                                                                                                    
                                                                                                    
\newcommand{\tab}{\hspace{5mm}}


\begin{document}

c Este programa utiliza la regla del trapecio\\
c para resolver el movimiento de un cuerpo\\
c sobre un plano inclinado considerando la\\
c resistencia del aire y la fricci\hbox{\rm\rlap/c}n \\
\textbf{program }trapecio \\
\textbf{real }a,b,d,suma,Ia,In,et,h,sum,Ln,La,el \\
\textbf{integer }i,n,j ,res \\
res =1\\
20 \textbf{if }(res\textbf{.eq.}1)\textbf{then \\
write}(*,*)'***************************************************' 
\\
\textbf{write}(*,*)'Integracion numerica' \\
\textbf{write}(*,*)'regla del Trapecio' \\
\textbf{write}(*,*)'***********************' \\
\textbf{write}(*,*) 'Ingrese el limite inferior entre 0 y L' \\
\textbf{read}(*,*) a \\
\textbf{write}(*,*) 'Ingrese el limite superior entre 0 y L' \\
\textbf{read}(*,*) b \\
\textbf{write}(*,*) 'Ingrese el numero N de puntos en [a,b], N es 
natural' \\
\textbf{read}(*,*) n \\
d=(b-a)/(2*(n-1)) \\
suma=0 \\
\textbf{do }2 i=1,n-2 \\
suma=suma+y(a+i*(2*d))\\
2 \textbf{continue }\\
In=d*(Y(a)+(2.*suma)+Y(b)) \\
Ia=g(b)-g(a) \\
et=abs((Ia-In)/Ia)*100\\
c definimos en esta parte la forma para determinar la distancia 
\\
h=In/(2*(n-1)) \\
sum=0 \\
\textbf{do }3 j=1,n-2 \\
sum=sum+R(j*(2*h))\\
3 \textbf{continue }\\
Ln=h*(R(0)+(2*sum)+R(In)) \\
La=T(In)-T(0) \\
el=abs((La-Ln)/La)*100. 

 \textbf{write}(6,*)'Resultado para N=',N,':' \\
\textbf{WRITE}(*,'(3(T3,A/),3(T3,A/T3,(3(A,1PE13.6))/))') \\
\&' \ensuremath{^3} Valor numerico\ensuremath{^3} Valor analit. \ensuremath{^3} ', 
\\
\&' \ensuremath{^3} estimado de la\ensuremath{^3} de la integral\ensuremath{^3} Diferencia 
', \\
\&' \ensuremath{^3} integral \ensuremath{^3} \ensuremath{^3} en \% ', \\
\&'\"{A}\"{A}\"{A}\"{A}\"{A}\"{A}\"{A}\"{A}\"{A}\"{A}\"{A}\"{A}\"{A}\"{A}{\AA}\"{A}\"{A}\"{A}\"{A}\"{A}\"{A}\"{A}\"{A}\"{A}\"{A}\"{A}\"{A}\"{A}\"{A}\"{A}{\AA}\"{A}\"{A}\"{A}\"{A}\"{A}\"{A}\"{A}\"{A}\"{A}\"{A}\"{A}\"{A}\"{A}\"{A}\"{A}{\AA}\"{A}\"{A}\"{A}\"{A}\"{A}\"{A}\"{A}\"{A}\"{A}\"{A}\"{A}\"{A}\"{A}\"{A}', 
\\
\&'Tiempo \ensuremath{^3} ' ,In , ' \ensuremath{^3} ' ,Ia , ' \ensuremath{^3} ',et 
, \\
\&'\"{A}\"{A}\"{A}\"{A}\"{A}\"{A}\"{A}\"{A}\"{A}\"{A}\"{A}\"{A}\"{A}\"{A}\ensuremath{^3}\"{A}\"{A}\"{A}\"{A}\"{A}\"{A}\"{A}\"{A}\"{A}\"{A}\"{A}\"{A}\"{A}\"{A}\"{A}\ensuremath{^3}\"{A}\"{A}\"{A}\"{A}\"{A}\"{A}\"{A}\"{A}\"{A}\"{A}\"{A}\"{A}\"{A}\"{A}\"{A}\ensuremath{^3}\"{A}\"{A}\"{A}\"{A}\"{A}\"{A}\"{A}\"{A}\"{A}\"{A}\"{A}\"{A}\"{A}\"{A}', 
\\
\&'Dista. \ensuremath{^3} ' ,Ln , ' \ensuremath{^3} ' ,La , ' \ensuremath{^3} ',el 
\\
\textbf{WRITE}(6,*)'El programa termino!' \\
\textbf{print}*, 'Si quiere continuar digie 1 si no cero' \\
\textbf{read}*, res \\
\textbf{goto }20 \\
\textbf{ENDIF \\
open}(10,file='trapecio21.txt',status='old') \\
\textbf{write}(10,200)'Resultado para N=',N,':' \\
\textbf{write}(10,201)'***************************************************' 
\\
\textbf{write}(10,202) ' Resultado de la Integral numerica para t: 
',In \\
\textbf{write}(10,203) ' Resultado de la Integral analitica para t: 
',Ia \\
\textbf{write}(10,204) ' Diferencia = ',et,'\%' \\
\textbf{write}(10,205) ' Resultado de la Integral numerica para L: 
',Ln \\
\textbf{write}(10,206) ' Resultado de la Integral analitica para L: 
',La \\
\textbf{write}(10,207) ' Diferencia = ',el,'\%' \\
\textbf{write}(10,208)'***************************************************'

200 \textbf{format}(a20,i4,a2)\\
201 \textbf{format}(a50)\\
202 \textbf{format}(a45,f8.7)\\
203 \textbf{format}(a45,f8.7)\\
204 \textbf{format}(a15,f8.7,a2)\\
205 \textbf{format}(a45,f8.7)\\
206 \textbf{format}(a45,f8.7)\\
207 \textbf{format}(a15,f8.7,a2)\\
208 \textbf{format}(a50)

 \textbf{end }

\textbf{c Declaracion de funciones \\
function }y(x) \\
\textbf{real }x \\
y= 3./(48-(12*(sqrt(3.)))-x) \\
\textbf{end }

  \textbf{function }g(z) \\
\textbf{real }z \\
g= 3*(3.3038-log(48.-(12*(sqrt(3.)))-z)) \\
\textbf{end }

 \textbf{function }R(t) \\
\textbf{real }t \\
R= (48-12*(sqrt(3.)))*(1-exp(-(t/3.))) \\
\textbf{END }

 \textbf{function }T(w) \\
\textbf{real }w \\
T= (48-12*(sqrt(3.)))*(w+3*exp(-(w/3.))-3) \\
\textbf{END }



\end{document}
