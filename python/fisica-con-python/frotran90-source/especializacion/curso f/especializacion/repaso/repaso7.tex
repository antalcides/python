%% This document created by Scientific Word (R) Version 3.5

%TCIDATA{LaTeXparent=0,0,est2.tex}
%TCIDATA{ChildDefaults=%
%chapter:5,page:51,section:5
%}


\section{}

\section{Pruebas de bondad de ajuste}

\subsection{Prueba chi cuadrado de bondad de ajuste}

Suponemos ahora un problema que puede ser caracterizado por una variable
aleatoria discreta cuyos valores representan K posibles categorias y ocurren
con probabilidades\newline $p_{k}:k=1,2,3,\cdots K$ en el cual nos interesa la
hip\'{o}tesis H$_{0}:p_{k}=p_{k_{0}};$ Siendo $p_{k_{0}}$ valores fijos,
Contra la alternativa l\'{o}gica H$_{1.}$

Como estad\'{\i}stico de prueba se escoge la llamada $\chi^{2}$%
\[
X^{2}=\sum_{k=1}^{K}\frac{\left(  N_{k}-np_{k_{0}}\right)  ^{2}}{np_{k_{0}}}%
\]
Se rechaza la hip\'{o}tesis $H_{0}$ con base en los valores concretos
$n_{1},\cdots,n_{k}$ si y s\'{o}lo si para el valos $x^{2}$ de $X^{2}$ vale
\[
x^{2}\geq c,
\]
donde $c$ es alg\'{u}n valos cr\'{\i}tico.

Ahora se usa el echo de que, bajo $H_{0}$, la estad\'{\i}stica $X^{2}%
\leadsto\chi_{K-1}^{2}$ en algunos libros se utiliza como notaci\'{o}n para el
estad\'{\i}stico $\chi^{2}$%
\[
X^{2}=\sum_{i=1}^{K}\frac{\left(  O_{k}-E_{k}\right)  ^{2}}{E_{k}}%
\]
donde $O_{k}$ es la frecuencia observada y $E_{k}$ es la frecuencia esperada.

En el caso de que la distribuci\'{o}n tenga un par\'{a}metro \ la
hip\'{o}tesis es
\[
H_{0}=p_{k}=p_{k}\left(  \theta\right)  ,k=1,2,\cdots,K
\]
siendo $\theta=\left(  \theta_{1},\cdots,\theta_{s}\right)  ,$ $s<K-1,$ un
par\'{a}meto; As\'{\i}
\[
X^{2}=\sum_{k=1}^{K}\frac{\left(  N_{k}-np_{k_{0}}\left(  \hat{\theta}\right)
\right)  ^{2}}{np_{k_{0}}\left(  \hat{\theta}\right)  }\leadsto\chi
_{K-1-s}^{2}%
\]
bajo $H_{0},$ as\'{\i} los grados de libertad se reducen exactamente un
n\'{u}mero igual al n\'{u}mero de p\'{a}rametros, y rechazariamos la
hip\'{o}tesis si $x^{2}>\chi^{2}$

\subsection{Pruebas de tablas de contingencia}

\paragraph{Prueba de independencia}

En un estudio estad\'{\i}stico es importante averiguar si dos variables de
clasificaci\'{o}n , ya sean cualitativas o cuantitativas son independientes.

En esta prueba los $n$ elementos de muestra de una poblaci\'{o}n pueden
clasificarse de acuerdo con dos criterios diferemtes. Por ello es interesante
saber si los dos m\'{e}todos de calsificaci\'{o}n son estad\'{\i}sticamente
independientes. Supongamos por ejemplo que el primer m\'{e}todo tiene $r$
niveles y que el segundo m\'{e}todo $c$ niveles . Sea $O_{ij}$ la frecuencia
observada para el nivel $i$ y para el nivel $j$ de los dos m\'{e}todos de
clasificaci\'{o}n , por lo que los datos aparecer\'{a}n como una tabla de $r$
renglones y $c$ columnas

Para La prueba de independencia se usan las siguientes hip\'{o}tesis

H$_{0}:$ Las dos variables de clasificaci\'{o}n son independientes.

H$_{1}:$ Las dos variables de clasificaci\'{o}n son dependientes.

Usandose el estad\'{\i}stico de prueba
\[
X^{2}=\sum_{i,j=1}^{r,c}\frac{\left(  O_{ij}-E_{ij}\right)  ^{2}}{E_{ij}}%
=\sum_{i,j=1}^{r,c}\frac{O_{ij}^{2}}{E_{ij}}-n
\]
Donde $p_{ij}=u_{i}v_{j}$ y
\begin{align*}
\hat{u}_{i}  &  =\frac{1}{n}\sum_{j=1}^{c}O_{ij}\\
\hat{v}_{j}  &  =\frac{1}{n}\sum_{i=1}^{r}O_{ij}\\
E_{ij}  &  =n\hat{u}\hat{v}=\frac{1}{n}\sum_{i=1}^{r}O_{ij}\sum_{j=1}%
^{c}O_{ij}%
\end{align*}
As\'{\i} tenemos que $X^{2}\leadsto\chi_{\left(  r-1\right)  \left(
c-1\right)  }^{2}$ y se rechazar\'{a} la hip\'{o}tesis si $X^{2}>\chi_{\left(
r-1\right)  \left(  c-1\right)  }^{2}$

\subsection{Anova}

\subsection{Anova con un factor}

El m\'{e}todo de ANOVA es un cr\'{\i}terio que requiere del c\'{a}lculo de dos
estimaciones independientes para $\sigma^{2},$ la varianza poblacional com\'{u}n.

estas dos estimaciones las denominaremos $S_{b}^{2},S_{w}^{2}$ ,donde
S$_{b}^{2}$ es la estimaci\'{o}n de la varianza entre las muestras y
S$_{w}^{2} $ es la estimaci\'{o}n de la varianza interior de las muestras; con
lo que resulta el estad\'{\i}stico
\[
F=\frac{S_{b}^{2}}{S_{w}^{2}}.
\]
En este caso tenemos $k$ muestras como se ilustra en la tabla
\[%
\begin{array}
[c]{ccccc}%
\bar{X}_{i} & \bar{X}_{1} & \bar{X}_{1} & \cdots & \bar{X}_{k}\\
S_{i} & S_{1} & S_{2} & \cdots & S_{k}\\
n_{i} & n_{1} & n_{2} & \cdots & n_{k}%
\end{array}
\]
Para simplificar los c\'{a}lculos suponemos que
\[
n=n_{1}=n_{2}=\cdots=n_{k},
\]
entonces hallamos $S_{w}^{2}$ la estimaci\'{o}n ponderada para $\sigma^{2}$%
\[
S_{w}^{2}=\sum_{i=1}^{k}\frac{S_{i}^{2}}{k},
\]
como esta estimaci\'{o}n se basa en $k$ muestras cada una de tama\~{n}o $n$ y
cada una tiene $\left(  n-1\right)  $ grados de libertad asociados ellas
entonces los grados de libertad $gl_{w}$ asociados a S$_{w}^{2}$ es
\[
gl_{w}=\sum_{i=1}^{k}\left(  n-1\right)  =k\left(  n-1\right)
\]
generalizando
\begin{align*}
S_{w}^{2}  &  =\frac{\sum_{i=1}^{k}\left(  n_{i}-1\right)  S_{i}^{2}}%
{\sum_{i=1}^{k}n_{i}-k}\\
gl_{w}  &  =\sum_{i=1}^{k}n_{i}-k\\
\bar{X}  &  =\sum_{i=1}^{k}\frac{\bar{X}_{i}}{k}\\
S_{b}^{2}  &  =n\sum_{i=1}^{k}\frac{\left(  \bar{X}_{i}-\bar{X}\right)  }{k-1}%
\end{align*}
donde los grados de libertad asociados a $S_{b}^{2}$ son
\[
gl_{b}=k-1
\]

\section{Modelos lineales}%

\begin{definition}
Sean Y y X variables aleatorias y supongamos que la relaci\'{o}n existente
entre ellas es
\begin{equation}
Y=\beta_0+\beta_1X+\epsilon%
\end{equation}
$donde\;\ \epsilon\sim N\left(   0,\sigma^{2}\right)   $ es un error
aleatorio, es decir
\begin{equation}
E\left(   Y|X\right)   =\beta_0+\beta_1X
\end{equation}
Notese que el modelo es lineal con relaci\'{o}n a los llamados coeficientes de
regresi\'{o}n $\beta_{0},\beta_{1},$ es decir el modelo
\begin{equation}
Y=\beta_0+\beta_1X^n+\epsilon,
\end{equation}
\end{definition} 

Donde$n\in\nz$ es tambi\'{e}n un modelo de regresi\'{o}n lineal.

En la definici\'{o}n 5.1 se establecieron 4 supuestos los cuales son

\begin{enumerate}
\item Para cada valor de $x_{i}$, las variables aleatorias $\epsilon_{i}$ se
distribuyen noermalmente

\item Para cada valos $x_{i}$, la media o valor esperado de $\epsilon_{i}$ es cero

\item Para cada valos $x_{i},$ la varianza de $\epsilon_{i}$ es contante

\item Los $\epsilon_{i}$ son independientes
\end{enumerate}

Como consecuencia de los cuatro supuestos se pueden hacer las siguientes observaciones

\begin{enumerate}
\item Los valores de $x$ son fijos

\item Los valores de los par\'{a}metros $\hat{\beta}_{0}$ y $\hat{\beta}_{1}$
son constanes, pero desconocidas, sim embargo pueden estimarse basandonos en
los cuatro supuestos y el m\'{e}todo de los m\'{\i}nimos cuadrados

\item Como el valor de $y$ para un $x$ fijo est\'{a} determinado por la
relaci\'{o}n $y=\beta_{0}+\beta_{1}x+\epsilon$. Por tanto los valores de $y$
dependeran de los valores de $\epsilon.$ Por tanto $y$ es una variable aleatoria
\end{enumerate}

Para un valor fijo de $x$, la distribuci\'{o}n muestral de $y$ es normal,
porque sus valores dependen de $\epsilon$ , y los valores de $\epsilon$ se
distribuyen normalmente. Como muestra la figura.
%TCIMACRO{\FRAME{dhFU}{8.2154cm}{4.2702cm}{0pt}{\Qcb{La distribuci\'{o}n
%muestral de $y$ para un valor fijo de $x$ tiene una media denotada por
%$\mu_{y|x},$ donde la ecuaci\'{o}n $E\left(  y|x\right)  =\beta_{0}+\beta
%_{1}x$ se llama ecuaci\'{o}n de regresi\'{o}n poblacional.}}{}{lineal.wmf}%
%{\special{ language "Scientific Word";  type "GRAPHIC";
%maintain-aspect-ratio TRUE;  display "USEDEF";  valid_file "F";
%width 8.2154cm;  height 4.2702cm;  depth 0pt;  original-width 8.2702in;
%original-height 11.694in;  cropleft "0";  croptop "1";  cropright "1";
%cropbottom "0";  filename '../Tacho/lineal.wmf';file-properties "NPEU";}}}%
%BeginExpansion
\begin{center}
\includegraphics[
natheight=11.694000in,
natwidth=8.270200in,
height=4.2702cm,
width=8.2154cm
]%
{../Tacho/lineal.wmf}%
\\
La distribuci\'{o}n muestral de $y$ para un valor fijo de $x$ tiene una media
denotada por $\mu_{y|x},$ donde la ecuaci\'{o}n $E\left(  y|x\right)
=\beta_{0}+\beta_{1}x$ se llama ecuaci\'{o}n de regresi\'{o}n poblacional.
\end{center}
%EndExpansion
Ahora determinaremos los LS-Estimadores $\hat{\beta}_{0}$ y $\hat{\beta}_{1}$
de $\beta_{0}$ y $\beta_{1}$ respectivamente

Un modelo lineal de acuerdo con la definici\'{o}n 5.1 se llama modelo lineal
simple, para este modelo sup\`{o}ngamos que tenemos $n$ pares de
observaciones, por ejemplo
\[
\left(  y_{1},x_{1}\right)  ,\left(  y_{2},x_{2}\right)  ,\left(  y_{3}%
,x_{3}\right)  ,\cdots,\left(  y_{n},x_{n}\right)  .
\]
Entonces empleamos estos datos para estimar los par\'{a}metros desconocidos
$\beta_{0}$ y $\beta_{1}$ por medio del m\'{e}todo de los m\'{\i}nimos
cuadrados explicado en el cap\'{\i}tulo 2.2.3. Por lo que
\begin{align*}
W\left(  \beta_{0},\beta_{1}\right)   &  =\sum_{i=1}^{n}e_{i}^{2}\\
&  =\sum_{i=1}^{n}\left(  y_{i}-\beta_{0}-\beta_{1}x_{i}\right)  ^{2}%
\end{align*}
Entonces los LS-Estimadores de $\beta_{0}$ y $\beta_{1}$ deben satisfacer
\begin{align*}
\frac{\partial W}{\partial\beta_{0}}\left|  _{\hat{\beta}_{0},\hat{\beta}_{1}%
}\right.   &  =-2\sum_{i=1}^{n}\left(  y_{i}-\hat{\beta}_{0}-\hat{\beta}%
_{1}x_{i}\right)  =0\\
\frac{\partial W}{\partial\beta_{1}}\left|  _{\hat{\beta}_{0},\hat{\beta}_{1}%
}\right.   &  =-2\sum_{i=1}^{n}\left(  y_{i}-\hat{\beta}_{0}-\hat{\beta}%
_{1}x_{i}\right)  x_{i}=0
\end{align*}
resolviendo este sistema de ecuaciones se obtienen $\hat{\beta}_{0}$ y
$\hat{\beta}_{1}$
\begin{align*}
\hat{\beta}_{0}  &  =\bar{y}-\hat{\beta}_{1}\bar{x}\\
\hat{\beta}_{1}  &  =\frac{\sum_{i=1}^{n}y_{i}x_{i}-\frac{\left(  \sum
_{i=1}^{n}y_{i}\right)  \left(  \sum_{i=1}^{n}x_{i}\right)  }{n}}{\sum
_{i=1}^{n}x_{i}^{2}-\frac{\left(  \sum_{i=1}^{n}x_{i}\right)  ^{2}}{n}}%
\end{align*}
Entonces el modelo de regresi\'{o}n lineal simple ajustado es
\[
\hat{y}=\hat{\beta}_{0}+\hat{\beta}_{1}x_{i}%
\]
Para simplifiar la notaci\'{o}n notaremos
\begin{align*}
S_{xx}  &  =\sum_{i=1}^{n}x_{i}^{2}-\frac{\left(  \sum_{i=1}^{n}x_{i}\right)
^{2}}{n}\\
S_{xy}  &  =\sum_{i=1}^{n}y_{i}x_{i}-\frac{\left(  \sum_{i=1}^{n}y_{i}\right)
\left(  \sum_{i=1}^{n}x_{i}\right)  }{n}%
\end{align*}
As\'{\i} $\hat{\beta}_{1}=\frac{S_{xy}}{S_{xx}}.$

\subsubsection{Estimaci\'{o}n puntual}

analizaremos ahora las propiedades de los estimadores $\hat{\beta}_{0}$ y
$\hat{\beta}_{1}$

\begin{itemize}
\item $E\left(  \hat{\beta}_{1}\right)  =E\left(  \frac{S_{xy}}{S_{xx}%
}\right)  =\beta_{1}$

\item $V\left(  \hat{\beta}_{1}\right)  =V\left(  \frac{S_{xy}}{S_{xx}%
}\right)  =\frac{\sigma^{2}}{S_{xx}}$

\item $E\left(  \hat{\beta}_{0}\right)  =\beta_{0}$

\item $V\left(  \hat{\beta}_{0}\right)  =\sigma^{2}\left(  \frac{1}{n}%
+\frac{\bar{x}^{2}}{S_{xx}}\right)  $

\item $cov=\left(  \hat{\beta}_{0},\hat{\beta}_{1}\right)  =-\frac{\sigma
^{2}\bar{x}}{S_{xx}}$
\end{itemize}

Queda como ejercicio la prueba de estas propiedades .

De acuerdo con esto observamos que $\hat{\beta}_{0}$ y $\hat{\beta}_{1}$ son
estimadores insesgados .

Ahora si notamos
\[
SS_{E}=\sum_{i=1}^{n}e_{i}^{2}=S_{yy}-\hat{\beta}_{1}S_{xy}%
\]
entonces $\hat{\sigma}^{2}=\frac{SS_{E}}{n-2}\equiv MS_{E}$ es un estimador
insesgado de $\sigma^{2}$ y $S_{yy}\equiv MS_{R}$

\subsection{Prueba de hip\'{o}tesis}

Nos queda presentar los estad\'{\i}sticos que utilizaremos para una prueba de hip\'{o}tesis

\begin{enumerate}
\item Si tenemos las hip\'{o}tesis
\begin{align*}
H_{0}  &  :\beta_{1}=\beta_{1,0}\\
H_{1}  &  :\beta_{1}\neq\beta_{1,0}%
\end{align*}
entonces se usa el estad\'{\i}stico
\[
t=\frac{\hat{\beta}_{1}-\beta_{1,0}}{\sqrt{\frac{MS_{E}}{S_{xx}}}}\leadsto
t_{\alpha/2,n-2}%
\]

\item Si
\begin{align*}
H_{0}  &  :\beta_{0}=\beta_{0,0}\\
H_{1}  &  :\beta_{0}\neq\beta_{0,0}%
\end{align*}
Se usa el estad\'{\i}stico
\[
t=\frac{\hat{\beta}_{0}-\beta_{0,0}}{\sqrt{\frac{MS_{E}}{S_{xx}}\left(
\frac{1}{n}+\frac{\bar{x}^{2}}{S_{xx}}\right)  }}\leadsto t_{\alpha/2,n-2}%
\]

\item Si
\begin{align*}
H_{0}  &  :\beta_{1}=0\\
H_{1}  &  :\beta_{1}\neq0
\end{align*}
Se usa el estad\'{\i}stico
\[
F=\frac{MS_{R}}{MS_{E}}\leadsto F_{\alpha,1,n-2}\leadsto t_{\alpha/2,n-2}^{2}%
\]
\end{enumerate}

\subsection{Intervalos de confianza}

Es posible encontrar los l\'{\i}mites para un intervalo de confianza del
$\left(  1-\alpha\right)  100\%$ para cada caso

\begin{enumerate}
\item Para $\beta_{1}$ es
\[
\hat{\beta}\pm t_{\alpha/2}\sqrt{\frac{MS_{E}}{S_{xx}}}%
\]
donde
\[
t_{\alpha/2}\leadsto t_{\alpha/2,n-2}%
\]

\item Una vez que se ha encontrado la ecuaci\'{o}n de regresi\'{o}n muestral y
que se ha determinado el modelo $E\left(  y|x\right)  =\beta_{0}+\beta_{1}x $
, podemos usar la ecuaci\'{o}n de regresi\'{o}n $\hat{y}=\hat{\beta}_{0}%
+\hat{\beta}_{1}x_{i}$ para realizar predicciones; al hacerlo queremos estimar
el valor promedio para $y$ dado $x$, es decir $E\left(  y|x\right)  .$
As\'{\i} como predecir el valor aleatorio $y$ para un valor $x$ dado.\newline
Los l\'{\i}mites del intervalo de confianza para $E\left(  y|x_{0}\right)  $
estan dados por la f\'{o}rmula
\[
\hat{y}\pm t_{\alpha/2}\sqrt{SM_{E}\left(  \frac{1}{n}+\frac{\left(
x_{0}-\bar{x}\right)  ^{2}}{S_{xx}}\right)  }%
\]
donde
\[
t_{\alpha/2}\leadsto t_{\alpha/2,n-2}%
\]

\item Los l\'{\i}mites del intervalo de confianza para un valor de un solo
valor aleatorio $y$ dado un valor particular $x=x_{0}$ est\'{a}n dados por la
expresi\'{o}n
\[
\hat{y}\pm t_{\alpha/2}\sqrt{SM_{E}\left(  1+\frac{1}{n}+\frac{\left(
x_{0}-\bar{x}\right)  ^{2}}{S_{xx}}\right)  }%
\]
donde
\[
t_{\alpha/2}\leadsto t_{\alpha/2,n-2}%
\]
\end{enumerate}

\begin{thebibliography}{9}                                                                                                %
\bibitem {}Williams Mendenhall. \emph{Estad\'{\i}stica matem\'{a}tica con
aplicaciones}\newline Grupo editorial iberoamericano . 1990

\bibitem {}Murray R . Spiegel . \emph{Estad\'{\i}stica} Mac Graw-Hill .1991

\bibitem {}Paul Meyer . \emph{Probabilidad y aplicaciones estad\'{\i}sticas}%
\newline Addison Wesley, 1992

\bibitem {}Walpole Ronald . \emph{Probabilidad y estad\'{\i}stica}\newline Mac
Graw-Hill . 1995

\bibitem {}Hines William. \emph{Probabilidad y estad\'{\i}stica}%
\newline CECSA. 1993
\end{thebibliography}