%% This document created by Scientific Word (R) Version 3.5

%TCIDATA{LaTeXparent=0,0,Est12.tex}
%TCIDATA{ChildDefaults=%
%chapter:2,page:33
%}


%\chapter{Medidas descriptivas}

\subsection{Medidas de tendencia central}

\subsection{Medias}

\subsubsection{Media aritm\'{e}tica}

La media aritm\'{e}tica de una muestra \ es la suma de cada uno de los valores
posibles multiplicado por su frecuencia, es decir.

Si la siguiente tabla \ representa la tabla de frecuencia de la muestra%

\[%
\begin{tabular}
[c]{|l|l|l|}\hline
$M$ & $n_{i}$ & $f_{i}$\\\hline
$x_{1}$ & $n_{1}$ & $f_{1}$\\\hline
$\vdots$ & $\vdots$ & $\vdots$\\\hline
$x_{k}$ & $n_{k}$ & $f_{k}$\\\hline
\end{tabular}
\]

La media es el valor :
\begin{equation}
\overset{\_}{x}=\frac{\overset{k}{\sum_{i=1}}x_{i}n_{i}}{n} \label{2-1}%
\end{equation}
y si los datos no est\'{a}n ordenados entonces $\ $%
\begin{equation}
\overset{\_}{x}=\frac{\overset{k}{\sum_{i=1}}x_{i}}{n} \label{2-2}%
\end{equation}%

\begin{remark}
En la definici\'{o}n de media se consider\'{o} que la variable de inter\'{e}s
$X$ es discreta, pero si la variable $X$ no es discreta sino continua. En la
f\'{o}rmula se reemplaza cada valor $x_{i}$ por la marca de clase
correspondiente es decir
\begin{equation}
\overset{\_}{x\,}=\frac{\overset{k}{\sum_i=1}m_in_i}{n} \label{2-3}
\end{equation}
\end{remark}

Este proceso hace que la media aritm\'{e}tica difiera de la media obtenida
seg\'{u}n (2.1), es decir habr\'{a} una perdida de precisi\'{o}n que ser\'{a}
mayor en cuanto mayor sea la diferencia entre las marcas de clase y los
valores reales, o sea entre mayor sea la longitud $a_{i}$ de los intervalos\ \ \ \ \ \ \ \ \ \ \ \ \ \ \ \ \

\subparagraph{Desventajas de la media}

La media es una medida muy usa\-da en estad\'{\i}stica, pero a pesar de eso
posee ciertas desventajas

\begin{itemize}
\item La media aritm\'{e}tica es muy sensible a los valores extremos, es decir
si una medida se aleja mucho de las otras har\'{a} que la media se aproxime
mucho a ella

\item No se recomienda usar cuando los datos se desplazan hacia los extremos

\item En el caso de variables continuas depende de los intervalos de clase

\item en le caso de variables discretas el valor puede no ser un valor de la muestra.
\end{itemize}

\subsubsection{}

Otra media es la llamada media cuadr\'{a}tica $\overset{\_}{x}_{c}$la cual es
la ra\'{\i}z cuadrada de la media aritm\'{e}tica de los cuadrados
\[
\overset{\_}{x}_{c}=\sqrt{\frac{\sum_{i=1}^{n}x_{i}^{2}}{n}}%
\]

\subsection{La mediana}

Sea $X$ una variable discreta cuyas observaciones han sido ordenadas de mayor
de mayor a menor, entonces se le llama mediana $\widetilde{x}$ al primer valor
de la variable que deja por debajo de si el 50\% de las observaciones es decir
si $n$ es el n\'{u}mero de observaciones, la mediana ser\'{a} la
observaci\'{o}n $\left[  |\frac{n}{2}|\right]  +1$%

\begin{definition}
Sea $x_{(1),}x_{(2)},x_{(3)},\cdots,x_{(n)}$ las observaciones de una muestra
para una variable $X$ donde $x_{(1)}$ representa la observaci\'{o}n m\'{a}s
peque\~{n}a, $x_{(2)}$ la observaci\'{o}n que le sigue en valor y as\'{\i}
sucesivamente $x_{(n)}$ denota la observaci\'{o}n de mayor valor, entonces la
mediana se define \[ \widetilde{x}=\left\{
\begin{tabular}
[c]{ll}
$x_{([n+1]/2)}$ & si $n$ impar\\ $\frac{x_{(n/2)}+x_{([n/2]+1)}}{2}$ & si $n$
par
\end{tabular}
\right.  \]
\end{definition}

En el caso de variables continuas, las clases vienen dadas por intervalos como
se indic\'{o} en el cap\'{\i}tulo anterior por tal raz\'{o}n para determinar
la mediana se escoge el intervalo donde se encuentra el valor para el cual
est\'{a}n debajo de \'{e}l la mitad de los datos. Entonces a partir de ese
intervalo se observan las frecuencias absolutas acumuladas y se aplica la
siguiente f\'{o}rmula
\[
\widetilde{x}=x_{i-1}+\frac{\frac{n}{2}-N_{i-1}}{n_{i}}a_{i}%
\]
de aqu\'{\i} se puede deducir que $\widetilde{x}$ el ``punto'' que divide al
histograma en dos partes de \'{a}reas iguales

\subsubsection{Propiedades y desventajas de la mediana}

\begin{enumerate}
\item Tiene la ventaja de no ser afectada por los valores extremos y por eso
se aconseja para distribuciones para las cuales los datos no se concentran en
el centro

\item Es f\'{a}cil de calcular

\item En el caso de variables discretas el valor de la mediana es un valor de
la variable

\item El mayor defecto es que las propiedades matem\'{a}ticas son muy
complicadas y esto hace que muy poco se use para realizar inferencias

\item Es funci\'{o}n de los intervalos escogidos en el caso de variables continuas
\end{enumerate}

\subsubsection{La moda}

Llamaremos moda $\widehat{x}$ a cualquier m\'{a}ximo relativo de la
distribuci\'{o}n de frecuencias, es decir, cualquier valor de la variable que
posea una frecuencia mayor que su anterior y posterior valor.

En el caso de variables continuas es m\'{a}s correcto hablar de intervalos
modales. Luego de determinar el intervalo de clase o intervalo modal,que es
aquel para el cual la distribuci\'{o}n de frecuencia posee un m\'{a}ximo
relativo, se determina la moda utilizando la siguiente f\'{o}rmula
\[
\widehat{x}=x_{i-1}+\frac{n_{i}-n_{i-1}}{\left(  n_{i}-n_{i-1}\right)
+\left(  n_{i}-n_{i+1}\right)  }a_{i}%
\]

\paragraph{Propiedades de la moda\newline }

La moda posee la siguientes propiedades

\begin{itemize}
\item Es muy f\'{a}cil de calcular

\item Puede no ser \'{u}nica

\item Es funci\'{o}n de los intervalos de su amplitud,n\'{u}mero y l\'{\i}mites
\end{itemize}

\subsubsection{Relaci\'{o}n entre la media, la moda y la mediana}

En el caso de distribuciones unimodales, la mediana est\'{a} con frecuencia
comprendida entre la media y la moda (incluso m\'{a}s cerca de la media).

En distribuciones que presentan cierta inclinaci\'{o}n, es m\'{a}s aconsejable
el uso de la mediana. Sin embargo en estudios relacionados con prop\'{o}sitos
estad\'{\i}sticos y de inferencia suele ser m\'{a}s apta la media.

Veamos un ejemplo de c\'{a}lculo de estas tres magnitudes.

\section{Medidas de posici\'{o}n}

A veces es importante obtener los valores de la variable que divi\-den la
poblaci\'{o}n en cuatro,diez o cien partes iguales,usualmente llamados
cuartiles deciles y percentiles respectivamente.

El procedimiento es similar al utilizado para determinar la mediana, como lo
indicaremos ahora.

\subsubsection{Percentil}

Para una variable discreta, se define el percentil de orden $k,$ como la
observaci\'{o}n que deja por debajo de si el $k\%$ de la poblaci\'{o}n es
decir
\begin{align*}
N_{k}  &  =n\frac{k}{100},\text{ si }n\text{ es impar}\\
\text{es decir }p_{k}  &  =x_{[|(n+1)\frac{k}{100}|]}%
\end{align*}
donde en sub \'{\i}ndice $[|(n+1)\frac{k}{100}|]$ indica que es la
posici\'{o}n $k$ a la que le corresponde ese valor de la frecuencia absoluta
acumulada. Si $n$ es par
\[
p_{k}=\frac{x_{[|n\frac{k}{100}|]}+x_{[|n\frac{k}{100}|]+1}}{2}%
\]

En el caso de variables continuas se busca el intervalo donde se encuentra
\ $p_{k},$ es decir se busca el valor que deja por debajo de si el $k\%$ de
las observaciones y se determina el intervalo $(x_{i-1,}x_{i}]$ donde se
encuentra y se utiliza la relaci\'{o}n
\[
p_{k}=x_{i-1}+\frac{n\frac{k}{100}-N_{i-1}}{n_{i}}\cdot a_{i}%
\]

\subsubsection{Quartiles}

Los cuartiles son tres y se definen:

\begin{itemize}
\item $Q_{1}=p_{25}$

\item $Q_{2}=p_{50}$

\item $Q_{3}=p_{75}$
\end{itemize}

\subsection{Deciles}

De manera an\'{a}loga se definen los deciles

los deciles son los valores que dividen las observaciones en 10 grupos de
igual tama\~{n}o es decir son el conjunto

$D_{1},D_{2},D_{3},\cdots,D_{10}$ y se definen

$D_{i}=p_{10\cdot i}\qquad i=1,2,3,\cdots,10$

\section{Medidas de variabilidad o dispersi\'{o}n}

\subsection{Varianza y desviaci\'{o}n t\'{\i}pica}

Como forma de medir la dispersi\'{o}n de los datos hemos descartado:

La varianza, $S_{n}^{2}$, se define como la media de las diferencias
cuadr\'{a}ticas de $n$ puntuaciones con respecto a su media aritm\'{e}tica, es
decir
\[
S_{n}^{2}=\frac{1}{n}\sum_{i=1}^{n}\left(  x_{i}-\overline{x}\right)  ^{2}%
\]

Para datos agrupados en tablas, usando las notaciones establecidas en el
cap\'{\i}tulo anterior, la varianza se puede escribir como
\[
S_{n}^{2}=\frac{1}{n}\sum_{i=1}^{k}\left(  x_{i}-\overline{x}\right)
^{2}\cdot n_{i}%
\]

Una f\'{o}rmula equivalente para el c\'{a}lculo de la varianza est\'{a} basada
en lo siguiente:
\[
S_{n}^{2}=\frac{1}{n}\sum_{i=1}^{n}x_{i}^{2}n_{i}-\overline{x}^{2}%
\]

La varianza no tiene la misma magnitud que las observaciones (ej. si las
observaciones se miden en metros, la varianza lo hace en $metros^{2}$ ). Si
queremos que la medida de dispersi\'{o}n sea de la misma dimensionalidad que
las observaciones bastar\'{a} con tomar su ra\'{\i}z cuadrada. Por ello se
define la desviaci\'{o}n t\'{\i}pica, $S_{n}$, como
\[
S_{n}=\sqrt{S_{n}^{2}}%
\]
