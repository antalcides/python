

\begin{theorem}
(Principio del m\'{a}ximo generalizado) Sea $L$ el operador dado en $(1),$
cuyos coeficientes satisfacen $\left( 2\right) ,\left( 3\right) $ y $\left(
4\right) $. Si $u\in W^{1,2}\left( \Omega \right) $ es una soluci\'{o}n d%
\'{e}bil de $Lu\geq 0\left( \leq 0\right) $ en $\Omega $, entonces
\begin{equation*}
\sup_{\Omega }u\leq \sup_{\partial \Omega }u^{+}\text{ \ }\left(
\inf_{\Omega }u\geq \inf_{\partial \Omega }u^{+}\right) .
\end{equation*}
\end{theorem}

%\begin{proof}
%Se define $l:=\sup\limits_{\partial \Omega }u^{+}$, y se toma $k\in \mathbb{R%
%}$ tal que $l\leq k<\sup\limits_{\Omega }u$. Entonces $v:=\left( u-k\right)
%^{+}\in W_{0}^{1,2}\left( \Omega \right) $ y $m(\mathbf{supp\ }\nabla v)\neq
%0.$ Por otro lado existe un $C>0,$ independiente de $k,$ tal que $m(\mathbf{%
%supp\ }\nabla v)>C^{-n},$ $n\geq 2,$ esto es una contradicci\'{o}n. En
%consecuencia
%\begin{equation*}
%\sup_{\Omega }u\leq l\text{.}
%\end{equation*}
%\end{proof}

\begin{corollary}
Si $u\in W_{0}^{1,2}\left( \Omega \right) $ y $Lu=0$ en $\Omega $ en el
sentido d\'{e}bil, entonces $u=0$ en $\Omega $ en el sentido d\'{e}bil.
\end{corollary}
