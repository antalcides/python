



\begin{proof}
$\left(  15\right)  $ se puede escribir de la forma
\[
\Delta u+c\left(  x\right)  u=0,
\]
donde $p:=m\left(  \alpha-1\right)  >2$, y $c\left(  x\right)  =\frac{g\left(
u\left(  x\right)  \right)  }{u\left(  x\right)  }=O\left(  \left\vert
x\right\vert ^{-p}\right)  .$ Luego de los lemas $15$ y $17a)$ se deduce que
\begin{equation}
\int f\left(  y\right)  dy>0.\tag{17}%
\end{equation}
Con eso y el lema $17b)$ resulta:
\begin{equation}
\text{existe un }\widetilde{\lambda}>0\text{ \ \ tal que \ \ }u_{1}\left(
x\right)  <0\text{ para }x_{1}\geq\widetilde{\lambda}\text{.}\tag{18}%
\end{equation}
Ahora, se hace una traslaci\'{o}n hasta un punto $\mathbf{q=}\left(
q_{1},...,q_{n}\right)  $ y se fijan las coordenadas radiales con respecto a
\'{e}ste punto (nuevo origen) de modo que
\begin{equation}
\int f\left(  y\right)  y_{j}dy=0,\text{ \ \ }j=1,...,n.\tag{19}%
\end{equation}
Luego existe un $\lambda_{0}>0$ tal que para todo $\lambda\geq\lambda_{0},$
\begin{equation}
u\left(  x\right)  >u\left(  x^{\lambda}\right)  \text{ \ \ si }x_{1}%
<\lambda\text{.}\tag{20}%
\end{equation}
De $\left(  18\right)  ,$ $\left(  20\right)  $, el lema $16$ y el lema $17c)$
resulta un intervalo m\'{a}ximal $\left(  \lambda_{1},\infty\right)  ,$
$\lambda_{1}>0,$ donde $\left(  18\right)  $ y $\left(  20\right)  $ valen.
Luego
\[
u\left(  x\right)  \geq u\left(  x^{\lambda_{1}}\right)  \text{ para\ }%
x_{1}<\lambda_{1}\text{ \ \ y \ \ }u_{1}<0\text{ para }x_{1}>\lambda
_{1}\text{.}%
\]
La simetr\'{\i}a de \ $u$ con respecto al hiperplano $x_{1}=0$ se sigue el
lema $16$ y con eso el teorema queda probado.
\end{proof}
