


%\section{Principio del M\'{a}ximo para Soluciones D\'{e}biles}

En lo siguiente $\Omega$ es un dominio acotado en $\mathbb{R}^{n}$ con
$\partial\Omega$ de clase $C^{1}$, y se considera el operador diferencial $L $
en forma de divergencia
\begin{equation}
L=\sum_{i,j=1}^{n}D_{i}\left(  a^{ij}\left(  x\right)  D_{j}+b^{i}\left(
x\right)  \right)  +\sum_{i=1}^{n}c^{i}\left(  x\right)  D_{i}+d\left(
x\right)  ,\tag{1}%
\end{equation}
donde

\begin{description}
\item[a)] $L$ es estrictamente el\'{\i}ptico, es decir, existe $\lambda_{0}>0$
tal que
\begin{equation}
\sum_{i,j=1}^{n}a^{ij}\left(  x\right)  \xi_{i}\xi_{j}\geq\lambda
_{0}\left\vert \xi\right\vert ^{2}\text{ para todo }x\in\Omega\text{ y }\xi
\in\mathbb{R}^{n}.\tag{2}%
\end{equation}


\item[b)] Los coeficientes $a^{ij},b^{i},c^{i}$ y $d$ $\left(
i,j=1,...,n\right)  $ son funciones medibles sobre $\Omega$. Adem\'{a}s son
\textbf{acotadas}, esto es existen constantes positivas $\Lambda$ y $\rho$
tales que para todo $x\in\Omega$ y $\lambda_{0}$ en $\left(  2\right)  $ se
cumple
\begin{equation}
\sum_{i,j=1}^{n}\left\vert a^{ij}\left(  x\right)  \right\vert ^{2}\leq
\Lambda,\text{ \ }\lambda_{0}^{-2}\sum_{i=1}^{n}\left(  \left\vert
b^{i}\left(  x\right)  \right\vert ^{2}+\left\vert c^{i}\left(  x\right)
\right\vert ^{2}\right)  +\lambda^{-1}\left\vert d\left(  x\right)
\right\vert \leq\rho^{2}\tag{3}%
\end{equation}


\item[c)]
\begin{equation}
\int_{\Omega}\left(  dv-\sum_{i=1}^{n}b^{i}D_{i}v\right)  dx\leq0\text{
\ }\forall v\geq0,v\in C_{0}^{1}\left(  \Omega\right)  .\tag{4}%
\end{equation}

\end{description}

\begin{definition}
Si $u\in W^{1,2}(\Omega)$ y $f\in L^{2}\left(  \Omega\right)  $, $u$ es una
soluci\'{o}n d\'{e}bil (subsoluci\'{o}n d\'{e}bil, supersoluci\'{o}n
d\'{e}bil), de
\[
Lu=f\text{ \ \ en }\Omega,
\]
si para cada $v\in C_{0}^{1}\left(  \Omega\right)  $ con $v\geq0$ se cumple
\begin{equation}
\mathcal{L}\left(  u,v\right)  =F\left(  v\right)  \text{ \ }\left(  \leq
F\left(  v\right)  ,\geq F\left(  v\right)  \right)  ,\tag{5}%
\end{equation}
donde
\begin{equation}
\mathcal{L}\left(  u,v\right)  :=\int_{\Omega}\left\{  \sum_{i,j=1}^{n}\left(
a^{ij}D_{j}u+b^{i}u\right)  D_{i}v-\left(  \sum_{i=1}^{n}c^{i}D_{i}%
u+du\right)  v\right\}  dx.\tag{6}%
\end{equation}
$\mathcal{L}$ se denomina la \textbf{forma bilineal asociada} a $L$, y
\begin{equation}
F\left(  v\right)  :=-\int_{\Omega}fvdx\text{ \ }\forall v\in C_{0}^{1}\left(
\Omega\right)  .\tag{7}%
\end{equation}

\end{definition}
