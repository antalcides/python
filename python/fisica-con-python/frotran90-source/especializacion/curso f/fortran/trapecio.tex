%% This document created by Scientific Word (R) Version 3.5

\documentclass{article}%
\usepackage{amsmath}
\usepackage{graphicx}%
\usepackage{amsfonts}%
\usepackage{amssymb}
%TCIDATA{OutputFilter=latex2.dll}
%TCIDATA{CSTFile=article.cst}
%TCIDATA{Created=Friday, October 24, 2003 07:58:30}
%TCIDATA{LastRevised=Friday, October 24, 2003 17:14:24}
%TCIDATA{<META NAME="GraphicsSave" CONTENT="32">}
%TCIDATA{<META NAME="DocumentShell" CONTENT="Standard LaTeX\Blank - Standard LaTeX Article">}
\setcounter{MaxMatrixCols}{30}
\newtheorem{theorem}{Theorem}
\newtheorem{acknowledgement}[theorem]{Acknowledgement}
\newtheorem{algorithm}[theorem]{Algorithm}
\newtheorem{axiom}[theorem]{Axiom}
\newtheorem{case}[theorem]{Case}
\newtheorem{claim}[theorem]{Claim}
\newtheorem{conclusion}[theorem]{Conclusion}
\newtheorem{condition}[theorem]{Condition}
\newtheorem{conjecture}[theorem]{Conjecture}
\newtheorem{corollary}[theorem]{Corollary}
\newtheorem{criterion}[theorem]{Criterion}
\newtheorem{definition}[theorem]{Definition}
\newtheorem{example}[theorem]{Example}
\newtheorem{exercise}[theorem]{Exercise}
\newtheorem{lemma}[theorem]{Lemma}
\newtheorem{notation}[theorem]{Notation}
\newtheorem{problem}[theorem]{Problem}
\newtheorem{proposition}[theorem]{Proposition}
\newtheorem{remark}[theorem]{Remark}
\newtheorem{solution}[theorem]{Solution}
\newtheorem{summary}[theorem]{Summary}
\newenvironment{proof}[1][Proof]{\noindent\textbf{#1.} }{\ \rule{0.5em}{0.5em}}

\begin{document}
IRINA RUIZ Y ANTALCIDES OLIVO\bigskip

\section{METODOS DE INTEGRACI\'{O}N NUMERICA}

\subsection{LA REGLA DEL TRAPECIO}

El m\'{e}todo de Euler aproxima la derivada en $\left[  x_{n},x_{n+1}\right]
$ por una constante, concretamente por su valor en $x_{n}$. \textquestiondown
Por qu\'{e} privilegiar al punto $x_{n}?$ \textquestiondown No ser\'{a} mejor
tomar , por ejemplo, como aproximaci\'{o}n constante de la derivada el
promedio de sus valores en los extremos del intervalo? En ese caso
\[
y(x_{n+1})=y\left(  x_{n}\right)  +h\frac{f\left(  x_{n},y\left(
x_{n}\right)  \right)  f\left(  x_{n+1},y\left(  x_{n+1}\right)  \right)  }%
{2}+R_{n}%
\]
donde $R_{n}$ es el error de truncaci\'{o}n, por lo que se aproxima de la forma.%

\[
y_{n+1}\approx y_{n}+h\frac{f\left(  x_{n},y\left(  x_{n}\right)  \right)
f\left(  x_{n+1},y\left(  x_{n+1}\right)  \right)  }{2}%
\]

Esta aproximaci\'{o}n conduce a la llamada regla del trapecio

la cual \ es convergente de grado dos, es decir%
\[
\left\Vert e_{n}\right\Vert \leq e^{\left(  b-a\right)  L/\left(  1-\frac
{hL}{2}\right)  }\left\Vert e_{0}\right\Vert +\frac{Ch^{2}}{L}\left(
e^{\left(  b-a\right)  L/\left(  1-\frac{hL}{2}\right)  }-1\right)  ,\text{ si
}0\leq n\leq N
\]%
\[
\text{donde }e_{n}=y\left(  x_{n}\right)  -y_{n},\ \frac{hL}{2}<1,\ C=\frac
{5}{12}\max_{x\in\left[  a,b\right]  }\left\Vert y^{\prime\prime\prime}\left(
x\right)  \right\Vert \
\]

a pesar de que la regla del trapecio tiene un orden de convergencia mayor que
el m\'{e}todo de Euler. eso no significa que sea un m\'{e}todo mejor , pues
hay otra diferencia importante entre ambos m\'{e}todos. el m\'{e}todo de Euler
es \textit{expl\'{\i}cito,} el valor de $y_{n+1}$ viene dado
expl\'{\i}citamente en t\'{e}rmino del valor anterior $y_{n}$ y se puede
calcular f\'{a}cilmente mediante la evaluaci\'{o}n de $f$ y una pocas
operaciones aritm\'{e}ticas. por el contrario, la regla del trapecio es un
m\'{e}todo \textit{impl\'{\i}cito}, para calcular $y_{n+1}$ hay que resolver
un sistema de ecuaciones no lineales, lo que en general es computacionalmente
costoso. de lo que se deduce, el orden no lo es todo. Aunque en un m\'{e}todo
de orden alto en principio hay que dar menos pasos, estos pueden ser muy
costosos, por lo que puede ser mejor un m\'{e}todo de menos orden en el que a
pesar de dar m\'{a}s pasos cada uno de ellos lleve menos trabajo computacional.

\subsection{Aplicaciones}

Para aplicar este m\'{e}todo resolveremos dos problemas de f\'{\i}sica
m\'{e}canica anal\'{\i}ticamente y usando Fortran lo resolveremos
computacionalmente y luego calculamos el error.

\begin{enumerate}
\item un objeto que pesa $48\ lb$ se suelta desde el reposo en la parte
superior de un plano inclinado met\'{a}lico que tiene una inclinaci\'{o}n de
$30%
%TCIMACRO{\U{ba}}%
%BeginExpansion
{{}^o}%
%EndExpansion
$ respecto a la horizontal. la resistencia del aire es num\'{e}ricamente igual
a un medio de la velocidad (en pies por segundo), el coeficiente de rozamiento
es $\frac{1}{4}$

\begin{enumerate}
\item \textquestiondown Cu\'{a}l es la velocidad del objeto dos segundos
despu\'{e}s de haberse soltado?

\item Si el plano mide mide $24~ft$ de longitud, \textquestiondown C\'{u}al es
la velocidad del cuerpo en el momento que llega al punto inferior?%
%TCIMACRO{\FRAME{dhF}{2.4362in}{1.6077in}{0pt}{}{}{fortran.jpg}%
%{\special{ language "Scientific Word";  type "GRAPHIC";
%maintain-aspect-ratio TRUE;  display "USEDEF";  valid_file "F";
%width 2.4362in;  height 1.6077in;  depth 0pt;  original-width 2.7268in;
%original-height 1.7902in;  cropleft "0";  croptop "1";  cropright "1";
%cropbottom "0";  filename 'fortran.jpg';file-properties "XNPEU";}}}%
%BeginExpansion
\begin{center}
\includegraphics[
natheight=1.790200in,
natwidth=2.726800in,
height=1.6077in,
width=2.4362in
]%
{fortran.jpg}%
\end{center}
%EndExpansion
\end{enumerate}
\end{enumerate}

\begin{solution}
La l\'{\i}nea de movimiento es a lo largo del plano, si escogemos el origen
del sistema de referencia en la parte superior y el sentido positivo de las
$x$ hacia abajo del plano, entonces las%
%TCIMACRO{\FRAME{dhF}{2.2053in}{1.6725in}{0pt}{}{}{plano.jpg}%
%{\special{ language "Scientific Word";  type "GRAPHIC";
%maintain-aspect-ratio TRUE;  display "USEDEF";  valid_file "F";
%width 2.2053in;  height 1.6725in;  depth 0pt;  original-width 2.8098in;
%original-height 2.1231in;  cropleft "0";  croptop "1";  cropright "1";
%cropbottom "0";  filename 'plano.jpg';file-properties "XNPEU";}}}%
%BeginExpansion
\begin{center}
\includegraphics[
natheight=2.123100in,
natwidth=2.809800in,
height=1.6725in,
width=2.2053in
]%
{plano.jpg}%
\end{center}
%EndExpansion
fuerzas que act\'{u}na sobre el objeto A son:\newline Su peso de 48 lb que
act\'{u}a verticalmente hacia abajo.\newline La fuerza normal, N que ejerce el
plano sobre el objeto la cual act\'{u}a en la direcci\'{o}n positiva y
perpendicular al plano.\newline La resitencia del aire $f_{a}$,que tiene un
valor num\'{e}rico igual a $\frac{v}{2}$, puesto que $v>0$ esta fuerza tiene
una direcci\'{o}n negativa en $x.$\newline La fuerza de rozamiento, que tiene
un valor $\mu N$ y tiene una direcci\'{o}n negativa en $x.$\newline de la
gr\'{a}fica 2 y aplicando la segunda ley de Newton obtenemos
\begin{align*}
\text{ en }x &  :\ \ ma\ =Wsen30%
%TCIMACRO{\U{ba}}%
%BeginExpansion
{{}^o}%
%EndExpansion
-f_{s}-f_{r}\\
f_{r} &  =\mu N\\
\text{en }y &  :0=N-W\cos30%
%TCIMACRO{\U{ba}}%
%BeginExpansion
{{}^o}%
%EndExpansion
\end{align*}
reemplazando los valores y resolviendo el sistema de las tres ecuaciones
anteriores se obtiene:%
\[
\frac{3}{2}\frac{dv}{dt}=24-6\sqrt{3}-\frac{1}{2}v
\]
al separar variables se obtiene%
\[
\frac{dv}{48-12\sqrt{3}-v}=\frac{dt}{3}%
\]
como cuando $t=0,\ v=0$ se tiene%
\begin{align}
\int_{0}^{v_{f}}\frac{dv}{48-12\sqrt{3}-v} &  =\frac{dv}{48-12\sqrt{3}-v}%
=\int_{0}^{2}\frac{dt}{3}\label{ec1}\\
v &  =\left(  48-12\sqrt{3}\right)  \left(  1-e^{-\frac{t}{3}}\right)
\nonumber
\end{align}
de modo que al resolver la integral nos da
\[
v\left(  2\right)  =10.2\left(  ft/s\right)
\]
Para la segunda parte se integra \ref{ec1} y como $x\left(  0\right)  =0$%
\[
x=\left(  48-12\sqrt{3}\right)  \left(  t+3e^{-\frac{t}{3}}-3\right)
\]
despejando $t$ y reemplazando en \ref{ec1} se obtiene
\[
v=12.3~ft/s
\]
\end{solution}

CODIGO EN FORTRAN
\begin{verbatim}
c     Este programa utiliza la regla del trapecio
c     para resolver el movimiento de un cuerpo
c     sobre un plano inclinado considerando la
c     resistencia del aire y la fricciTCIMACRO{\U{a2}}%BeginExpansion\hbox{\rm\rlap/c}EndExpansionn
      program trapecio
      real a, b,d,suma,It,t,v
      integer i,n
      write(*,*)'***************************************************'
      write(*,*)'Integracion numerica'
      write(*,*)'regla del Trapecio'
      write(*,*)'***********************'
      write(*,*) 'Ingrese el limite inferior entre 0 y L'
      read(*,*) a
      write(*,*) 'Ingrese el limite superior entre 0 y L'
      read(*,*) b
      write(*,*) 'Ingrese el numero N de puntos en [a,b], N es natural'
      read(*,*) n
      d=(b-a)/2
      suma=0
      do 2  i=1,n
      suma=suma+y(i)
2     continue
      It=d*(Y(a)+(2.*suma)+Y(b))
      write(*,*) 'Ingrese el valor de t'
      read(*,*) t
      v=(48.-12.*sqrt(3.))*(1.-exp(-t/3.))
      print*,' La velocidad es v=',v,' ft/s'
      write(*,*)'Resultado para N=',N,':'
      write(*,*)'**************************************************'
      write(*,*) ' Resulatado de la Integral numerica',It
      write(*,*)'*************************************************'
      end
      
c      Declaracion de funciones
      function y(x)
      real x
      y= 1./(48-(12*(sqrt(3.)))-x)
      end
      
      

      
\end{verbatim}

\section{DETERMINACION DE RAICES}

\subsection{METODO DE BISECCION}

\bigskip

Si f es una funci\'{o}n continua en el intervalo $\left[  a.b\right]  $ y si
$f\left(  a\right)  f\left(  b\right)  <0$, entonces $f$ debe tener un cero en
$\left(  a,b\right)  .$ Esta es una consecuencia del teorema del valor
intermedio para funciones continuas.

El m\'{e}todo de bisecci\'{o}n explota esta idea de la siguiente manera , si
$f\left(  a\right)  f\left(  b\right)  <0,$ se calcula $c=1/2(a+b)$ y se
averigua si $f\left(  a\right)  f\left(  c\right)  <0,$ si lo es, entonces $f$
tiene un cero en $\left[  a,c\right]  .$ A continuaci\'{o}n se renombra $c$
como $b$ y se comienza una vez mas con elintervalo $\left[  a,b\right]  ,$
cuya longitud es igual a la mitad de la longitud del intervalo original.

Si $f\left(  a\right)  f\left(  c\right)  >0,$ entoonces $f\left(  a\right)
f\left(  b\right)  <0,$y en este caso se renombra $c$ como $a$ , en ambos
casos se ha generado un nuevo intervalo que contiene un cero de $f$ y el
proceso puede repetirse.Claro est\'{a} que si $f\left(  a\right)  f\left(
c\right)  =0,$ entonces $f\left(  c\right)  =0$ y con ello se ha encontrado un
cero . Sin embargo por los errores de redondeo es poco factible que $f\left(
c\right)  =0,$ as\'{\i} el criterio para concluir no deber\'{a} depender de
que $\ f\left(  c\right)  $ sea cero . Se debe permitir una tolerancia
razonable tal como $f\left(  c\right)  <10^{-5}.$ Si hay varios ceros en el
intervalo dado, el m\'{e}todo de la bisecci\'{o}n encuentra uno cada vez. Este
m\'{e}todo tambi\'{e}n se conoce como el m\'{e}todo de la bipartici\'{o}n.

\bigskip

A la hora de programar es conveniente contar con varios criterios que detengan
el programa, uno es m\'{a}ximo n\'{u}mero de pasos que se permitiran, esto
reduce las posibilidades de que el programa se quede en un ciclo infinito. Por
otra parte la ejecuci\'{o}n del programa se puede detener ya sea cuando el
error sea lo suficientemente peque\~{n}o o cuando lo sea el valor de $f\left(
c\right)  .$

\bigskip

ANALISIS DE ERRORES

\bigskip Si $\left[  a_{0},b_{0}\right]  ,\left[  a_{1},b_{1}\right]  $,
...,$\left[  a_{n},b_{n}\right]  $ son los intervalos que resultan del
proceso, se pueden hacer las siguientes observaciones:%

\begin{align*}
a_{0}  & \leqslant a_{1}\leqslant a_{2}\leqslant...\leqslant a_{n}\text{
}\left(  1\right) \\
b_{0}  & \geqslant b_{1}\geqslant b_{2}\geqslant...\geqslant b_{1}\text{
}\left(  2\right) \\
b_{n+1}-a_{n+1}  & =\frac{1}{2}\left(  b_{n}-a_{n}\right)  \text{ \ \ }\left(
n\geqslant0\right)  \text{ }\left(  3\right)
\end{align*}

La sucesi\'{o}n $\left[  a_{n}\right]  $ converge debido a que es creciente y
est\'{a} acotada superiormente , la sucesi\'{o}n $\left[  b_{n}\right]  $
tambi\'{e}n converge por razones analogas .Si se utiliza $\left(  3\right)  $
repetidamente se llega a que :%

\[
b_{n}-a_{n}=2^{-n}\left(  b_{0}-a_{0}\right)
\]

As\'{\i}%

\[
\lim_{n\rightarrow\infty}b_{n}-\lim_{n\rightarrow\infty}a_{n}=\lim
_{n\rightarrow\infty}2^{n}\left(  b_{0}-a_{0}\right)  =0
\]

Si se escribe%

\[
r=\lim_{n\rightarrow\infty}b_{n}=\lim_{n\rightarrow\infty}a_{n}%
\]

entonces tomando l\'{\i}mite en la desigualdad $0\geqslant f\left(
a_{n}\right)  f\left(  b_{n}\right)  ,$ entonces se obtiene $\ 0\geqslant
\left[  f\left(  r\right)  \right]  ^{2},$ y por tanto $f\left(  r\right)  =0.$

\bigskip

Sup\'{o}ngase que en cierta etapa del proceso \ se ha definido el intervalo
$\left[  a_{n},b_{n}\right]  .$ Si se detiene el proceso en este momento, la
raiz se encontrar\'{a} en este intervalo, en esta etapa la mejor
estimaci\'{o}n de la raiz es el punto medio del intervalo.

El error se acota de la siguiente forma:%

\[
\left|  r-c_{n}\right|  \leq\frac{1}{2}\left|  b_{n}-a_{n}\right|
\leq2^{-\left(  n+1\right)  }\left(  b_{0}-a_{0}\right)
\]

Resumiendo lo anterior \ se tiene el siguiente

TEOREMA:

Si $\left[  a_{0},b_{0}\right]  ,\left[  a_{1},b_{1}\right]  $, ...,$\left[
a_{n},b_{n}\right]  ,$..., denotan los intervalos en el m\'{e}todo de la
bisecci\'{o}n, entonces los l\'{\i}mites $\lim_{n\rightarrow\infty}b_{n}$ y
$\lim_{n\rightarrow\infty}a_{n},$ existen, son iguales y representan un cero
de $f.$ Si $r=\lim_{n\rightarrow\infty}c_{n}$ y $c_{n}=\frac{1}{2}\left(
a_{n}+b_{n}\right)  $, entonces:
\begin{verbatim}
\begin{equation*}
\left| r-c_{n}\right| \leq 2^{-\left( n+1\right) }\left( b_{0}-a_{0}\right)
\end{equation*}
\end{verbatim}

En el siguiente ejemplo muestra un programa que resuelve una ecuaci\'{o}n
cuadr\'{a}tica introducida por un usuario.
\begin{verbatim}
c     **Este programa halla las raices de una ecuaci\UNICODE{0xa2}n cuadrtica por el
c     mtodo de la secante**
      program biseccion
     
c     **Declaraci\UNICODE{0xa2}n de variables**

      real G,H,I,D,R,u,v,w,x,a,b,c,e,dis
      integer n,j,resp,k
      D=0.00001
      R=0.00001
      n=300
      resp=1
      do while(resp.EQ.1)
        print*, 'Este programa resuelve una ecuaci\UNICODE{0xa2}n cuadrtica:'
        print*, 'ax**2 + bx  +c '
        Print*, 'Digite el coeficiente a'
        read*, G
        Print*, 'Digite el coeficiente b'
        read*, H
        Print*, 'Digite  c'
        read*, I
        dis=(H**2)-(4*G*I)
        if(dis.GT.0)then
         Print*, 'Digite los l\UNICODE{0xa1}mites del intervalo'
         read*, a
         read*, b

         u= (G*(a**2))+(H*a)+I
         v= (G*(b**2))+(H*b)+I
         if(u.EQ.0)then
          print*,'La raiz es',a
         else
          if(v.EQ.0)then
           print*,'La raiz es',b
          else
            k=1
            do while((v*u.GE.0).and.(k.LT.30))
             k=k+1
             a=a+0.1
             u= (G*(a**2))+(H*a)+I
             v= (G*(b**2))+(H*b)+I
            enddo
            e=b-a
            j=1
            if((u*v).LT.0)then
               do while(j.LT.n)
                 j=j+1
                 e=e/2
                 c=a+e
                 w=(G*(c**2))+(H*c)+I

                 if(abs(e).GE.D)then
                   if(abs(w).GE.R)then
                     if((u*w).LT.0)then
                      b=c
                      v=w
                     else
                      a=c
                      u=w
                     endif
                   else
                     j=n
                   endif
                 else
                 j=n
                 endif
               enddo
               print*, 'La raiz es',c

            else
               print*, 'No existen raices en el intervalo dado,'
               print*,'o el la longitud del intervalo es muy grande'
            endif
           endif
          endif
        else
         print*,'Esta ecuaci\UNICODE{0xa2}n no tiene raices reales'
        endif
        print*,'Si desea continuar digite 1, sino digite 0'
        read*,resp
      enddo


      end
\end{verbatim}
\end{document}