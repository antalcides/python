\documentstyle{article}
\newlength{\defaultparindent}
\setlength{\defaultparindent}{\parindent}
 
%
%*****************************************************************%
\begin{document}
c     Este programa utiliza la regla del trapecio para

c     determinar el tiempo que tarda un cuerpo en deslizarse

c     desde la parte mas alta de un plano partiendo del reposo

c     hasta alcanzar una velocidad v.

c     En este problema se considera la resistencia del aire y 
la

c     friccion del plano sobre el cuerpo

      program trapeciom

      real v0,vf,d,suma,h,Tn,Ta,et,X,sum,Ln,La,el,e,Y

      integer i,j,n,res

        res =1

30     if (res.eq.1)then

      write(*,*)'******************************************************'

      write(6,*)'Integracion numerica'

      write(6,*)'Regla del Trapecio'

      write(6,*)'******************************************************'

      write(6,*)'Ingrese el valor de v0 entre (0 , 27.215)'

      read(*,*) v0

      write(6,*)'Ingrese el valor de vf mayor que v0 y menor 
que 27.215'

      read(*,*) vf

      write(6,*)'Ingrese el tama?o de la particion N,  N menor 
de 60'

      read(*,*) n

c     Empieza el algoritmo para calcular la integral usando N 
subintervalos

      d=(vf-v0)/n

      suma=F(vf)+F(v0)

      X=v0

      do 10 i=2,n

      X=X+d

      suma=suma+2*F(X)

10    continue

      Tn=(d/2.)*suma

c     Ahora calculamos la distancia recorrida



      h=Tn/n

      sum=R(Tn)+R(0)

      Y=0

      do 20 j=2,n

      Y=Y+h

      sum=sum+2*R(Y)

20    continue

      Ln=(h/2.)*sum

      Ta=g(vf)-g(v0)

      et=abs((Tn-Ta)/Ta)*100

      La=P(Tn)-P(0)

      el=abs((Ln-La)/La)*100

      e=s(vf)/(12*n**2)

      write(6,*)'***************************************************'

      write(6,*)'Resultado para N=',N,':'

      WRITE(*,'(3(T3,A/),3(T3,A/T3,(3(A,1PE13.6))/))')

     \&'              ? Valor numerico? Valor analit. ?      
        ',

     \&'              ? estimado de la? de la integral?   Diferencia 
',

     \&'              ? integral      ?               ?   en 
\%       ',

     \&'?????????????????????????????????????????????????????????????',

     \&'Tiempo        ? ' ,Tn    ,  ' ? ' ,Ta    ,  ' ? ',et 
         ,

     \&'?????????????????????????????????????????????????????????????',

     \&'Dista.        ? ' ,Ln    ,  ' ? ' ,La    ,  ' ? ',el

      write(6,*) ' Error = ',e

      write(6,*)'***************************************************'

      WRITE(6,*)'El programa termino!'

       print*, 'Si quiere continuar digie 1 si no cero'

       read*,  res

       goto 30

      ENDIF

      open(10,file='trapeciom.txt',status='old')

      write(10,200)'Resultado para N=',N,':'

      write(10,201)'***************************************************'

      write(10,202) ' Resultado de la Integral numerica para 
t: ',Tn

      write(10,203) ' Resultado de la Integral analitica para 
t: ',Ta

      write(10,204) ' Diferencia = ',et,'\%'

      write(10,205) ' Resultado de la Integral numerica para 
L: ',Ln

      write(10,206) ' Resultado de la Integral analitica para 
L: ',La

      write(10,207) ' Diferencia = ',el,'\%'

      write(10,208) ' Error = ',e

      write(10,209)'***************************************************'





200   format(a20,i4,a2)

201   format(a50)

202   format(a45,f8.6)

203   format(a45,f8.6)

204   format(a15,f8.6,a2)

205   format(a45,f8.6)

206   format(a45,f8.6)

207   format(a15,f8.6,a2)

208   format(a10,D8.3)

209   format(a50)



      end

      

      

      

      

      

      

c           Declaracion de funciones

      function F(x)

      real x

      F= 3./(48-(12*(sqrt(3.)))-x)

      end





      function g(x)

      real x

      g= 3*(3.3038-log(48.-(12*(sqrt(3.)))-x))

      end



      function R(x)

      real x

      R= (48.-12.*(sqrt(3.)))*(1.-exp(-(x/3.)))

      END



      function P(x)

      real x

      P= (48.-12.*(sqrt(3.)))*(x+3.*exp(-(x/3.))-3.)

      END

      

      function s(x)

      real x

      s=6./(48.-(12.*(sqrt(3.)))-x)**3

      end



      

      

\end{document}